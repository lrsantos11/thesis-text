\begin{center}
  \large{\textbf{Abstract}}
\end{center}

\selectlanguage{english}
% FIXME Delete lines from this one until 14th.



In this work we propose a predictor-corrector interior point method for linear programming in a primal-dual context, in which the next iterate will be chosen by the minimization of a polynomial merit function  of three variables-parameters: the first one is the steplenght, the second one defines the central path and the last one models the weight that a corrector direction should have. The merit function minimization is performed by subjecting it to constraints defined by a neighborhood of the central path that allows wide steps. In this framework, we combine  different directions, such as predictor, corrector or centering ones, to produce a better direction. Our method generalizes most of predictor-corrector interior point methods, depending on the choice of the variables-parameters described above. Convergence analysis of the method is carried out, by considering an initial point that has good practical performance, which results in Q-linear convergence of the iterates in polynomial complexity. Numerical experiments, using Netlib set test,  showed that this approach is competitive when compared to more well established solvers, such as PCx.




% In this work we solve linear optimization problems on an Interior Point Methods environment by combining different directions, such as predictor, corrector or centering ones, to produce a better direction. We measure how good the new direction is by using a polynomial merit function on three variables. One of them is the step length, the other defines the central path and the last one models the weight that a corrector directions should have in a predictor-corrector method. Some numerical tests show that this approach is competitive when compared to more well established solvers as PCx, using the Netlib test set.

\vspace{.5cm}
\textbf{Keywords}:
% FIXME Remover a linha abaixo.
Linear Programming, Interior Point Methods.
% TODO Inserir as palavras-chave em inglês aqui.
\selectlanguage{brazilian}
