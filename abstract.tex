\begin{center}
  \large{\textbf{Abstract}}
\end{center}

\selectlanguage{english}
% FIXME Delete lines from this one until 14th.



In this work we propose a predictor-corrector interior point method for linear programming in a primal-dual context, where the next iterate is chosen by the minimization of a polynomial merit function  of three variables: the first one is the step length, the second one defines the central path and the last one models the weight that a corrector direction must have. The merit function minimization is performed by restricting it to constraints defined by a neighborhood of the central path that allows wide steps. In this framework, we combine  different directions, such as the predictor, the corrector and the centering directions, with the aim of producing a better direction. The proposed method generalizes most of predictor-corrector interior point methods, depending on the choice of the variables described above. Convergence analysis of the method is carried out,  considering an initial point that has a good practical performance, which results in Q-linear convergence of the iterates with polynomial complexity. Numerical experiments are made, using the Netlib test set, which show that this approach is competitive when compared to  well established solvers, such as PCx.





% In this work we solve linear optimization problems on an Interior Point Methods environment by combining different directions, such as predictor, corrector or centering ones, to produce a better direction. We measure how good the new direction is by using a polynomial merit function on three variables. One of them is the step length, the other defines the central path and the last one models the weight that a corrector directions should have in a predictor-corrector method. Some numerical tests show that this approach is competitive when compared to more well established solvers as PCx, using the Netlib test set.

\vfill
\textbf{Keywords}:
% FIXME Remover a linha abaixo.
Linear Programming, Interior Point Methods,  Algorithm analysis.
% TODO Inserir as palavras-chave em inglês aqui.
\selectlanguage{brazilian}
