%!TEX root = tese.tex
%%%%%%%%%%%%%%%%%%%%%%%%%%%%%% Pacotes: básicos %%%%%%%%%%%%%%%%%%%%%%%%%%%%%% 
\usepackage[utf8]{inputenx}
\usepackage{cmap}
\usepackage[T1]{fontenc}
 \usepackage{ae}
\usepackage[english,brazilian]{babel}
\usepackage{indentfirst}
\usepackage[top=3cm,bottom=3cm,outer=2cm,inner=2cm]{geometry}
\usepackage[backend=biber,sortcites=true,hyperref=true,maxbibnames=9,maxcitenames=3,sorting=nyt]{biblatex}
\usepackage{csquotes}
% FIXME Para compatibilidade de alguns pacotes com Koma-Script
\usepackage{scrhack} 


%%%%%%%%%%%%%%%%%%%%%%%%%%%%%%% Pacotes: layoyt %%%%%%%%%%%%%%%%%%%%%%%%%%%%%%%
\usepackage{etoolbox}  % É preciso para mudar o layout do frontmatter


%%%%%%%%%%%%%%%%%%%%%%%%%%%%%%% Pacotes: links %%%%%%%%%%%%%%%%%%%%%%%%%%%%%%%
\usepackage{url}
\usepackage{hyperref}
% FIXME breakurl sempre depois de hhyperref. Use apenas se não estiver gerando 
% via PDFLATEX
% \usepackage{breakurl}
% FIXME Comente o pacote abaixo quando for concluir sua defesa e for entregar a
% versão final.
 % \usepackage[]{showkeys}
 % \renewcommand*\showkeyslabelformat[1]{%
    % \fbox{\parbox[t]{\marginparwidth}{\raggedright\normalfont\small\ttfamily#1}}}

%%%%%%%%%%%%%%%%%%%%%%%%%%%%%%%% Pacotes: ams %%%%%%%%%%%%%%%%%%%%%%%%%%%%%%%% 
% \usepackage{amsmath}
% \usepackage{amsfonts}
% \usepackage{amssymb}
% \usepackage{amsthm}
% \usepackage{breqn}


%%%%%%%%%%%%%%%%%%%%%%%%%%%%%% Pacotes: tabelas %%%%%%%%%%%%%%%%%%%%%%%%%%%%%%
 \usepackage{multicol}
 \usepackage{multirow}
 \usepackage{array}

\usepackage{booktabs}
\usepackage{longtable}

%%%%%%%%%%%%%%%%%%%%%%%%%%%%%% Pacotes: cores %%%%%%%%%%%%%%%%%%%%%%%%%%%%%%%% 
\usepackage[usenames,dvipsnames,svgnames,table]{xcolor}


%%%%%%%%%%%%%%%%%%%%%%%%%%%%%% Pacotes: figuras %%%%%%%%%%%%%%%%%%%%%%%%%%%%%% 
\usepackage{pdfpages}
\usepackage{graphicx}
% \usepackage{wrapfig}
% \usepackage{tikz}
% \usetikzlibrary{fit}


%%%%%%%%%%%%%%%%%%%%%%%%%%%%% Pacotes: algoritmos %%%%%%%%%%%%%%%%%%%%%%%%%%%%% 
\usepackage{algorithmicx}
\usepackage[chapter]{algorithm}
\usepackage{algpseudocode}
\floatname{algorithm}{Algoritmo}
\renewcommand{\listalgorithmname}{Lista de Algoritmos}
 \algrenewcommand{\algorithmicrequire}{\textbf{Dado:}}
% \algrenewcommand{\algorithmicreturn}{\textbf{Sa\' ida:}}
 \algrenewcommand{\algorithmicend}{\textbf{Fim}}
% \algrenewcommand{\algorithmicif}{\textbf{Se}}
% \algrenewcommand{\algorithmicthen}{\textbf{Ent\~ao}}
% \algrenewcommand{\algorithmicelse}{\textbf{Sen\~ao}}
% \algrenewcommand{\algorithmicelsif}{\algorithmicelse\ \algorithmicif}
% \algrenewcommand{\algorithmicendif}{\algorithmicend\ \algorithmicif}
\algrenewcommand{\algorithmicfor}{\textbf{para}}
\algrenewcommand{\algorithmicforall}{\textbf{para todo}}
\algrenewcommand{\algorithmicdo}{\textbf{fa\c{c}a}}
% \algrenewcommand{\algorithmicendfor}{\algorithmicend\ \algorithmicfor}
% \algrenewcommand{\algorithmicwhile}{\textbf{Enquanto}}
% \algrenewcommand{\algorithmicendwhile}{\algorithmicend\ \algorithmicwhile}
% \algrenewcommand{\algorithmicloop}{\textbf{loop}}
% \algrenewcommand{\algorithmicendloop}{\algorithmicend\ \algorithmicloop}
 \algrenewcommand{\algorithmicrepeat}{\textbf{Repita}}
 \algrenewcommand{\algorithmicuntil}{\textbf{At\'e}}
%\algrenewcommand{\algorithmicprint}{\textbf{Imprima}}
%\algrenewcommand{\algorithmicreturn}{\textbf{Retorna}}
%\algrenewcommand{\algorithmictrue}{\true}
%\algrenewcommand{\algorithmicfalse}{\false}
\algrenewcommand{\algorithmicprocedure}{\textbf{Fun\c{c}\~{a}o}}





%%%%%%%%%%%%%%%%%%%%%%%%%%%%%% Pacotes: códigos %%%%%%%%%%%%%%%%%%%%%%%%%%%%%% 
% \usepackage{textcomp}
% \usepackage{listings}
% \renewcommand\lstlistingname{Código}
% \renewcommand\lstlistlistingname{Lista de Códigos}


%%%%%%%%%%%%%%%%%%%%%%%%%%%%%%% Pacotes: index %%%%%%%%%%%%%%%%%%%%%%%%%%%%%%% 
% \usepackage{makeidx}
% \makeindex


%%%%%%%%%%%%%%%%%%%%%%%%%%%%%%% Pacotes: fontes %%%%%%%%%%%%%%%%%%%%%%%%%%%%%% 
\usepackage{lmodern} \normalfont
 \DeclareFontShape{T1}{lmr}{bx}{sc} { <-> ssub * cmr/bx/sc }{}
 \usepackage{mathrsfs}


%%%%%%%%%%%%%%%%%%%%%%%%%%%%%%% Pacotes: pessoal %%%%%%%%%%%%%%%%%%%%%%%%%%%%%% 

\usepackage{enumerate} %Controle melhor do env enumerate
\usepackage{acronym}
\usepackage{lrsmath,lrsthm}
\usepackage[output-decimal-marker={,}]{siunitx}
\usepackage{chngcntr}
\counterwithout{table}{chapter}
\counterwithout{table}{section}



%use as the last one to correct enviroments
 \usepackage[pagewise,mathlines]{lineno}
 \linenumbers
 % \let\oldalign\align{}\def\align{\par\vspace{-\parskip}\oldalign}