%!TEX root = tese.tex
\chapter{Experimentos Numéricos}
\label{chap:numerical}

Neste capítulo descrevemos alguns detalhes de implementação que utilizamos em nossos experimento numéricos, bem os resultados de nosso algoritmo, comparando-o com o  PCx~\cite{Czyzyk:1999hk}, uma implementação do método preditor-corretor de \textcite{Mehrotra:1992wr} que inclui correções múltiplas de \textcite{Gondzio:1996uw}. 



\section{Detalhes de Implementação}

O Algoritmo \ref{alg:optimized-choice-of-parameters} foi implementado em  \texttt{C/C++}, utilizando a \emph{framework} que o PCx disponibiliza. Isso significa, que todas as rotinas de álgebra linear, ponto inicial, critério de parada, entre outras, foram compartilhadas entre nosso algoritmo e o PCx. Além disso, todas as opções de compilação são as mesmas. Com efeito, a chamada do programa é exatamente igual e há apenas um comando \texttt{\#define} que faz com que o compilador seja desviado, dentro do \emph{loop} principal do PCx, para executar nosso algoritmo, retornando em seguida para a rotina principal e dar prosseguimento a execução usual do PCx.
Por conta disso, chamaremos nossa implementação de PCx-EOP. Isso foi feito, para fazermos uma comparação justa entre nossos resultados e os que o PCx produz. 



compilado com  compilador Intel  Composer 2013


\subsection{Ponto Inicial}

Documentação do PCx

\subsection{Critério de Parada}

Recompute Dual Variable

\subsection{Solução do subproblema de otimização \texorpdfstring{de $\nextphi$}{da função de mérito}}


\begin{itemize}
\item Descrever a heurística utilizada para resolver o subproblema de otimização
global de polinômios. Artigo escrito em conjunto com os orientadores está sendo
finalizado para submissão em que tal subproblema também aparece, porém num
método similar. Tal trabalho contempla também uma biblioteca para resolver o
subproblema de otimização de polinômios.
\begin{itemize}
  \item Este método similar, já em fase final de implementação,
  demonstra-se competitivo com o \texttt{PCx} nos testes preliminares
 \end{itemize}
\item Implementação do método proposto, utilizando-se da biblioteca supra
citada para resolver os subproblemas de otimização de polinômios.
\item Realizar  testes
numéricos com conjunto de teste da \texttt{Netlib} e complementares, com
comparação ao \texttt{PCx} e com o método desenvolvido pelo grupo.
\end{itemize}



\section{Resultados Numéricos}