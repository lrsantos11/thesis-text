%!TEX root = tese.tex
\chapter{Experimentos Numéricos}
\label{chap:numerical}

Neste capítulo descrevemos alguns detalhes de implementação que utilizamos em nossos experimento numéricos, bem os resultados de nosso algoritmo, comparando-o com o  PCx~\cite{Czyzyk:1999hk}.



\section{Detalhes de Implementação}

O Algoritmo \ref{alg:optimized-choice-of-parameters} foi implementado em  \texttt{C/C++}, utilizando como base a implementação PCx~\cite{Czyzyk:1999hk} do algoritmo preditor-corretor de \textcite{Mehrotra:1992wr}. O PCx pode ser configurado para executar correções de ordem superior de \textcite{Gondzio:1996uw} e permitimos que fossem feitas no máximo 2 correções através do arquivo de configuração, além de  refinamento da solução por gradiente conjugado presente na implementação original. Denominamos essa implementação de referência como PCx-r.

  Chamaremos nossa implementação de PCx-EOP. Com efeito, a chamada de PCx-EOP coincide com a da implementação de referência e há a introdução do código de nosso algoritmo dentro do \emph{loop} principal de PCx-r,  retornando em seguida para a rotina principal e dando prosseguimento a sua execução usual. Isso foi feito, para fazermos uma comparação justa entre nossos resultados e os que o PCx-r produz. 

  Assim, herdamos todas as rotinas de preprocessamento -- dentre as quais leitura de dados, precondicionamento e escalonamento, incluindo escalonamento de~\textcite{Curtis:1972cp}, reordenação de linhas e colunas, tratamento de colunas densas e fatoração simbólica de Cholesky --, de ponto inicial, de critério de parada e saída de dados,   bem como todas as rotinas de álgebra linear -- incluindo a estratégia  de \textcite{Ng:1993uz} para resolução de sistemas lineares esparsos via fatoração de Cholesky  e refinamento por gradientes conjugados. Além disso, para problemas canalizados, a estrutura de dados do PCx é feita de forma a fazer o papel da matriz $E$ e das transformações usadas em \eqref{eq:introPL-primal-bounded} e \eqref{eq:introPL-dual-bounded}, tal que os problemas canalizados tenham sua resolução executada com  as mesmas funcionalidades dos problemas não canalizados. 



Além disso, para todos os códigos foram usadas as mesmas opções de compilação e o mesmo computador. Com essa estratégia, diferenças nos tempos de CPU ou no número de iterações podem ser atribuídas apenas à implementação de cada algoritmo.
\subsection{Ponto Inicial}

O ponto inicial dado por \textcite{Mehrotra:1992wr}, foi genericamente descrito na seção \ref{subsec:initial-point}. Ali, mostramos que tal estratégia encontra o ponto $(\xtil,\ytil,\ztil)$, que é solução de norma mínima que satisfaz as restrições primais e
duais. Tal terna de pontos é encontrada através de
\begin{equation}
	\label{eq:intial-point-LSquare}
	\xtil = A^T(AA^T)^{-1}b, \quad \ytil = (AA^T)^{-1}Ac\quad \text{ e }
\quad \ztil = c - A^T\ytil.
\end{equation}

Além disso, o ponto é transladado para o ortante positivo 
através do uso de constantes  $\vartheta_x>0$ e $\vartheta_ z>0$ tais que  
\begin{equation}
	\label{eq:intial-point-generated}
(x^0,y^0,z^0) = (\xtil+ \vartheta_x e,\ytil,\ztil+\vartheta_z e)
\end{equation}
garantindo que $(x^0,z^0)>0$.


 A rotina \verb|InitialPoint| do PCx é responsável por gerar o ponto inicial, e portanto calcular  $\vartheta_x$ e $\vartheta_ z$. A implementação feita no PCx possui  pequenas alterações em relação ao proposto por \citeauthor{Mehrotra:1992wr}, porém que não  estão documentadas no Manual de Utilização do PCx~\cite{Czyzyk:1998vw}. Por isso, para fins de clareza, documentaremos como o ponto inicial é de fato encontrado, deixando o mérito de tal estratégia para os autores citados.

  Além disso, tal ponto inicial é utilizado em ambos PCx-r e PCx-EOP e a análise de convergência e complexidade relatada no Capítulo~\ref{chap:convergence}, em particular a Condição~\ref{cond:xzzero-xzstar}, continuam válidas ao utilizá-lo para iniciar nosso algoritmo.

O que a rotina  \verb|InitialPoint| faz é primeiramente encontrar  $(\xtil,\ytil,\ztil)$ conforme a Equação~\eqref{eq:intial-point-LSquare}. As constantes são calculadas
\begin{equation}
	\label{eq:initial-point-tilde-var}
\tilde{\vartheta_x} = \max\{\displaystyle\num{-1.5}\cdot\min_{i}\{\xtil_{i}\},\num{e-2}\} \quad \text{ e }\quad  \tilde{\vartheta_z} = \max\{\displaystyle\num{-1.5}\cdot\min_{i}\{\ztil_{i}\},\num{e-2}\}.
\end{equation}
Só então é que se obtém as translações dadas por 

\begin{equation}
	\label{eq:initial-point-var-x}
\vartheta_{x} = \tilde{\vartheta_x} + \num{0.5}\cdot\dfrac{(\xtil+\tilde{\vartheta_x} e)^{T}(\ztil+\tilde{\vartheta_z} e)}{\norm{\ztil+\tilde{\vartheta_z} e}_{1}} 
\end{equation}
e
\begin{equation}
	\label{eq:initial-point-var-z}
\vartheta_{z} = \tilde{\vartheta_z} + \num{0.5}\cdot\dfrac{(\xtil+\tilde{\vartheta_x} e)^{T}(\ztil+\tilde{\vartheta_z} e)}{\norm{\xtil+\tilde{\vartheta_x} e}_{1}} 
\end{equation}
a fim de finalmente produzir o ponto inicial como em \eqref{eq:intial-point-generated}.

No trabalho original de \textcite{Mehrotra:1992wr}, o menor valor possível para  $\tilde{\vartheta_x}$ e $\tilde{\vartheta_z}$ dados pela Equação~\eqref{eq:initial-point-tilde-var} é $\num{0}$, ao invés de \num{e-2}, 	que é o valor utilizado pelo PCx original. Em nossos testes, tanto com PCx-r quanto com PCx-EOP, utilizamos \num{e-1}. 


\subsection{Critério de Parada}


O PCx tem implementado dois critérios de parada que podem ser escolhidos pelo usuário. Sua   documentação~\cite{Czyzyk:1998vw} declara que é utilizado o critério dado em \eqref{eq:termination-criteria}.   Além disso, é possível trocar o critério de parada relacionado à complementaridade, dentro do código. Conforme explanamos na Seção~\ref{subsection:termination-criteria}, uma escolha possível seria a dada na equação \eqref{eq:termination-criteria-pcx}. Porém ao invés do teste dado pela Equação~\eqref{eq:termination-criteria-pcx-gap} -- ou por \eqref{eq:termination-criteria-gap} o que seria razoável --, a implementação do PCx usa

\[
\dfrac{x^{T}z/n}{1 + \abs{c^Tx}}\leq
	\tol,
\]
em que $\tol$ é a tolerância aceita. Essa escolha tem um viés dimensional, já que o divide por $n$ o \emph{gap}, fazendo com que o programa acuse otimalidade  antes do que deveria. Isso porque, usar esse teste equivale a utilizar como tolerância para o critério dado em \eqref{eq:termination-criteria-pcx-gap} $n\cdot\tol$. 


Fizemos uma pequena alteração no código, no momento em que testamos o critério de parada, que nos parece  contribuir para um melhor desempenho, pelo menos  no que diz respeito a verificação da otimalidade do ponto em questão. Para tanto, reutilizamos a rotina \verb|RecomputeDualVariables| do PCx. Tal rotina  é utilizada no PCx original somente quando já há a indicação de otimalidade e o programa saiu do \emph{loop} principal.

Descrevemos agora a rotina \verb|RecomputeDualVariables|. Seja $(\xtil,\ytil,\ztil)$ o ponto que o critério de parada do PCx indicou como sendo ótimo. O que a rotina faz, é primeiro calcular o vetor  $t= c - A^T\ytil$. É evidente que, por conta do modo como funcionam os \ac{MPI}, deveríamos ter $t\geq0$ e mais que isso, pela definição de problema dual, deveríamos ter $\ztil = t$. 

 O código então primeiro testa, para cada  $i = 1,\ldots,n$ se $t_{i}<0$.  Nos casos afirmativos, o programa assinala $t_{i}\leftarrow 0$. Finalmente estabelece-se que  $\ztil \leftarrow t $, o que finaliza a rotina. Basicamente isso significa que, a rotina garante que a variável dual $\ztil$, tenha o valor calculado pela definição do problema dual, e não pelo algoritmo. O teste para verificar se $t_{i}$ é negativo serve para detectar possíveis valores inconvenientes, já que $\zstar\geq 0$.

 Pois bem, a alteração que fizemos utiliza os princípios da rotina   \verb|RecomputeDualVariables| \emph{em toda iteração} $k$. Neste caso, o que fazemos é encontrar $t= c - A^T\yk$. Testamos então se para algum  $i$ ocorre $t_{i}<0$ e em caso afirmativo assinalamos $t_{i}\leftarrow 0$. Com isso, consideramos a terna $(\xk,\yk,t)$ com a qual realizamos os testes de otimalidade dados por
 \eqref{eq:termination-criteria-pcx}, porém com as seguintes modificações: o critério \eqref{eq:termination-criteria-pcx-dual} é transformado em
\[\dfrac{\norm{t}}{1 + \norm{c}} \leq \tol,\]	
e o critério  \eqref{eq:termination-criteria-pcx-gap} é transformado em 
\[
	\dfrac{(\xk)^{T}t}{1 + \abs{c^Tt}}\leq\tol.
\]
Se esses testes são satisfeitos, então assinalamos otimalidade no algoritmo e saímos do \emph{loop} principal. 
Isso é feito porque,  como $t$ não é mais interior, $(\xk)^{T}t$ tem possivelmente valor menor que $(\xk)^{T}\zk$. Essa estratégia diz que $(\xk,\yk,t)$ pode ser um ponto melhor que $(\xk,\yk,\zk)$.  Caso não seja satisfeito, continuamos a execução do programa com o ponto atual sendo $(\xk,\yk,\zk)$. Em ambos os casos $t$ é descartado e o ponto utilizado é sempre $(\xk,\yk,\zk)$. Considere que se tivermos otimalidade, \verb|RecomputeDualVariables| será chamada -- pois estamos fora do \emph{loop} e o programa assinala $(\xk,\yk,\zk)\leftarrow (\xk,\yk,t)$. 


O critério de parada usado  na implementação em PCx-r e PCx-EOP é o dado na Equação~\eqref{eq:termination-criteria-pcx}, porém com  a alteração relatada acima. 




% \begin{itemize}
% \item Descrever a heurística utilizada para resolver o subproblema de otimização
% global de polinômios. Artigo escrito em conjunto com os orientadores está sendo
% finalizado para submissão em que tal subproblema também aparece, porém num
% método similar. Tal trabalho contempla também uma biblioteca para resolver o
% subproblema de otimização de polinômios.
% \begin{itemize}
%   \item Este método similar, já em fase final de implementação,
%   demonstra-se competitivo com o \texttt{PCx} nos testes preliminares
%  \end{itemize}
% \end{itemize}



\subsection{Conjunto de testes}

O conjunto de testes que vamos utilizar foi retirado da Netlib\footnote{Disponíveis em \url{http://www.netlib.org}.}~\cite{Dongarra:1987jk}. Escolhemos primeiramente os \num{95} problemas de programação linear factíveis que estão no diretório \verb|/lp/data/| de tal repositório.  Além disso, do mesmo repositório escolhemos \num{12} dos \num{16} problemas Kennington~\cite{Kennington:1990vo} que se encontram em \verb|/lp/data/kennington/|. Para finalizar, escolhemos 1 problema -- \texttt{qap-8} -- dos \num{3} do conjunto  QAP, que estão disponíves no diretório \verb|/lp/generators/qap|. Os problemas QAP são  construído via um gerador  e  aparecem a partir da linearização de um problema  de designação quadrática. No total então, temos 108 problemas e por isso, tal conjunto de testes é chamado aqui de Netlib-108. 


A Tabela~\ref{tab:netlib108} mostra os \num{108} problemas com seus nomes na primeira coluna. As colunas 2 a 4 mostram o número de linhas, colunas e de canalizações de cada problema, após as rotinas de pré-processamento feitas pelo PCx. Indicamos que um problema  não é canalizado por colocar o número \num{0} na coluna correspondente. A coluna 4 tem o número de elementos não-nulos (NEM) da matriz $A$ das restrições e a última coluna apresenta a densidade da matriz resultante da decomposição de de Cholesky que é para resolução de cada problema.


Alguns comentários sobre a escolha de Netlib-108 cabem aqui. Primeiramente, os \num{4} problemas  Kennington -- \texttt{kennington-ken-13}, \texttt{kennington-ken-18}, \texttt{ken\-ning\-ton-osa-60} e \texttt{ken\-ning\-ton-pds-20} --  e os \num{2} QAP -- \texttt{qap12} e \texttt{qap15} --  que foram deixados de fora dos testes, o foram pois suas estruturas e tamanhos não fazem sentido numa implementação que resolve as equações normais que aparecem em \ac{MPI} via fatoração de Cholesky. 

Além disso, tal gama de problemas é a Netlib mais natural, originária de problemas reais e consensual entre os usuários de \ac{PL}. Com efeito, tais problemas ou pelo menos parte deles, foram testados por \textcite{Mehrotra:1992wr,Colombo:2008ia,Mehrotra:2005do,Jarre:1999tl,Gondzio:1996uw} entre outros pesquisadores com trabalhos importantes da área e estão inclusive na documentação do PCx~\cite{Czyzyk:1998vw,Czyzyk:1999hk}, com a finalidade de demonstrar os testes de seu desempenho. Nesse sentido e assim como esses autores, pensamos que tal conjunto é capaz permitir testes adequados de nosso método. 


{\small \onehalfspacing


% \tabletail{\midrule  
% \multicolumn{6}{r}{Continua na próxima página.} \\ }
% \tablelasttail{\bottomrule}

\begin{longtable}{>{\ttfamily}l
r%S[table-number-alignment = right]
r%S[table-number-alignment = right]
r%S[table-number-alignment = right]
r%[table-number-alignment = right]
S[fixed-exponent = -1,
table-format = 2.5,
table-omit-exponent,
table-number-alignment = center-decimal-marker]}

\caption{\normalfont Conjunto de testes Netlib-108.\label{tab:netlib108}} \\
\toprule
 {\normalfont \bfseries Problema}                  & {\normalfont \bfseries Linhas} & {\normalfont \bfseries Colunas} & {\normalfont \bfseries Canalizações} & {\normalfont \bfseries NEN} & {\normalfont \bfseries Densidade ($\times$\num{e-1})}    \\
\otoprule
\endfirsthead


\caption[]{\normalfont Conjunto de testes Netlib-108 (continuação).} \\

\toprule
 {\normalfont \bfseries Problema}                  & {\normalfont \bfseries Linhas} & {\normalfont \bfseries Colunas} & {\normalfont \bfseries Canalizações} & {\normalfont \bfseries NEN} & {\normalfont \bfseries Densidade ($\times$\num{e-1}) }    \\
\otoprule
\endhead


\bottomrule

\endlastfoot

\midrule  
\multicolumn{6}{r}{\scriptsize Continua na próxima página.}
\endfoot
25fv47              & 788       & 1843      & 0           & 33809    & 1,0763E-01 \\
80bau3b             & 2140      & 11066     & 2968        & 41367    & 1,7598E-02 \\
adlittle            & 55        & 137       & 0           & 404      & 2,4893E-01 \\
afiro               & 27        & 51        & 0           & 107      & 2,5652E-01 \\
agg                 & 390       & 477       & 0           & 12297    & 1,5913E-01 \\
agg2                & 514       & 750       & 0           & 21482    & 1,6068E-01 \\
agg3                & 514       & 750       & 0           & 21482    & 1,6068E-01 \\
bandm               & 240       & 395       & 0           & 3936     & 1,3250E-01 \\
beaconfd            & 86        & 171       & 0           & 820      & 2,1011E-01 \\
blend               & 71        & 111       & 0           & 913      & 3,4815E-01 \\
bnl1                & 610       & 1491      & 0           & 12089    & 6,3338E-02 \\
bnl2                & 1964      & 4008      & 0           & 81275    & 4,1632E-02 \\
boeing1             & 331       & 697       & 243         & 5725     & 1,0149E-01 \\
boeing2             & 126       & 265       & 73          & 2029     & 2,4767E-01 \\
bore3d              & 81        & 138       & 7           & 1034     & 3,0285E-01 \\
brandy              & 133       & 238       & 0           & 2755     & 3,0397E-01 \\
capri               & 241       & 436       & 131         & 3962     & 1,3228E-01 \\
cycle               & 1420      & 2773      & 77          & 56102    & 5,4941E-02 \\
czprob              & 671       & 2779      & 0           & 3520     & 1,4146E-02 \\
d2q06c              & 2132      & 5728      & 0           & 137349   & 5,9965E-02 \\
d6cube              & 403       & 5443      & 0           & 54840    & 6,7285E-01 \\
degen2              & 444       & 757       & 0           & 16319    & 1,6331E-01 \\
degen3              & 1503      & 2604      & 0           & 120906   & 1,0638E-01 \\
dfl001              & 5984      & 12143     & 13          & 1638085  & 9,1325E-02 \\
e226                & 198       & 429       & 0           & 3229     & 1,5968E-01 \\
etamacro            & 334       & 669       & 131         & 10843    & 1,9140E-01 \\
fffff800            & 322       & 826       & 0           & 9573     & 1,8155E-01 \\
finnis              & 438       & 935       & 33          & 4984     & 4,9676E-02 \\
fit1d               & 24        & 1049      & 1026        & 296      & 9,8611E-01 \\
fit1p               & 627       & 1677      & 399         & 627      & 1,5950E-03 \\
fit2d               & 25        & 10524     & 10500       & 324      & 9,9680E-01 \\
fit2p               & 3000      & 13525     & 7500        & 3000     & 3,3300E-04 \\
forplan             & 121       & 447       & 22          & 3304     & 4,4307E-01 \\
ganges              & 1113      & 1510      & 397         & 29677    & 4,7015E-02 \\
gfrd-pnc            & 590       & 1134      & 258         & 2112     & 1,0440E-02 \\
greenbea            & 1933      & 4164      & 264         & 49055    & 2,5740E-02 \\
greenbeb            & 1932      & 4154      & 268         & 47783    & 2,5085E-02 \\
grow7               & 140       & 301       & 280         & 2730     & 2,7143E-01 \\
grow15              & 300       & 645       & 600         & 6090     & 1,3200E-01 \\
grow22              & 440       & 946       & 880         & 9058     & 9,1302E-02 \\
israel              & 174       & 316       & 0           & 11488    & 7,5314E-01 \\
kb2                 & 43        & 68        & 9           & 503      & 5,2082E-01 \\
lotfi               & 133       & 346       & 0           & 1369     & 1,4727E-01 \\
maros-r7            & 2152      & 7440      & 0           & 534188   & 2,3023E-01 \\
maros               & 655       & 1437      & 0           & 13454    & 6,1192E-02 \\
modszk1             & 665       & 1599      & 0           & 10550    & 4,6210E-02 \\
nesm                & 654       & 2922      & 1596        & 21776    & 1,0030E-01 \\
perold              & 593       & 1374      & 259         & 21782    & 1,2220E-01 \\
pilot.ja            & 810       & 1804      & 331         & 47924    & 1,4485E-01 \\
pilot.we            & 701       & 2814      & 287         & 15605    & 6,2086E-02 \\
pilot               & 1368      & 4543      & 1040        & 200812   & 2,1388E-01 \\
pilot4              & 396       & 1022      & 247         & 14279    & 1,7959E-01 \\
pilot87             & 1971      & 6373      & 1578        & 425654   & 2,1863E-01 \\
pilotnov            & 848       & 2117      & 332         & 46353    & 1,2774E-01 \\
recipe              & 64        & 123       & 56          & 277      & 1,1963E-01 \\
sc50a               & 49        & 77        & 0           & 242      & 1,8118E-01 \\
sc50b               & 48        & 76        & 0           & 231      & 1,7969E-01 \\
sc105               & 104       & 162       & 0           & 569      & 9,5599E-02 \\
sc205               & 203       & 315       & 0           & 1156     & 5,1178E-02 \\
scagr7              & 127       & 183       & 0           & 730      & 8,2646E-02 \\
scagr25             & 469       & 669       & 0           & 2944     & 2,4636E-02 \\
scfxm1              & 305       & 568       & 0           & 4430     & 9,1965E-02 \\
scfxm2              & 610       & 1136      & 0           & 9284     & 4,8261E-02 \\
scfxm3              & 915       & 1704      & 0           & 14138    & 3,2681E-02 \\
scorpion            & 340       & 412       & 0           & 2275     & 3,6419E-02 \\
scrs8               & 421       & 1199      & 0           & 4477     & 4,8143E-02 \\
scsd1               & 77        & 760       & 0           & 1392     & 4,5657E-01 \\
scsd6               & 147       & 1350      & 0           & 2545     & 2,2875E-01 \\
scsd8               & 397       & 2750      & 0           & 5879     & 7,2083E-02 \\
sctap1              & 284       & 644       & 0           & 2435     & 5,6859E-02 \\
sctap2              & 1033      & 2443      & 0           & 11736    & 2,1028E-02 \\
sctap3              & 1408      & 3268      & 0           & 16100    & 1,5532E-02 \\
seba                & 448       & 901       & 447         & 53711    & 5,3299E-01 \\
share1b             & 112       & 248       & 0           & 1417     & 2,1700E-01 \\
share2b             & 96        & 162       & 0           & 1041     & 2,1550E-01 \\
shell               & 487       & 1451      & 117         & 3983     & 3,1534E-02 \\
ship04l             & 292       & 1905      & 0           & 2641     & 5,8524E-02 \\
ship04s             & 216       & 1281      & 0           & 1778     & 7,1588E-02 \\
ship08l             & 470       & 3121      & 0           & 4442     & 3,8090E-02 \\
ship08s             & 276       & 1604      & 0           & 2295     & 5,6632E-02 \\
ship12l             & 610       & 4171      & 0           & 5506     & 2,7955E-02 \\
ship12s             & 340       & 1943      & 0           & 2507     & 4,0433E-02 \\
sierra              & 1212      & 2705      & 2016        & 12862    & 1,6687E-02 \\
stair               & 356       & 532       & 6           & 14682    & 2,2889E-01 \\
standata            & 314       & 796       & 96          & 2395     & 4,5397E-02 \\
standgub            & 314       & 796       & 96          & 2395     & 4,5397E-02 \\
standmps            & 422       & 1192      & 96          & 3957     & 4,2070E-02 \\
stocfor1            & 102       & 150       & 0           & 805      & 1,4494E-01 \\
stocfor2            & 1980      & 2868      & 0           & 22841    & 1,1147E-02 \\
stocfor3            & 15362     & 22228     & 0           & 177936   & 1,4430E-03 \\
truss               & 1000      & 8806      & 0           & 53509    & 1,0602E-01 \\
tuff                & 257       & 567       & 25          & 7051     & 2,0962E-01 \\
vtp.base            & 72        & 111       & 32          & 505      & 1,8094E-01 \\
wood1p              & 171       & 1718      & 0           & 11645    & 7,9064E-01 \\
woodw               & 708       & 5364      & 0           & 30027    & 1,1839E-01 \\
kennington-cre-a    & 2994      & 6692      & 0           & 33212    & 7,0760E-03 \\
kennington-cre-b    & 5336      & 36382     & 0           & 248629   & 1,7277E-02 \\
kennington-cre-c    & 2375      & 5412      & 0           & 28528    & 9,6940E-03 \\
kennington-cre-d    & 4102      & 28601     & 0           & 212094   & 2,4966E-02 \\
kennington-ken-07   & 1437      & 2613      & 2613        & 10034    & 9,0220E-03 \\
kennington-ken-11   & 10085     & 16740     & 16740       & 102906   & 1,9240E-03 \\
kennington-osa-07   & 1081      & 25030     & 0           & 28276    & 4,7469E-02 \\
kennington-osa-14   & 2300      & 54760     & 0           & 60795    & 2,2550E-02 \\
kennington-osa-30   & 4313      & 104337    & 0           & 115081   & 1,2141E-02 \\
kennington-pds-02   & 2609      & 7339      & 1914        & 44301    & 1,2633E-02 \\
kennington-pds-06   & 9156      & 28472     & 8448        & 589339   & 1,3951E-02 \\
kennington-pds-10   & 15648     & 48780     & 15125       & 1687660  & 1,3721E-02 \\
qap8                & 912       & 1632      & 0           & 193944   & 4,6526E-01 \\
\end{longtable}

}




\subsection{Solução do subproblema de otimização \texorpdfstring{de $\nextphi$}{da função de mérito}}

Em cada iteração de nosso método, precisamos resolver o problema dado em \eqref{eq:pop-subproblem}, isto é, 
\begin{equation}
	\label{eq:pop-subproblem-1}
	\begin{array}{lc}
\displaystyle \min_{(\al,\mu,\sig)} & \hat\varphi(\al,\mu,\sig) \\
\text{s. a.} &\begin{cases} g_C^i(\al,\mu,\sig) \geq 0 \quad \forall i = 1,\ldots,n \\
				g_L(\al,\mu,\sig)   \geq 0 	\\
				 0\leq (\al,\mu,\sig) \leq u,
				 	
				 \end{cases}.
\end{array}
\end{equation}
em que $\varphi$ bem como $g_{C}^{i}$ e $g_{L}$ são polinômios de grau total 6 nas variáveis $(\al,\mu,\sig)$. Este é um chamado Problema de Otimização de Polinômios (POP) e é considerado um problema não linear  difícil de ser resolvido~\cite{Laurent:2010kp}.

Testamos algumas  implementações especializadas em POP, dentre elas GloptiPoly~\cite{Henrion:2009eb} e SaparsePOP~\cite{Waki:2008ie}, os quais baseiam-se na transformação do POP em um problema de Programação Semidefinida~\cite{Lasserre:2001fw}. Nenhuma delas, entretanto, foi capaz de resolver os subproblemas, em particular pela quantidade de restrições do problema, que é  $n+1$.  


O subproblema foi resolvido utilizando da implementação de \textcite{VillasBoas:2012ur,VillasBoas2013:wn}, especialmente desenvolvida para resolver um problema com as mesmas características que o nosso.  Tal estratégia é brevemente resumida a seguir, adaptada à nossa notação.

Seja
\[\Omega = \left\{(\al,\mu,\sig): g_C^i(\al,\mu,\sig) \geq0, \forall i = 1,\ldots,n,\: g_L(\al,\mu,\sig)   \geq 0,\:		 0\leq (\al,\mu,\sig) \leq u\right\}\]
o conjunto factível de~\eqref{eq:pop-subproblem-1}. Note que pelas demonstrações que fizemos no Capítulo \ref{chap:convergence}, $\Omega$ é não-vazio. Considere então o problema alternativo
\begin{equation}
		\label{eq:pop-subproblem-2}
\displaystyle  \min_{\mu}\left\{  \min_{\sig} \left\{ \min_{\al} \left\{ \nextphi(\al,\mu,\sig):    (\al,\mu,\sig) \in \Omega  \right\} \right\}\right\}, \\
\end{equation}
ou mais convenientemente 
\[
\displaystyle \min_{\mu}\Psi(\mu),
\]
em que  $\displaystyle \Psi = \min_{\sig}\Phi(\mu,\sig)$ e 
\[
\Phi(\mu,\sig) = \left\{
	\begin{array}{l}
\displaystyle \min_{\al}  \nextphi(\al,\mu,\sig) \\
\text{s. a. }\:  (\al,\mu,\sig) \in\Omega
\end{array}\right..
\]
  
Primeiramente, vamos verificar a relação existente entre os mínimos de \eqref{eq:pop-subproblem-1} e de \eqref{eq:pop-subproblem-2}. Sejam $(\al^*,\mu^*,\sig^*)$ uma solução global do  primeiro problema e $(\bar{\al}, \bar{\mu},\bar{\sig})$ uma solução global de segundo. Então por definição
$ \nextphi(\al^*,\mu^*,\sig^*) \leq \nextphi (\bar{\al},\bar{\mu},\bar{\sig}).$
Por outro lado 
\[
\nextphi(\baral,\barmu,\barsig) = \min_{\al}\{ \min_{\sig}\Phi(\barmu,\barsig)\} \leq \Phi(\mu^{*},\sig^{*}) \leq \nextphi(\al^*,\mu^*,\sig^*).
\]
Assim, os dois problemas tem o mesmo mínimo. A segunda formulação, entretanto, permite encontrar aproximações numéricas do minimizador de $\nextphi$ de maneira mais fácil. 

Para esse problema alternativo, um gráfico típico de $\Phi(\mu,\sig)$ é dado na Figura \ref{fig:bnl1-Phi}, tomada na iteração \num{4} do problema \texttt{bnl1}. O aspecto desse gráfico -- , com um \emph{zoom} conveniente perto do mínimo --  acaba se repetindo  em todos os problema da Netlib-108 que estudamos, o que sugere que $\Phi$ tenha derivadas contínuas por partes próximo ao mínimo.

\begin{figure}[htbp]
\centering
\includegraphics[width=.5\textwidth]{figuras/bnl1-phi}
  	\caption{\label{fig:bnl1-Phi} Gráfico de $\Phi(\mu,\sig)$ para $\al$ fixo de \texttt{bnl1}.}
  \end{figure}

 Uma característica interessante é desse problema alternativo é que em quase todos os problemas e iterações, a restrição de $\Omega$ que estava ativa ao final da resolução do subproblema é exatamente aquela relacionada ao índice de bloqueio indicado pelo teste da razão.

Primeiramente considere  um malha construída para o intervalo $[0,u_{\mu}]$ no qual  $\mu$ está definido. Usando o fato de que o índice de bloqueio do teste da razão tem relação com a restrição ativa, uma partição $\mathcal{P} = 0<\sig_{1}<\sig_{2}<\cdots<u_{\sig}$ do intervalo $[0,u_{\sig}]$, para  a  qual $\sig$ está definido é construída, tal que se $\barsig\in I_{j} = [\sig_{j},\sig_{j+1}]$, então existe um índice $t$ tal para este $\barsig$, todo o teste da razão para as componentes não negativas das variáveis $(x,z)$ podem ser substituídas por um único teste da razão, usando o índice $t$. Dessa forma, para cada subintervalo $I_{j} = [\sig_{j},\sig_{j+1}]$ define-se o teste da razão principal. 

Esse índice $t$ permite escolhamos a restrição $g_{C}^{t}(\al,\sig)\geq0$, a qual é mais restritiva que o teste da razão,   sendo  então declaradas como \emph{restrição principal}  de $I_{j}$. De fato, a partição $\mathcal{P}$ agora é reconstruída tal que cada ponto na fronteira do subintervalo está associado com a interseção de restrições principais, gerando então uma região de relevância.

Essa estratégia leva a descartar restrições que são irrelevantes, reduzindo o número das que precisam ser utilizadas de fato, na solução do subproblema.  Para cada intervalo $I_{j}$, determina-se um tamanho de passo máximo $\al^{\max}_{j}$ e o mínimo de $\nextphi$ em cada intervalo. Isso é feito 	de maneira analítica, usando condições de primeira ordem. Finalmente, o menor dos $\nextphi$ encontrados para cada intervalo é declarado como ótimo.

Na prática, esse método funciona porque o tamanho da partição é muito menor ao tamanho dos problemas -- para todo Netlib-108 para todas as iterações a partição tem em média tamanho \num{3.7} com desvio padrão \num{2.9} e máximo \num{39}. De fato, o tempo  de CPU gasto na solução do subproblema  para toda Netlib-108 foi \num{1,37}\% do tempo total. Um detalhe é que em grande parte dos casos, o melhor resultado de $\nextphi$ foi encontrado com $\mu=0$.




\section{Resultados Numéricos}

Os testes computacionais foram rodados em um computador Intel i7 930 2.8GHz com \num{12}GB de memória RAM em sistema operacional Windows 7 \num{64}bits e compilados utilizando Microsoft Visual Studio \texttt{C++} 2010 com compilador Intel \texttt{C++} Compiler  XE 13.0.

A Tabela \ref{tab:testes-PCx-r-EOP}, mostra o número total de iterações, o tempo de CPU e a saída programa -- \texttt{0} indica convergência e \texttt{3} indica que otimalidade não foi alcançada -- para as implementações PCx-r e PCx-EOP. 




Os resultados mostram que PCx-EOP resolveu o problema em tempo de CPU e número de iterações comparável ao do PCx-r, com tempos totais de \SI{115.171}{s} para o PCx-r e \SI{156,291}{s} para o PCx-EOP. 
 

Os perfis de desempenho de \textcite{Dolan:2002du} para tempo de CPU e número de iterações  mostram que PCx-EOP tem robustez comparável ao PCx-r. Seus resultados encontram-se nas Figuras \ref{fig:perproftime} e \ref{fig:perprofiter}.


\begin{figure}[htbp]
\centering
\includegraphics[width=.7\textwidth]{figuras/time}
  	\caption{\label{fig:perproftime} Perfil de desempenho de PCx-r e PCx-EOP em relação ao tempo de CPU.}
  \end{figure}

  \begin{figure}[htbp]
\centering
\includegraphics[width=.7\textwidth]{figuras/iter}
  	\caption{\label{fig:perprofiter}  Perfil de desempenho de PCx-r e PCx-EOP em relação ao número de iterações.}
  \end{figure}

{\small \onehalfspacing

\begin{longtable}{>{\ttfamily}l
c%S[table-number-alignment = right]
c%S[table-number-alignment = right]
>{\ttfamily}c%S[table-number-alignment = right]
c%[table-number-alignment = right]
c
>{\ttfamily}c}

\caption{\normalfont Resultados comparativos de PCx-r e PCx-EOP.\label{tab:testes-PCx-r-EOP}} \\

\toprule
   & \multicolumn{3}{c}{{ \normalfont\bfseries PCx-r}} & \multicolumn{3}{c}{{ \normalfont\bfseries PCx-EOP}}\\
\cmidrule(lr){2-4} \cmidrule(l){5-7}
     { \normalfont\bfseries Problema}            & { \normalfont\bfseries Iter.} & { \normalfont\bfseries CPU (\si{s})}  & { \normalfont\bfseries Status}  & { \normalfont\bfseries Iter.} & { \normalfont\bfseries CPU (\si{s})} & { \normalfont\bfseries Status} \\
  \otoprule
\endfirsthead


\caption[]{\normalfont Resultados comparativos de PCx-r e PCx-EOP (continuação).} \\

\toprule
   & \multicolumn{3}{c}{{ \normalfont\bfseries PCx-r}} & \multicolumn{3}{c}{{ \normalfont\bfseries PCx-EOP}}\\
\cmidrule(lr){2-4} \cmidrule(l){5-7}
     { \normalfont\bfseries Problema}            & { \normalfont\bfseries Iter.} & { \normalfont\bfseries CPU (\si{s})}  & { \normalfont\bfseries Status}  & { \normalfont\bfseries Iter.} & { \normalfont\bfseries CPU (\si{s})} & { \normalfont\bfseries Status} \\
  \otoprule
 \endhead


\bottomrule

\endlastfoot

\midrule  
\multicolumn{7}{r}{\scriptsize Continua na próxima página.}
\endfoot
25fv47              & 24 & 0,193  & 0 & 27 &  0,382  & 0  \\
80bau3b             & 39 & 0,473  & 0 & 38 &  0,882  & 0  \\
adlittle            & 10 & 0,038  & 0 & 18 &  0,163  & 0  \\
afiro               & 6  & 0,031  & 0 & 12 &  0,111  & 0  \\
agg                 & 17 & 0,083  & 0 & 21 &  0,219  & 0  \\
agg2                & 20 & 0,118  & 0 & 23 &  0,268  & 0  \\
agg3                & 19 & 0,115  & 0 & 22 &  0,256  & 0  \\
bandm               & 15 & 0,065  & 0 & 24 &  0,237  & 0  \\
beaconfd            & 10 & 0,059  & 0 & 16 &  0,189  & 0  \\
blend               & 8  & 0,033  & 0 & 18 &  0,168  & 0  \\
bnl1                & 33 & 0,169  & 0 & 50 &  0,602  & 3  \\
bnl2                & 29 & 0,551  & 0 & 31 &  0,749  & 0  \\
boeing1             & 21 & 0,099  & 0 & 29 &  0,304  & 0  \\
boeing2             & 12 & 0,053  & 0 & 21 &  0,233  & 0  \\
bore3d              & 14 & 0,053  & 0 & 23 &  0,212  & 0  \\
brandy              & 17 & 0,082  & 3 & 22 &  0,240  & 0  \\
capri               & 17 & 0,071  & 0 & 25 &  0,284  & 0  \\
cycle               & 28 & 0,304  & 0 & 24 &  0,414  & 0  \\
czprob              & 25 & 0,137  & 0 & 29 &  0,376  & 0  \\
d2q06c              & 27 & 0,812  & 0 & 28 &  1,060  & 0  \\
d6cube              & 17 & 0,298  & 0 & 23 &  0,586  & 0  \\
degen2              & 10 & 0,074  & 0 & 11 &  0,147  & 0  \\
degen3              & 23 & 0,637  & 0 & 15 &  0,533  & 0  \\
dfl001              & 49 & 46,355 & 0 & 50 &  46,316 & 0  \\
e226                & 17 & 0,073  & 0 & 22 &  0,221  & 0  \\
etamacro            & 27 & 0,121  & 0 & 31 &  0,332  & 0  \\
fffff800            & 29 & 0,139  & 0 & 35 &  0,422  & 0  \\
finnis              & 22 & 0,096  & 0 & 29 &  0,323  & 0  \\
fit1d               & 16 & 0,089  & 0 & 23 &  0,348  & 0  \\
fit1p               & 16 & 0,141  & 0 & 27 &  0,426  & 3  \\
fit2d               & 22 & 0,450  & 0 & 35 &  1,151  & 0  \\
fit2p               & 21 & 0,634  & 0 & 31 &  1,050  & 3  \\
forplan             & 20 & 0,087  & 0 & 28 &  0,322  & 0  \\
ganges              & 15 & 0,128  & 0 & 21 &  0,313  & 0  \\
gfrd-pnc            & 15 & 0,064  & 0 & 19 &  0,199  & 0  \\
greenbea            & 49 & 0,511  & 3 & 31 &  0,551  & 3  \\
greenbeb            & 38 & 0,401  & 0 & 37 &  0,689  & 0  \\
grow7               & 13 & 0,057  & 0 & 21 &  0,215  & 0  \\
grow15              & 16 & 0,081  & 0 & 25 &  0,276  & 0  \\
grow22              & 17 & 0,101  & 0 & 27 &  0,365  & 0  \\
israel              & 19 & 0,130  & 0 & 21 &  0,232  & 0  \\
kb2                 & 10 & 0,039  & 0 & 25 &  0,236  & 0  \\
lotfi               & 14 & 0,056  & 0 & 22 &  0,253  & 0  \\
maros-r7            & 13 & 2,006  & 0 & 20 &  2,959  & 0  \\
maros               & 17 & 0,095  & 0 & 23 &  0,270  & 0  \\
modszk1             & 18 & 0,095  & 0 & 32 &  0,362  & 0  \\
nesm                & 28 & 0,197  & 0 & 29 &  0,463  & 0  \\
perold              & 32 & 0,200  & 0 & 47 &  0,587  & 0  \\
pilot.ja            & 33 & 0,390  & 0 & 38 &  0,694  & 0  \\
pilot.we            & 44 & 0,286  & 0 & 50 &  0,714  & 0  \\
pilot               & 34 & 1,879  & 0 & 36 &  2,155  & 0  \\
pilot4              & 35 & 0,234  & 0 & 43 &  0,522  & 3  \\
pilot87             & 33 & 4,804  & 0 & 35 &  5,155  & 0  \\
pilotnov            & 16 & 0,173  & 0 & 16 &  0,289  & 0  \\
recipe              & 7  & 0,030  & 0 & 15 &  0,148  & 0  \\
sc50a               & 7  & 0,029  & 0 & 14 &  0,131  & 0  \\
sc50b               & 6  & 0,026  & 0 & 10 &  0,093  & 0  \\
sc105               & 9  & 0,037  & 0 & 19 &  0,177  & 0  \\
sc205               & 9  & 0,038  & 0 & 18 &  0,210  & 0  \\
scagr7              & 12 & 0,047  & 0 & 22 &  0,224  & 0  \\
scagr25             & 16 & 0,068  & 0 & 24 &  0,238  & 0  \\
scfxm1              & 16 & 0,068  & 0 & 23 &  0,278  & 0  \\
scfxm2              & 19 & 0,106  & 3 & 27 &  0,356  & 0  \\
scfxm3              & 18 & 0,103  & 0 & 26 &  0,300  & 0  \\
scorpion            & 10 & 0,063  & 0 & 12 &  0,142  & 0  \\
scrs8               & 21 & 0,091  & 0 & 33 &  0,420  & 0  \\
scsd1               & 8  & 0,035  & 0 & 17 &  0,171  & 0  \\
scsd6               & 10 & 0,046  & 0 & 19 &  0,228  & 0  \\
scsd8               & 9  & 0,048  & 0 & 19 &  0,221  & 0  \\
sctap1              & 13 & 0,055  & 0 & 18 &  0,181  & 0  \\
sctap2              & 12 & 0,069  & 0 & 21 &  0,275  & 0  \\
sctap3              & 13 & 0,082  & 0 & 17 &  0,280  & 0  \\
seba                & 11 & 0,191  & 0 & 14 &  0,329  & 0  \\
share1b             & 17 & 0,093  & 0 & 23 &  0,220  & 0  \\
share2b             & 16 & 0,063  & 0 & 23 &  0,221  & 0  \\
shell               & 20 & 0,114  & 0 & 22 &  0,287  & 0  \\
ship04l             & 11 & 0,051  & 0 & 19 &  0,199  & 0  \\
ship04s             & 11 & 0,048  & 0 & 19 &  0,195  & 0  \\
ship08l             & 13 & 0,067  & 0 & 23 &  0,325  & 0  \\
ship08s             & 10 & 0,045  & 0 & 16 &  0,166  & 0  \\
ship12l             & 14 & 0,079  & 0 & 19 &  0,238  & 0  \\
ship12s             & 12 & 0,056  & 0 & 17 &  0,190  & 0  \\
sierra              & 18 & 0,144  & 0 & 20 &  0,316  & 0  \\
stair               & 15 & 0,082  & 0 & 20 &  0,264  & 0  \\
standata            & 12 & 0,050  & 0 & 16 &  0,159  & 0  \\
standgub            & 12 & 0,051  & 0 & 16 &  0,160  & 0  \\
standmps            & 22 & 0,095  & 0 & 30 &  0,349  & 0  \\
stocfor1            & 10 & 0,039  & 0 & 19 &  0,236  & 0  \\
stocfor2            & 19 & 0,136  & 0 & 25 &  0,368  & 0  \\
stocfor3            & 30 & 0,801  & 0 & 36 &  1,483  & 0  \\
truss               & 17 & 0,197  & 0 & 27 &  0,503  & 0  \\
tuff                & 20 & 0,090  & 0 & 23 &  0,299  & 0  \\
vtp.base            & 9  & 0,042  & 0 & 17 &  0,157  & 0  \\
wood1p              & 22 & 0,304  & 0 & 23 &  0,448  & 0  \\
woodw               & 31 & 0,264  & 0 & 34 &  0,545  & 0  \\
kennington-cre-a    & 25 & 0,274  & 0 & 23 &  0,494  & 0  \\
kennington-cre-b    & 43 & 2,420  & 0 & 40 &  3,035  & 0  \\
kennington-cre-c    & 25 & 0,245  & 0 & 25 &  0,453  & 0  \\
kennington-cre-d    & 41 & 1,998  & 0 & 38 &  2,529  & 0  \\
kennington-ken-07   & 13 & 0,107  & 0 & 20 &  0,314  & 0  \\
kennington-ken-11   & 20 & 0,489  & 0 & 23 &  1,034  & 0  \\
kennington-osa-07   & 21 & 0,380  & 0 & 54 &  1,811  & 0  \\
kennington-osa-14   & 24 & 0,881  & 0 & 77 &  4,754  & 0  \\
kennington-osa-30   & 24 & 1,680  & 0 & 81 &  9,036  & 0  \\
kennington-pds-02   & 24 & 0,301  & 0 & 28 &  0,648  & 0  \\
kennington-pds-06   & 32 & 5,592  & 0 & 39 &  7,198  & 0  \\
kennington-pds-10   & 40 & 31,566 & 0 & 46 &  35,754 & 0  \\
qap8                & 7  & 0,435  & 0 & 8  &  0,476  & 0  \\

\end{longtable}
}
