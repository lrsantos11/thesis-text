%!TEX root = tese.tex
\chapter{Experimentos Numéricos}
\label{chap:numerical}

Neste capítulo descrevemos alguns detalhes de implementação que utilizamos em nossos experimento numéricos, bem os resultados de nosso algoritmo, comparando-o com o  PCx~\cite{Czyzyk:1999hk}.



\section{Detalhes de Implementação}

O Algoritmo \ref{alg:optimized-choice-of-parameters} foi implementado em  \texttt{C/C++}, utilizando como base a implementação PCx~\cite{Czyzyk:1999hk} do algoritmo preditor-corretor de \textcite{Mehrotra:1992wr}. O PCx pode ser configurado para executar correções de ordem superior de \textcite{Gondzio:1996uw} e permitimos que fossem feitas no máximo 2 correções através do arquivo de configuração, além de  refinamento da solução por gradiente conjugado presente na implementação original. Denominamos essa implementação de referência como PCx-r.

  Chamaremos nossa implementação de PCx-EOP. Com efeito, a chamada de PCx-EOP coincide com a da implemenatação de referência e há a introdução do código de nosso algoritmo dentro do \emph{loop} principal de PCx-r,  retornando em seguida para a rotina principal e dando prosseguimento a sua execução usual. Com isso, herdamos todas as rotinas de preprocessamento -- dentre as quais leitura de dados, precondicionamento e escalamento, reordenação de linhas e colunas, tratamento de colunas densas e fatoração simbólica de Cholesky --, de ponto inicial, de critério de parada e saída de dados,   bem como todas as rotinas de álgebra linear -- incluindo-se aí as rotinas de \textcite{Ng:1993uz} para resolução de sistemas lineares esparsos via fatoração de Cholesky  e refinamento por gradientes conjugados. Isso foi feito, para fazermos uma comparação justa entre nossos resultados e os que o PCx-r produz. 



Além disso, para todos os códigos foram usadas as mesmas opções de compilação e o mesmo computador. Com essa estratégia, diferenças nos tempos de CPU ou no número de iterações podem ser atribuídas apenas à implementação de cada algoritmo.
\subsection{Ponto Inicial}

O ponto inicial dado por \textcite{Mehrotra:1992wr}, foi genericamente descrito na seção \ref{subsec:initial-point}. Ali, mostramos que tal estratégia encontra o ponto $(\xtil,\ytil,\ztil)$, que é solução de norma mínima que satisfaz as restrições primais e
duais. Tal terna de pontos é encontrada fazendo-se
\begin{equation}
	\label{eq:intial-point-LSquare}
	\xtil = A^T(AA^T)^{-1}b, \quad \ytil = (AA^T)^{-1}Ac\quad \text{ e }
\quad \ztil = c - A^T\ytil.
\end{equation}

Além disso, o ponto é transladado para o ortante positivo 
através do uso de constantes  $\vartheta_x>0$ e $\vartheta_ z>0$ tais que  
\begin{equation}
	\label{eq:intial-point-generated}
(x^0,y^0,z^0) = (\xtil+ \vartheta_x e,\ytil,\ztil+\vartheta_z e)
\end{equation}
garantindo que $(x^0,z^0)>0$.


 A rotina \verb|InitialPoint| do PCx é responsável por gerar o ponto inicial, e portanto calcular  $\vartheta_x$ e $\vartheta_ z$. A implementação feita no PCx possui  pequenas alterações em relação ao proposto por \citeauthor{Mehrotra:1992wr}, porém que não  estão documentadas no Manual de Utilização do PCx~\cite{Czyzyk:1998vw}. Por isso, para fins de clareza, documentaremos como o ponto inicial é de fato encontrado, deixando o mérito de tal estratégia para os autores citados.

  Além disso, tal ponto inicial é utilizado em ambos PCx-r e PCx-EOP e a análise de convergência e complexidade relatada no Capítulo~\ref{chap:convergence}, em particular a Condição~\ref{cond:xzzero-xzstar}, continuam válidas ao utilizá-lo para iniciar nosso algoritmo.

O que a rotina  \verb|InitialPoint| faz é primeiramente encontrar  $(\xtil,\ytil,\ztil)$ conforme a Equação~\eqref{eq:intial-point-LSquare}. Calcula-se então  as constantes
\begin{equation}
	\label{eq:initial-point-tilde-var}
\tilde{\vartheta_x} = \max\{\displaystyle\num{-1.5}\cdot\min_{i}\{\xtil_{i}\},\num{e-2}\} \quad \text{ e }\quad  \tilde{\vartheta_z} = \max\{\displaystyle\num{-1.5}\cdot\min_{i}\{\ztil_{i}\},\num{e-2}\}.
\end{equation}
Só então é que se obtém as translações dadas por 

\begin{equation}
	\label{eq:initial-point-var-x}
\vartheta_{x} = \tilde{\vartheta_x} + \num{0.5}\cdot\dfrac{(\xtil+\tilde{\vartheta_x} e)^{T}(\ztil+\tilde{\vartheta_z} e)}{\norm{\ztil+\tilde{\vartheta_z} e}_{1}} 
\end{equation}
e
\begin{equation}
	\label{eq:initial-point-var-z}
\vartheta_{z} = \tilde{\vartheta_z} + \num{0.5}\cdot\dfrac{(\xtil+\tilde{\vartheta_x} e)^{T}(\ztil+\tilde{\vartheta_z} e)}{\norm{\xtil+\tilde{\vartheta_x} e}_{1}} 
\end{equation}
a fim de finalmente produzir o ponto inicial como em \eqref{eq:intial-point-generated}.

No trabalho original de \textcite{Mehrotra:1992wr}, o menor valor possível para  $\tilde{\vartheta_x}$ e $\tilde{\vartheta_z}$ dados pela Equação~\eqref{eq:initial-point-tilde-var} é $\num{0}$, ao invés de \num{e-2}, 	que é o valor utilizado pelo PCx original. Em nossos testes, tanto com PCx-r quanto com PCx-EOP, utilizamos \num{e-1}. 


\subsection{Critério de Parada}


O PCx tem implementado dois critérios de parada que podem ser escolhidos pelo usuário. Como na sua   documentação temos o critério dado em \eqref{eq:termination-criteria}.   Além disso, é possível trocar o critério de parada relacionado à complementaridade. Conforme explanamos na Seção~\ref{subsection:termination-criteria}, uma escolha possível seria a dada na equação \eqref{eq:termination-criteria-pcx}. Porém ao invés do teste dado pela Equação~\eqref{eq:termination-criteria-pcx-gap}, o PCx oferece a possibilidade de utilizar

\[
\dfrac{x^{T}z/n}{1 + \abs{c^Tx}}\leq
	\tol,
\]
em que $\tol$ é a tolerância aceita. Essa escolha tem o inconveniente de tornar o numerador menor, já que o divide por $n$, fazendo com que o programa acuse otimalidade  antes do que deveria. Isso porque, usar esse numerador, equivale a utilizar como tolerância para o critério dado em \eqref{eq:termination-criteria-pcx-gap} $n\cdot\tol$. 

Fizemos uma pequena alteração ao critério de parada, que nos parece  contribuir para um melhor desempenho  no que diz respeito a verificação da otimalidade do ponto em questão. Para tanto reutilizamos a rotina \verb|RecomputeDualVariables| do PCx. Tal rotina  é utilizada no PCx original somente quando já tem-se indicação de otimalidade e o programa saiu do \emph{loop} principal e é responsável por recalcular as variáveis duais relacionadas à  $x\geq0$, de modo à tentar eliminar algum erro de arredondamento.

Descrevemos agora a rotina \verb|RecomputeDualVariables|. Seja $(\xtil,\ytil,\ztil)$ o ponto que o critério de parada do PCx indicou como sendo ótimo. O que a rotina faz, é primeiro calcular o vetor  $t= c - A^T\ytil$. É evidente que, por conta do modo como funcionam os \ac{MPI}, deveríamos ter $t\geq0$ e mais que isso, pela definição de problema dual, deveríamos ter $\ztil = t$. No entanto, note que  se para algum $i$, $t_{i}$

 O código então primeiro testa, para cada  $i = 1,\ldots,n$ se $t_{i}<0$.  Nos casos afirmativos, faz-se $t_{i}\leftarrow 0$. Finalmente estabelece-se que  $\ztil \leftarrow t $, o que finaliza a rotina. Basicamente isso significa que, a rotina garante que a variável dual $\ztil$, esteja na borda do politopo, quando ela deve estar. 




O critério de parada   na implementação em PCx-r e PCx-EOP é o dado na Equação~\eqref{eq:termination-criteria}, porém com uma garantia a mais. Criamos a rotina \verb|RecomputeDualVariablesOpt| que em cada iteração $k$, calcula  com que $(\xk,\yk,\zk)$ passe pela 



\subsection{Solução do subproblema de otimização \texorpdfstring{de $\nextphi$}{da função de mérito}}


\begin{itemize}
\item Descrever a heurística utilizada para resolver o subproblema de otimização
global de polinômios. Artigo escrito em conjunto com os orientadores está sendo
finalizado para submissão em que tal subproblema também aparece, porém num
método similar. Tal trabalho contempla também uma biblioteca para resolver o
subproblema de otimização de polinômios.
\begin{itemize}
  \item Este método similar, já em fase final de implementação,
  demonstra-se competitivo com o \texttt{PCx} nos testes preliminares
 \end{itemize}
\item Implementação do método proposto, utilizando-se da biblioteca supra
citada para resolver os subproblemas de otimização de polinômios.
\item Realizar  testes
numéricos com conjunto de teste da \texttt{Netlib} e complementares, com
comparação ao \texttt{PCx} e com o método desenvolvido pelo grupo.
\end{itemize}



\subsection{Conjunto de Problemas de teste}

\begin{tabular}{llll}
\toprule
Nome     & Linhas  & Colunas  & NEN \\
\midrule
25fv47   & 822   & 1571  & 11127    \\
80bau3b  & 2263  & 9799  & 29063    \\
adlittle & 57    & 97    & 465      \\
afiro    & 28    & 32    & 88       \\
agg      & 489   & 163   & 2541     \\
agg2     & 517   & 302   & 4515     \\
agg3     & 517   & 302   & 4531     \\
bandm    & 306   & 472   & 2659     \\
beaconfd & 174   & 262   & 3476     \\
blend    & 75    & 83    & 521      \\
bnl1     & 644   & 1175  & 6129     \\
bnl2     & 2325  & 3489  & 16124    \\
boeing1  & 351   & 384   & 3865     \\
boeing2  & 167   & 143   & 1339     \\
bore3d   & 234   & 315   & 1525     \\
brandy   & 221   & 249   & 2150     \\
capri    & 272   & 353   & 1786     \\
cycle    & 1904  & 2857  & 21322    \\
czprob   & 930   & 3523  & 14173    \\
d2q06c   & 2172  & 5167  & 35674    \\
d6cube   & 416   & 6184  & 43888    \\
degen2   & 445   & 534   & 4449     \\
degen3   & 1504  & 1818  & 26230    \\
dfl001   & 6072  & 12230 & 41873    \\
e226     & 224   & 282   & 2767     \\
etamacro & 401   & 688   & 2489     \\
fffff800 & 525   & 854   & 6235     \\
finnis   & 498   & 614   & 2714     \\
fit1d    & 25    & 1026  & 14430    \\
fit1p    & 628   & 1677  & 10894    \\
fit2d    & 26    & 10500 & 138018   \\
fit2p    & 3001  & 13525 & 60784    \\
forplan  & 162   & 421   & 4916     \\
ganges   & 1310  & 1681  & 7021     \\
gfrd-pnc & 617   & 1092  & 3467     \\
greenbea & 2393  & 5405  & 31499    \\
greenbeb & 2393  & 5405  & 31499    \\
grow15   & 301   & 645   & 5665     \\
grow22   & 441   & 946   & 8318     \\
grow7    & 141   & 301   & 2633     \\
israel   & 175   & 142   & 2358     \\
kb2      & 44    & 41    & 291      \\
lotfi    & 154   & 308   & 1086     \\
maros    & 847   & 1443  & 10006    \\
maros-r7 & 3137  & 9408  & 151120   \\
modszk1  & 688   & 1620  & 4158     \\
nesm     & 663   & 2923  & 13988    \\
perold   & 626   & 1376  & 6026     \\
pilot    & 1442  & 3652  & 43220    \\
pilot.ja & 941   & 1988  & 14706    \\
pilot.we & 723   & 2789  & 9218     \\
pilot4   & 411   & 1000  & 5145     \\
pilot87  & 2031  & 4883  & 73804    \\
pilotnov & 976   & 2172  & 13129    \\
qap8     & 913   & 1632  & 8304     \\
qap12    & 3193  & 8856  & 44244    \\
qap15    & 6331  & 22275 & 110700   \\
recipe   & 92    & 180   & 752      \\
sc105    & 106   & 103   & 281      \\
sc205    & 206   & 203   & 552      \\
sc50a    & 51    & 48    & 131      \\
sc50b    & 51    & 48    & 119      \\
scagr25  & 472   & 500   & 2029     \\
scagr7   & 130   & 140   & 553      \\
scfxm1   & 331   & 457   & 2612     \\
scfxm2   & 661   & 914   & 5229     \\
scfxm3   & 991   & 1371  & 7846     \\
scorpion & 389   & 358   & 1708     \\
scrs8    & 491   & 1169  & 4029     \\
scsd1    & 78    & 760   & 3148     \\
scsd6    & 148   & 1350  & 5666     \\
scsd8    & 398   & 2750  & 11334    \\
sctap1   & 301   & 480   & 2052     \\
sctap2   & 1091  & 1880  & 8124     \\
sctap3   & 1481  & 2480  & 10734    \\
seba     & 516   & 1028  & 4874     \\
share1b  & 118   & 225   & 1182     \\
share2b  & 97    & 79    & 730      \\
shell    & 537   & 1775  & 4900     \\
ship04l  & 403   & 2118  & 8450     \\
ship04s  & 403   & 1458  & 5810     \\
ship08l  & 779   & 4283  & 17085    \\
ship08s  & 779   & 2387  & 9501     \\
ship12l  & 1152  & 5427  & 21597    \\
ship12s  & 1152  & 2763  & 10941    \\
sierra   & 1228  & 2036  & 9252     \\
stair    & 357   & 467   & 3857     \\
standata & 360   & 1075  & 3038     \\
standgub & 362   & 1184  & 3147     \\
standmps & 468   & 1075  & 3686     \\
stocfor1 & 118   & 111   & 474      \\
stocfor2 & 2158  & 2031  & 9492     \\
stocfor3 & 16676 & 15695 & 74004    \\
truss    & 1001  & 8806  & 36642    \\
tuff     & 334   & 587   & 4523     \\
vtp.base & 199   & 203   & 914      \\
wood1p   & 245   & 2594  & 70216    \\
woodw    & 1099  & 8405  & 37478    \\
\bottomrule
\end{tabular}

\section{Resultados Numéricos}