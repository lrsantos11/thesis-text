%!TEX root = tese.tex
\chapter{Experimentos Numéricos}
\label{chap:numerical}

Neste capítulo descrevemos alguns detalhes de implementação que utilizamos em nossos experimento numéricos, bem os resultados de nosso algoritmo, comparando-o com o  PCx~\cite{Czyzyk:1999hk}.



\section{Detalhes de Implementação}

O Algoritmo \ref{alg:optimized-choice-of-parameters} foi implementado em  \texttt{C/C++}, utilizando como base a implementação PCx~\cite{Czyzyk:1999hk} do algoritmo preditor-corretor de \textcite{Mehrotra:1992wr}. O PCx pode ser configurado para executar correções de ordem superior de \textcite{Gondzio:1996uw} e permitimos que fossem feitas no máximo 2 correções através do arquivo de configuração, além de  refinamento da solução por gradiente conjugado presente na implementação original. Denominamos essa implementação de referência como PCx-r.

  Chamaremos nossa implementação de PCx-EOP. Com efeito, a chamada de PCx-EOP coincide com a da implementação de referência e há a introdução do código de nosso algoritmo dentro do \emph{loop} principal de PCx-r,  retornando em seguida para a rotina principal e dando prosseguimento a sua execução usual. Isso foi feito, para fazermos uma comparação justa entre nossos resultados e os que o PCx-r produz. 

  Assim, herdamos todas as rotinas de preprocessamento -- dentre as quais leitura de dados, precondicionamento e escalonamento, incluindo escalonamento de~\textcite{Curtis:1972cp}, reordenação de linhas e colunas, tratamento de colunas densas e fatoração simbólica de Cholesky --, de ponto inicial, de critério de parada e saída de dados,   bem como todas as rotinas de álgebra linear -- incluindo a estratégia  de \textcite{Ng:1993uz} para resolução de sistemas lineares esparsos via fatoração de Cholesky  e refinamento por gradientes conjugados. Além disso, para problemas canalizados, a estrutura de dados do PCx é feita de forma a fazer o papel da matriz $E$ e das transformações usadas em \eqref{eq:introPL-primal-bounded} e \eqref{eq:introPL-dual-bounded}, tal que os problemas canalizados tenham sua resolução executada com  as mesmas funcionalidades dos problemas não canalizados. 



Além disso, para todos os códigos foram usadas as mesmas opções de compilação e o mesmo computador. Com essa estratégia, diferenças nos tempos de CPU ou no número de iterações podem ser atribuídas apenas à implementação de cada algoritmo.
\subsection{Ponto Inicial}

O ponto inicial dado por \textcite{Mehrotra:1992wr}, foi genericamente descrito na seção \ref{subsec:initial-point}. Ali, mostramos que tal estratégia encontra o ponto $(\xtil,\ytil,\ztil)$, que é solução de norma mínima que satisfaz as restrições primais e
duais. Tal terna de pontos é encontrada através de
\begin{equation}
	\label{eq:intial-point-LSquare}
	\xtil = A^T(AA^T)^{-1}b, \quad \ytil = (AA^T)^{-1}Ac\quad \text{ e }
\quad \ztil = c - A^T\ytil.
\end{equation}

Além disso, o ponto é transladado para o ortante positivo 
através do uso de constantes  $\vartheta_x>0$ e $\vartheta_ z>0$ tais que  
\begin{equation}
	\label{eq:intial-point-generated}
(x^0,y^0,z^0) = (\xtil+ \vartheta_x e,\ytil,\ztil+\vartheta_z e)
\end{equation}
garantindo que $(x^0,z^0)>0$.


 A rotina \verb|InitialPoint| do PCx é responsável por gerar o ponto inicial, e portanto calcular  $\vartheta_x$ e $\vartheta_ z$. A implementação feita no PCx possui  pequenas alterações em relação ao proposto por \citeauthor{Mehrotra:1992wr}, porém que não  estão documentadas no Manual de Utilização do PCx~\cite{Czyzyk:1998vw}. Por isso, para fins de clareza, documentaremos como o ponto inicial é de fato encontrado, deixando o mérito de tal estratégia para os autores citados.

  Além disso, tal ponto inicial é utilizado em ambos PCx-r e PCx-EOP e a análise de convergência e complexidade relatada no Capítulo~\ref{chap:convergence}, em particular a Condição~\ref{cond:xzzero-xzstar}, continuam válidas ao utilizá-lo para iniciar nosso algoritmo.

O que a rotina  \verb|InitialPoint| faz é primeiramente encontrar  $(\xtil,\ytil,\ztil)$ conforme a Equação~\eqref{eq:intial-point-LSquare}. As constantes são calculadas
\begin{equation}
	\label{eq:initial-point-tilde-var}
\tilde{\vartheta_x} = \max\{\displaystyle\num{-1.5}\cdot\min_{i}\{\xtil_{i}\},\num{e-2}\} \quad \text{ e }\quad  \tilde{\vartheta_z} = \max\{\displaystyle\num{-1.5}\cdot\min_{i}\{\ztil_{i}\},\num{e-2}\}.
\end{equation}
Só então é que se obtém as translações dadas por 

\begin{equation}
	\label{eq:initial-point-var-x}
\vartheta_{x} = \tilde{\vartheta_x} + \num{0.5}\cdot\dfrac{(\xtil+\tilde{\vartheta_x} e)^{T}(\ztil+\tilde{\vartheta_z} e)}{\norm{\ztil+\tilde{\vartheta_z} e}_{1}} 
\end{equation}
e
\begin{equation}
	\label{eq:initial-point-var-z}
\vartheta_{z} = \tilde{\vartheta_z} + \num{0.5}\cdot\dfrac{(\xtil+\tilde{\vartheta_x} e)^{T}(\ztil+\tilde{\vartheta_z} e)}{\norm{\xtil+\tilde{\vartheta_x} e}_{1}} 
\end{equation}
a fim de finalmente produzir o ponto inicial como em \eqref{eq:intial-point-generated}.

No trabalho original de \textcite{Mehrotra:1992wr}, o menor valor possível para  $\tilde{\vartheta_x}$ e $\tilde{\vartheta_z}$ dados pela Equação~\eqref{eq:initial-point-tilde-var} é $\num{0}$, ao invés de \num{e-2}, 	que é o valor utilizado pelo PCx original. Em nossos testes, tanto com PCx-r quanto com PCx-EOP, utilizamos \num{e-1}. 


\subsection{Critério de Parada}


O PCx tem implementado dois critérios de parada que podem ser escolhidos pelo usuário. Sua   documentação~\cite{Czyzyk:1998vw} declara que é utilizado o critério dado em \eqref{eq:termination-criteria}.   Além disso, é possível trocar o critério de parada relacionado à complementaridade, dentro do código. Conforme explanamos na Seção~\ref{subsection:termination-criteria}, uma escolha possível seria a dada na equação \eqref{eq:termination-criteria-pcx}. Porém ao invés do teste dado pela Equação~\eqref{eq:termination-criteria-pcx-gap} -- ou por \eqref{eq:termination-criteria-gap} o que seria razoável --, a implementação do PCx usa

\[
\dfrac{x^{T}z/n}{1 + \abs{c^Tx}}\leq
	\tol,
\]
em que $\tol$ é a tolerância aceita. Essa escolha tem um viés dimensional, já que o divide por $n$ o \emph{gap}, fazendo com que o programa acuse otimalidade  antes do que deveria. Isso porque, usar esse teste equivale a utilizar como tolerância para o critério dado em \eqref{eq:termination-criteria-pcx-gap} $n\cdot\tol$. 


Fizemos uma pequena alteração no código, no momento em que testamos o critério de parada, que nos parece  contribuir para um melhor desempenho, pelo menos  no que diz respeito a verificação da otimalidade do ponto em questão. Para tanto, reutilizamos a rotina \verb|RecomputeDualVariables| do PCx. Tal rotina  é utilizada no PCx original somente quando já há a indicação de otimalidade e o programa saiu do \emph{loop} principal.

Descrevemos agora a rotina \verb|RecomputeDualVariables|. Seja $(\xtil,\ytil,\ztil)$ o ponto que o critério de parada do PCx indicou como sendo ótimo. O que a rotina faz, é primeiro calcular o vetor  $t= c - A^T\ytil$. É evidente que, por conta do modo como funcionam os \ac{MPI}, deveríamos ter $t\geq0$ e mais que isso, pela definição de problema dual, deveríamos ter $\ztil = t$. 

 O código então primeiro testa, para cada  $i = 1,\ldots,n$ se $t_{i}<0$.  Nos casos afirmativos, o programa assinala $t_{i}\leftarrow 0$. Finalmente estabelece-se que  $\ztil \leftarrow t $, o que finaliza a rotina. Basicamente isso significa que, a rotina garante que a variável dual $\ztil$, tenha o valor calculado pela definição do problema dual, e não pelo algoritmo. O teste para verificar se $t_{i}$ é negativo serve para detectar possíveis valores inconvenientes, já que $\zstar\geq 0$.

 Pois bem, a alteração que fizemos utiliza os princípios da rotina   \verb|RecomputeDualVariables| \emph{em toda iteração} $k$. Neste caso, o que fazemos é encontrar $t= c - A^T\yk$. Testamos então se para algum  $i$ ocorre $t_{i}<0$ e em caso afirmativo assinalamos $t_{i}\leftarrow 0$. Com isso, consideramos a terna $(\xk,\yk,t)$ com a qual realizamos os testes de otimalidade dados por
 \eqref{eq:termination-criteria-pcx}, porém com as seguintes modificações: o critério \eqref{eq:termination-criteria-pcx-dual} é transformado em
\[\dfrac{\norm{t}}{1 + \norm{c}} \leq \tol,\]	
e o critério  \eqref{eq:termination-criteria-pcx-gap} é transformado em 
\[
	\dfrac{(\xk)^{T}t}{1 + \abs{c^Tt}}\leq\tol.
\]
Se esses testes são satisfeitos, então assinalamos otimalidade no algoritmo e saímos do \emph{loop} principal. 
Isso é feito porque,  como $t$ não é mais interior, $(\xk)^{T}t$ tem possivelmente valor menor que $(\xk)^{T}\zk$. Essa estratégia diz que $(\xk,\yk,t)$ pode ser um ponto melhor que $(\xk,\yk,\zk)$.  Caso não seja satisfeito, continuamos a execução do programa com o ponto atual sendo $(\xk,\yk,\zk)$. Em ambos os casos $t$ é descartado e o ponto utilizado é sempre $(\xk,\yk,\zk)$. Considere que se tivermos otimalidade, \verb|RecomputeDualVariables| será chamada -- pois estamos fora do \emph{loop} e o programa assinala $(\xk,\yk,\zk)\leftarrow (\xk,\yk,t)$. 


O critério de parada usado  na implementação em PCx-r e PCx-EOP é o dado na Equação~\eqref{eq:termination-criteria-pcx}, porém com  a alteração relatada acima. 



\subsection{Solução do subproblema de otimização \texorpdfstring{de $\nextphi$}{da função de mérito}}

Em cada iteração de nosso método, precisamos resolver o problema dado em \eqref{eq:pop-subproblem}, isto é, 
\begin{equation}
	\label{eq:pop-subproblem-1}
	\begin{array}{lc}
\displaystyle \min_{(\al,\mu,\sig)} & \hat\varphi(\al,\mu,\sig) \\
\text{s. a.} &\begin{cases} g_C^i(\al,\mu,\sig) \geq 0 \quad \forall i = 1,\ldots,n \\
				g_L(\al,\mu,\sig)   \geq 0 	\\
				 0\leq (\al,\mu,\sig) \leq u,
				 	
				 \end{cases}.
\end{array}
\end{equation}
em que $\varphi$ bem como $g_{C}^{i}$ e $g_{L}$ são polinômios de grau máximo 3 nas variáveis $(\al,\mu,\sig)$. Este é um chamado Problema de Otimização de Polinômios (POP) e é considerado um problema não linear  difícil de ser resolvido~\cite{Laurent:2010kp}, em particular pela quantidade de restrições do problema, que é  $(m+n+1)$. 

Testamos algumas  implementações especializadas em POP, dentre elas GloptiPoly~\cite{Henrion:2009eb} e SaparsePOP~\cite{Waki:2008ie}, os quais baseiam-se na transformação de um POP em um problema de Programação Semidefinida~\cite{Lasserre:2001fw}. Nenhuma delas, entretanto, foi capaz de resolver os subproblemas. 


O subproblema foi resolvido utilizando da implementação desenvolvida por \textcite{VillasBoas:2012ur,VillasBoas2013:wn}. A estratégia usada por esses autores para resolver um subproblema similar, é a seguinte.

Considere primeiro $\Omega = \{(\al,\mu,\sig): g_C^i(\al,\mu,\sig) \geq, \forall i = 1,\ldots,n,\: g_L(\al,\mu,\sig)   \geq 0,\:		 0\leq (\al,\mu,\sig) \leq u\}$ e considere o problema alternativo
\begin{equation}
		\label{eq:pop-subproblem-2}
	\begin{array}{c}
\displaystyle \min_{\mu} \min_{\sig} \min_{\al}  \nextphi(\al,\mu,\sig) \\
\qquad\qquad\qquad\quad \text{s. a. }  (\al,\mu,\sig) \in \Omega
\end{array},
\end{equation}
ou mais convenientemente 
\[
\displaystyle \min_{\mu}\Psi(\mu)
\]
em que  $\displaystyle \Psi = \min_{\sig}\Phi(\mu,\sig)$ e 
\[
\Phi(\mu,\sig) = 
	\begin{array}{l}
\displaystyle \min_{\al}  \nextphi(\al,\mu,\sig) \\
\text{s. a.}  (\al,\mu,\sig) \in\Omega
\end{array}.
\]
  
Primeiramente, vamos considerar qual relação existente entre os mínimos de \eqref{eq:pop-subproblem-1} e de \eqref{eq:pop-subproblem-2}. Sejam $(\al^*,\mu^*,\sig^*)$ uma solução global de  primeiro e $(\bar{\al}, \bar{\mu},\bar{\sig})$ uma solução global de segundo. Então por definição
$ \nextphi(\al^*,\mu^*,\sig^*) \leq \nextphi (\bar{\al},\bar{\mu},\bar{\sig}).$
Por outro lado 
\[
\nextphi(\baral,\barmu,\barsig) = \min_{\al}\min_{\sig}\Phi(\barmu,\barsig) \leq \Phi(\mu^{*},\sig^{*}) \leq \nextphi(\al^*,\mu^*,\sig^*).
\]
Assim, os dois problemas tem o mesmo mínimo. A segunda formulação, entretanto, permite encontrar aproximações numéricas do mínimo que fazendo o uso de \emph{splines} cúbicas. 



% \begin{itemize}
% \item Descrever a heurística utilizada para resolver o subproblema de otimização
% global de polinômios. Artigo escrito em conjunto com os orientadores está sendo
% finalizado para submissão em que tal subproblema também aparece, porém num
% método similar. Tal trabalho contempla também uma biblioteca para resolver o
% subproblema de otimização de polinômios.
% \begin{itemize}
%   \item Este método similar, já em fase final de implementação,
%   demonstra-se competitivo com o \texttt{PCx} nos testes preliminares
%  \end{itemize}
% \end{itemize}



\subsection{Conjunto de testes}

O conjunto de testes que vamos utilizar foi retirado da Netlib\footnote{Disponíveis em \url{http://www.netlib.org}.}~\cite{Dongarra:1987jk}. Escolhemos primeiramente os \num{95} problemas de programação linear factíveis que estão no diretório \verb|/lp/data/| de tal repositório.  Além disso, do mesmo repositório escolhemos \num{12} dos \num{16} problemas Kennington~\cite{Kennington:1990vo} que se encontram em \verb|/lp/data/kennington/|. Para finalizar, escolhemos 1 problema -- \texttt{qap-8} -- dos \num{3} do conjunto  QAP, que estão disponíves no diretório \verb|/lp/generators/qap|. Os problemas QAP são  construído via um gerador  e  aparecem a partir da linearização de um problema  de designação quadrática. No total então, temos 108 problemas e por isso, tal conjunto de testes é chamado aqui de Netlib-108. 


A Tabela~\ref{tab:netlib108} mostra os \num{108} problemas com seus nomes na primeira coluna. As colunas 2 a 4 mostram o número de linhas, colunas e de canalizações de cada problema, após as rotinas de pré-processamento feitas pelo PCx. Indicamos que um problema  não é canalizado por colocar o número \num{0} na coluna correspondente. A coluna 4 tem o número de elementos não-nulos (NEM) da matriz $A$ das restrições e a última coluna apresenta a densidade da matriz resultante da decomposição de de Cholesky que é para resolução de cada problema.


Alguns comentários sobre a escolha de Netlib-108 cabem aqui. Primeiramente, os \num{4} problemas  Kennington -- \texttt{kennington-ken-13}, \texttt{kennington-ken-18}, \texttt{ken\-ning\-ton-osa-60} e \texttt{ken\-ning\-ton-pds-20} --  e os \num{2} QAP -- \texttt{qap12} e \texttt{qap15} --  que foram deixados de fora dos testes, o foram pois suas estruturas e tamanhos não fazem sentido numa implementação que resolve as equações normais que aparecem em \ac{MPI} via fatoração de Cholesky. 

Além disso, tal gama de problemas é a Netlib mais natural, originária de problemas reais e consensual entre os usuários de \ac{PL}. Com efeito, tais problemas ou pelo menos parte deles, foram testados por \textcite{Mehrotra:1992wr,Colombo:2008ia,Mehrotra:2005do,Jarre:1999tl,Gondzio:1996uw} entre outros pesquisadores com trabalhos importantes da área e estão inclusive na documentação do PCx~\cite{Czyzyk:1998vw,Czyzyk:1999hk}, com a finalidade de demonstrar os testes de seu desempenho. Nesse sentido e assim como esses autores, pensamos que tal conjunto é capaz permitir testes adequados de nosso método. 


{\footnotesize \onehalfspacing


% \tabletail{\midrule  
% \multicolumn{6}{r}{Continua na próxima página.} \\ }
% \tablelasttail{\bottomrule}

\begin{longtable}{>{\ttfamily}lrrrrc}

\caption{\normalfont Conjunto de testes Netlib-108.\label{tab:netlib108}} \\
\toprule
 \textbf{\sffamily Problema}                  & \textbf{\sffamily Linhas} & \textbf{\sffamily Colunas} & \textbf{\sffamily Canalizações} & \textbf{\sffamily NEN} & \textbf{\sffamily Densidade}    \\
\midrule
\endfirsthead


\caption[]{\normalfont Conjunto de testes Netlib-108 (continuação).} \\

\toprule
 \textbf{\sffamily Problema}                  & \textbf{\sffamily Linhas} & \textbf{\sffamily Colunas} & \textbf{\sffamily Canalizações} & \textbf{\sffamily NEN} & \textbf{\sffamily Densidade}    \\
\midrule
\endhead


\bottomrule

\endlastfoot

\midrule  
\multicolumn{6}{r}{\scriptsize Continua na próxima página.}
\endfoot


25fv47       & 788       & 1843      & 0           & 33809    & \num{1,0763E-01} \\
80bau3b      & 2140      & 11066     & 2968        & 41367    & \num{1,7598E-02} \\
adlittle     & 55        & 137       & 0           & 404      & \num{2,4893E-01} \\
afiro        & 27        & 51        & 0           & 107      & \num{2,5652E-01} \\
agg          & 390       & 477       & 0           & 12297    & \num{1,5913E-01} \\
agg2         & 514       & 750       & 0           & 21482    & \num{1,6068E-01} \\
agg3         & 514       & 750       & 0           & 21482    & \num{1,6068E-01} \\
bandm        & 240       & 395       & 0           & 3936     & \num{1,3250E-01} \\
beaconfd     & 86        & 171       & 0           & 820      & \num{2,1011E-01} \\
blend        & 71        & 111       & 0           & 913      & \num{3,4815E-01} \\
bnl1         & 610       & 1491      & 0           & 12089    & \num{6,3338E-02} \\
bnl2         & 1964      & 4008      & 0           & 81275    & \num{4,1632E-02} \\
boeing1      & 331       & 697       & 243         & 5725     & \num{1,0149E-01} \\
boeing2      & 126       & 265       & 73          & 2029     & \num{2,4767E-01} \\
bore3d       & 81        & 138       & 7           & 1034     & \num{3,0285E-01} \\
brandy       & 133       & 238       & 0           & 2755     & \num{3,0397E-01} \\
capri        & 241       & 436       & 131         & 3962     & \num{1,3228E-01} \\
cycle        & 1420      & 2773      & 77          & 56102    & \num{5,4941E-02} \\
czprob       & 671       & 2779      & 0           & 3520     & \num{1,4146E-02} \\
d2q06c       & 2132      & 5728      & 0           & 137349   & \num{5,9965E-02} \\
d6cube       & 403       & 5443      & 0           & 54840    & \num{6,7285E-01} \\
degen2       & 444       & 757       & 0           & 16319    & \num{1,6331E-01} \\
degen3       & 1503      & 2604      & 0           & 120906   & \num{1,0638E-01} \\
dfl001       & 5984      & 12143     & 13          & 1638085  & \num{9,1325E-02} \\
e226         & 198       & 429       & 0           & 3229     & \num{1,5968E-01} \\
etamacro     & 334       & 669       & 131         & 10843    & \num{1,9140E-01} \\
fffff800     & 322       & 826       & 0           & 9573     & \num{1,8155E-01} \\
finnis       & 438       & 935       & 33          & 4984     & \num{4,9676E-02} \\
fit1d        & 24        & 1049      & 1026        & 296      & \num{9,8611E-01} \\
fit1p        & 627       & 1677      & 399         & 627      & \num{1,5950E-03} \\
fit2d        & 25        & 10524     & 10500       & 324      & \num{9,9680E-01} \\
fit2p        & 3000      & 13525     & 7500        & 3000     & \num{3,3300E-04} \\
forplan      & 121       & 447       & 22          & 3304     & \num{4,4307E-01} \\
ganges       & 1113      & 1510      & 397         & 29677    & \num{4,7015E-02} \\
gfrd-pnc     & 590       & 1134      & 258         & 2112     & \num{1,0440E-02} \\
greenbea     & 1933      & 4164      & 264         & 49055    & \num{2,5740E-02} \\
greenbeb     & 1932      & 4154      & 268         & 47783    & \num{2,5085E-02} \\
grow7        & 140       & 301       & 280         & 2730     & \num{2,7143E-01} \\
grow15       & 300       & 645       & 600         & 6090     & \num{1,3200E-01} \\
grow22       & 440       & 946       & 880         & 9058     & \num{9,1302E-02} \\
israel       & 174       & 316       & 0           & 11488    & \num{7,5314E-01} \\
kb2          & 43        & 68        & 9           & 503      & \num{5,2082E-01} \\
lotfi        & 133       & 346       & 0           & 1369     & \num{1,4727E-01} \\
maros-r7     & 2152      & 7440      & 0           & 534188   & \num{2,3023E-01} \\
maros        & 655       & 1437      & 0           & 13454    & \num{6,1192E-02} \\
modszk1      & 665       & 1599      & 0           & 10550    & \num{4,6210E-02} \\
nesm         & 654       & 2922      & 1596        & 21776    & \num{1,0030E-01} \\
perold       & 593       & 1374      & 259         & 21782    & \num{1,2220E-01} \\
pilot.ja     & 810       & 1804      & 331         & 47924    & \num{1,4485E-01} \\
pilot.we     & 701       & 2814      & 287         & 15605    & \num{6,2086E-02} \\
pilot        & 1368      & 4543      & 1040        & 200812   & \num{2,1388E-01} \\
pilot4       & 396       & 1022      & 247         & 14279    & \num{1,7959E-01} \\
pilot87      & 1971      & 6373      & 1578        & 425654   & \num{2,1863E-01} \\
pilotnov     & 848       & 2117      & 332         & 46353    & \num{1,2774E-01} \\
recipe       & 64        & 123       & 56          & 277      & \num{1,1963E-01} \\
sc50a        & 49        & 77        & 0           & 242      & \num{1,8118E-01} \\
sc50b        & 48        & 76        & 0           & 231      & \num{1,7969E-01} \\
sc105        & 104       & 162       & 0           & 569      & \num{9,5599E-02} \\
sc205        & 203       & 315       & 0           & 1156     & \num{5,1178E-02} \\
scagr7       & 127       & 183       & 0           & 730      & \num{8,2646E-02} \\
scagr25      & 469       & 669       & 0           & 2944     & \num{2,4636E-02} \\
scfxm1       & 305       & 568       & 0           & 4430     & \num{9,1965E-02} \\
scfxm2       & 610       & 1136      & 0           & 9284     & \num{4,8261E-02} \\
scfxm3       & 915       & 1704      & 0           & 14138    & \num{3,2681E-02} \\
scorpion     & 340       & 412       & 0           & 2275     & \num{3,6419E-02} \\
scrs8        & 421       & 1199      & 0           & 4477     & \num{4,8143E-02} \\
scsd1        & 77        & 760       & 0           & 1392     & \num{4,5657E-01} \\
scsd6        & 147       & 1350      & 0           & 2545     & \num{2,2875E-01} \\
scsd8        & 397       & 2750      & 0           & 5879     & \num{7,2083E-02} \\
sctap1       & 284       & 644       & 0           & 2435     & \num{5,6859E-02} \\
sctap2       & 1033      & 2443      & 0           & 11736    & \num{2,1028E-02} \\
sctap3       & 1408      & 3268      & 0           & 16100    & \num{1,5532E-02} \\
seba         & 448       & 901       & 447         & 53711    & \num{5,3299E-01} \\
share1b      & 112       & 248       & 0           & 1417     & \num{2,1700E-01} \\
share2b      & 96        & 162       & 0           & 1041     & \num{2,1550E-01} \\
shell        & 487       & 1451      & 117         & 3983     & \num{3,1534E-02} \\
ship04l      & 292       & 1905      & 0           & 2641     & \num{5,8524E-02} \\
ship04s      & 216       & 1281      & 0           & 1778     & \num{7,1588E-02} \\
ship08l      & 470       & 3121      & 0           & 4442     & \num{3,8090E-02} \\
ship08s      & 276       & 1604      & 0           & 2295     & \num{5,6632E-02} \\
ship12l      & 610       & 4171      & 0           & 5506     & \num{2,7955E-02} \\
ship12s      & 340       & 1943      & 0           & 2507     & \num{4,0433E-02} \\
sierra       & 1212      & 2705      & 2016        & 12862    & \num{1,6687E-02} \\
stair        & 356       & 532       & 6           & 14682    & \num{2,2889E-01} \\
standata     & 314       & 796       & 96          & 2395     & \num{4,5397E-02} \\
standgub     & 314       & 796       & 96          & 2395     & \num{4,5397E-02} \\
standmps     & 422       & 1192      & 96          & 3957     & \num{4,2070E-02} \\
stocfor1     & 102       & 150       & 0           & 805      & \num{1,4494E-01} \\
stocfor2     & 1980      & 2868      & 0           & 22841    & \num{1,1147E-02} \\
stocfor3     & 15362     & 22228     & 0           & 177936   & \num{1,4430E-03} \\
truss        & 1000      & 8806      & 0           & 53509    & \num{1,0602E-01} \\
tuff         & 257       & 567       & 25          & 7051     & \num{2,0962E-01} \\
vtp.base     & 72        & 111       & 32          & 505      & \num{1,8094E-01} \\
wood1p       & 171       & 1718      & 0           & 11645    & \num{7,9064E-01} \\
woodw        & 708       & 5364      & 0           & 30027    & \num{1,1839E-01} \\
kennington-cre-a    & 2994      & 6692      & 0           & 33212    & \num{7,0760E-03} \\
kennington-cre-b    & 5336      & 36382     & 0           & 248629   & \num{1,7277E-02} \\
kennington-cre-c    & 2375      & 5412      & 0           & 28528    & \num{9,6940E-03} \\
kennington-cre-d    & 4102      & 28601     & 0           & 212094   & \num{2,4966E-02} \\
kennington-ken-07   & 1437      & 2613      & 2613        & 10034    & \num{9,0220E-03} \\
kennington-ken-11   & 10085     & 16740     & 16740       & 102906   & \num{1,9240E-03} \\
kennington-osa-07   & 1081      & 25030     & 0           & 28276    & \num{4,7469E-02} \\
kennington-osa-14   & 2300      & 54760     & 0           & 60795    & \num{2,2550E-02} \\
kennington-osa-30   & 4313      & 104337    & 0           & 115081   & \num{1,2141E-02} \\
kennington-pds-02   & 2609      & 7339      & 1914        & 44301    & \num{1,2633E-02} \\
kennington-pds-06   & 9156      & 28472     & 8448        & 589339   & \num{1,3951E-02} \\
kennington-pds-10   & 15648     & 48780     & 15125       & 1687660  & \num{1,3721E-02} \\
qap8      & 912       & 1632      & 0           & 193944   & \num{4,6526E-01} \\
% \bottomrule
\end{longtable}
}



\section{Resultados Numéricos}