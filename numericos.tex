%!TEX root = tese.tex
\chapter{Experimentos Numéricos}
\label{chap:numerical}

Vamos escolher um $\gamma\in(0,1)$ adequado, para construir a vizinhança. Várias escolhas são possíveis, como, por exemplo a de \textcite{Colombo:2008ia} que fazem $\ga = 1/10$. No entanto, vamos utilizar a estratégia de \textcite{Zhang:2006ic}
\[
\gamma \leq \frac{\min(x^0z^0)}{(x^0)^Tz^0/n},
\]
que garante que o ponto inicial estará dentro da vizinhança. Mais detalhes sobre a escolha de $\gamma$ será feita na Capítulo~\ref{chap:numerical}, que versa a respeito da implementação.l