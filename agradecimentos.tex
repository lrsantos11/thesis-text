%!TEX root = tese.tex

\addchap*{Agradecimentos}
% FIXME Escrever os agradecimentos.

Agradeço primeiramente a Deus, o qual foi sempre suporte na jornada de muitos anos em busca da realização de meus sonhos.

Agradeço também aos meus familiares, pelo apoio  e  inventivo para que eu pudesse escolher uma profissão com a qual me sentisse feliz e realizado.

Aos meus três orientadores e co-autores desta tese, sem os quais este trabalho não existiria. Ao Aurelio, pela calma, paciência e apoio durante todo trajeto do doutorado. Ao Fernando, pelas inúmeras vezes em que usou toda sua habilidade para me ensinar não só matemática, como  tantas outras  coisas não menos importantes. Ao Clóvis, pela humildade com que soube compartilhar todo seu conhecimento. 

Ao professor Jacek Gondzio e à University of Edinburgh, por me receberem tão bem na Escócia.


Aos professores Chico, Marta, Felipe e Pedro, por fazerem parte da banca bem como à professora Márcia, por participar do exame de qualificação e darem todos, contribuições e dicas valorosas para a finalização deste trabalho.

Aos professores do IMECC Mário Martínez, Nino, Rodney, Joni, Márcia, Moretti, Chico e Carlile,  além dos meus orientadores Aurelio, Fernando e Clóvis. Vocês foram fundamentais na minha formação como  matemático, professor e  pesquisador,  mas sobretudo como pessoa. Tenho a alegria de considerá-los amigos.  

Também a todos os técnicos-administrativos do IMECC, que  sempre me ajudaram prontamente, muitas vezes para além de suas responsabilidades.

Não posso deixar de agradecer aos  colegas e amigos que fiz nesses  anos de Unicamp. Todos vocês fazem parte da minha história e não serão esquecidos.

Agradeço  à Fundação de Amparo à Pesquisa do Estado de São Paulo (FAPESP), pelo suporte financeiro fundamental para a realização deste trabalho.

Finalmente agradeço à minha amada Carolina. Seu amor, caráter, companheirismo e incentivo, sua firmeza, amizade, lealdade e determinação me fazem uma pessoa melhor. Esta tese é dedicada a você, amor.

