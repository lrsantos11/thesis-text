%!TEX root = tese.tex
\addchap{Introdução}



\section*{Objetivos e estrutura deste trabalho}




Desde que o  \acrodef{MPI}{Métodos de Pontos Interiores} \acf{MPI} Primais-Duais ficaram famosos nos anos 90 e que o Método Preditor-Corretor de \textcite{Mehrotra:1992wr} surgiu em \citeyear{Mehrotra:1992wr}, muito tem-se feito na tentativa de estender e melhorar tal método. Algumas das principais questões atuais em \ac{MPI}, incluem o modo como se pode combinar diferentes direções, seja direção corretora, seja preditora, seja ainda alguma outra direção de  ordem superior, tal que uma melhor direção seja gerada em cada passo.  Como veremos a seguir, uma direção pode ser entendida a solução do sistema de Newton para as condições KKT do \ac{PL}. Nesse sentido, as direções utilizadas a cada iteração são encontradas através da solução de um sistema linear com a mesma matriz de coeficientes  -- constante em cada iteração --, porém com vetor do lado direito alterado.


\textcite{Gondzio:1996uw}, por exemplo, combina a ideia inicial preditora-corretora de Mehrotra, porém permitindo múltiplas correções em um mesmo iterando, tentado aumentar o tamanho do passo que pode ser dado, impondo que os iterados estejam o mais perto possível da trajetória central. \textcite{Colombo:2008ia} desenvolvem essa ideia, escolhendo as direções preditora e corretora de forma diferente e permitindo, conforme o caso, quantas correções se deseje. 

Um outro caminho é percorrido por \textcite{Jarre:1999tl} que resolvem, em cada iteração, um subproblema que tenta melhorar os resíduos. Tal subproblema  é em si um problema de PL de dimensão bem  menor em relação ao problema original e cujas variáveis são o peso que cada componente da direção que utilizam deve ter. A solução desse subproblema é feita através do método Simplex. Para tais autores, o subproblema que resolvem pode ser entendido como uma forma de buscar o próximo iterando no subespaço gerado pelas direções que utilizam em seu método.


Nesse mesmo sentido, \textcite{Mehrotra:2005do}, também obtém a direção de busca combinando direções corretoras e preditoras  através de um pequeno subproblema de \ac{PL}. Aqui, as múltiplas  direções corretoras são encontradas fazendo uso de informações geradas a partir de um  subespaço de Krylov apropriado.



Além disso, conforme \textcite{Hung:1996br}, métodos de pontos interiores que são seguidores de caminho  -- veja Seção \ref{sec:path-following-methods} --  geram uma sequência de pontos dentro de uma certa vizinhança da  trajetória central, a qual previne os iterados de prematuramente chegarem muito perto da fronteira da região de factibilidade. Para tanto, é necessário impor condições pré-definidas que façam com que o próximo ponto esteja dentro de uma dessas vizinhanças, garantido  o bom desempenho do método. 




Nesta tese, mostraremos algumas alternativas de respostas para  essas e outras questões correlatas. Assim o objetivo principal desse trabalho é \emph{propor} e \emph{implementar} um Método de Pontos Interiores com pontos infactíves para \ac{PL}, do tipo seguidor de caminho, fazendo  uso de polinômios reais nas variáveis $(\al,\mu,\sig)$, em que $\al$ é o tamanho do passo, $\mu$ é o parâmetro que define a trajetória central, e $\sig$ modela o peso que uma direção corretora deve ter; trata-se portanto de um método preditor-corretor. Neste sentido, esses parâmetros são vistos como variáveis, e sua escolha é feita de forma adiada, através da solução de um subproblema que minimiza uma função de mérito \emph{preditiva} que é um polinômio nessas três variáveis e que está sujeita a restrições que impõe que o próximo iterando esteja dentro de uma vizinhança da trajetória central. A função de mérito é preditiva, no sentido que prediz os resíduos lineares e complementares do próximo iterado. Esse é  o que chamamos de Método de Escolha Otimizada de Parâmetros (EOP).


Uma abordagem similar foi proposta primeiramente por \textcite{VillasBoas:2003tg}, em um contexto auto-dual. Já num contexto primal-dual \textcite{VillasBoas:2012ur,VillasBoas2013:wn}, também fazem uso de uma função de mérito polinomial, porém definem uma trajetória central mais geral, que envolve também os resíduos lineares na sua definição.  Em particular, a abordagem desses autores é utilizada para resolver o subproblema que surge em nosso método.

Além disso, temos o objetivo de \emph{demonstrar} resultados de convergência do método proposto, utilizando as ferramentas de análise numérica para Métodos Preditores-Corretores do tipo Mehrotra infactíveis, apresentadas principalmente por \textcite{Zhang:2006ic} e trabalhos que são sequências desse e das ideias resumidas em \textcite[cap. 7]{Wright:Primal-dual-interior-point:1997h}. Dentre essas ferramentas, há a necessidade de escolher um ponto inicial adequado para que a convergência do método seja garantida.  

Em geral, conforme alerta \textcite[p. 112]{Wright:Primal-dual-interior-point:1997h}, vários autores utilizam pontos inicias que dependem da norma de uma solução ótima, valor que é desconhecido o que torna a implementação do algoritmo impossível, pelo menos da forma descrita.  
 Para contornar tal problemática,  estabeleceremos uma Condição a qual o ponto inicial deve obedecer a fim de que seja possível demonstrar a convergência e a  do algoritmo. Tal condição deve levar  em conta o tamanho dos dados do problema a  ser  resolvido e a distância entre o ponto inicial e uma solução ótima.  O ponto inicial será escolhido conforme a heurística de \textcite{Mehrotra:1992wr} e objetivamos fazer com que tal ponto obedeça esta condição, pelo menos para todo conjunto de testes que utilizamos. 



Em relação a experimentos computacionais, o objetivo é implementar o método proposto e testá-lo. Para os testes, utilizaremos o conjunto de testes \texttt{Netlib}\footnote{Disponível em \url{www.netlib.org/lp/data}.}~\cite{Dongarra:1987jk,Gay:1985ts}. Também objetivamos comparar nosso método com o PCx~\cite{Czyzyk:1999hk}, uma implementação do método preditor-corretor de Mehrotra. 



Com o fim de alcançar  esses objetivos, esta monografia foi organizada com a seguinte estrutura: 

\begin{itemize}

	\item nos Capítulos \ref{chap:linprog} e  \ref{chap:mpis} apresentamos o problema de programação linear, bem como o estado da arte no que diz respeito aos métodos de pontos interiores; 
\item no Capítulo \ref{chap:merit-function} apresentamos o
desenvolvimento de um polinômio real de três variáveis e de grau total 6, como função de mérito  que sirva não só como medida
de factibilidade e otimalidade de um ponto, mas que permita escolher, de forma
otimizada, os parâmetros que determinam uma melhor direção em cada iteração
de um método de pontos interiores primal-dual do tipo seguidor de caminhos; 
\item no Capítulo \ref{chap:convergence} fizemos a análise de convergência e de complexidade do método proposto; 
\item no Capítulo \ref{chap:numerical} relatamos detalhes de nossa implementação, no que diz respeito ao conjunto de testes escolhido, ao ponto inicial, ao critério de parada que utilizamos em nosso algoritmo. Além disso, fazemos uma breve explanação de como se resolve o subproblema de otimização que surge em cada iteração, com base em \cite{VillasBoas2013:wn,VillasBoas:2003tg,VillasBoas:2012ur}. Por fim relatamos os resultados dos experimentos numéricos que foram realizados.


\end{itemize}

 De modo a  concluir esta tese, no último capítulo fizemos considerações finais e encaminhamentos para trabalhos futuros.

