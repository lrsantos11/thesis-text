%!TEX root = tese.tex
\chapter{Resultados Técnicos}
\label{sec:tech-resultas}

\section{Lemas para a prova Teorema \ref{thm:next=residual}}
A prova do Teorema \ref{thm:next=residual} é consequência direta  dos  Lemas
\ref{lemma:linear-residual} e \ref{lemma:nonlinear-residual}, cujas
demonstrações seguem abaixo.

\begin{lema}\label{lemma:linear-residual}
O resíduo da parte linear de  \eqref{eq:ScaledKKT}, para o próximo iterado é
escrito como

\[
\hat{\rho}_L(\al,\mu,\sig) = (1-\al)\rho_L
\]
 
\end{lema}

\begin{proof}
Para a equação \eqref{eq:ScaledKKT-fac-primal} temos
\[
\begin{aligned}
\hat{\rho}_P & = H_P(A\hat{x} -b) = H_P(A(x + \al(\dex +
\Decox )) -b) \\ 
			& = H_P(A(x +  \al\dex) - b) +
 \al H_PA\Decox.
\end{aligned}
\]
Como $A\dex = b- Ax$, vale a seguinte igualdade:
\[
\begin{aligned}
H_P(A(x +  \al\dex) - b) & = H_P((1-\al)A\dex) \\
							& = 	 H_P((\al -1)(b-Ax)) \\
							& =  (1 - \al)\underbrace{H_P(Ax - b)}_{\rho_P} 	\\
							& = (1 - \al)\rho_P.					
\end{aligned}
\]


Por outro lado, pela equação \eqref{eq:corrector-nonlinear}, $A\Dex^c= 0$. Logo
$\hat{\rho}_P = (1-\al)\rho_P$.

A prova para parte dual da factibilidade é similar. Com efeito,

\begin{align}
\hat{\rho}_D & = H_D(A^T\hat{y} + \hat{z} -c)   \notag \\
			& = H_D\left[ A^T\left(y + \al(\dey + \Decoy)\right) + \left(z + \al(\dez + \Decoz )\right)
			-c\right] \notag\\
 			& = H_D\left[A^Ty  + z -c +  \al(A^T\dey + \dez)\right] +
 			\label{eq:next-rho-dual-a}\\
 			& \quad + H_D\underbrace{\left(A^T\Decoy   + \Decoz\right)}_{=0 \text{ por
 			\eqref{eq:corrector-nonlinear}}} \notag 
\end{align}

% 
Como $A^T\dey + \dez = c - A^Ty - z$, a equação \eqref{eq:next-rho-dual-a}
torna-se
\[
H_D\left[A^Ty  + z -c +  \al(A^T\dey + \dez)\right]  = (1-\al)H_D 
(A^Ty + z - c)  = (1-\al)\rho_D.
\]

Portanto, 
\[
\hat{\rho}_D = (1-\al)\rho_D .
\]
\end{proof}
 
 
\begin{lema}\label{lemma:nonlinear-residual}
O resíduo da parte da complementaridade de  \eqref{eq:ScaledKKT}, para o próximo iterado é
escrito como

\begin{equation}
\label{eq:next-residual-complementar}
\hat{\rho}_C = (1-\al)\rho_C + \al\mu e+ \al(\al-\sig)L_{0,0} +
\al^2\La(\mu,\sig).
\end{equation}
em que os vetores $L_{i,k}$, $i,j \in {0,1,2}$ são definidos por
\eqref{eq:defining-Lij}.
\end{lema}

\begin{proof} 
Para encontrar o resíduo da parte complementar para o próximo iterado,
considere que pode-se escrever
 $\hat{\rho}_C  =
\hat{x}\hat{z}$. Então
\begin{align}
\hat{x}\hat{z} & = \left(x + \al(\dex + \Decox) \right)\left( z + \al(\dez + \Decoz
)\right) \notag \\
& = \al^2(\dex+\De^c_x)(\dez + \Dez^c) +  \al\left[  (x\dez+z\dex) +
(x\Decoz + z\Decox)\right] + xz \label{eq:next-residual-complementar-1}
\end{align}


Note que  $ (x\dez+z\dex) = -xz$ por \eqref{eq:affine-scaling-system}. Além
disso usando a equação \eqref{eq:Corrector-spllited}, claramente  $x\Dez^\mu +
z\Dex^\mu = e$ e $x\Dez^\sig + z\Dex^\sig = - \dex\dez$. Então $(x\Dez^c +
z\Dex^c) = \mu e - \sig\dex\dez$.
 
Assim, a equação \eqref{eq:next-residual-complementar-1} torna-se
\begin{subequations}
\begin{align}
\hat{x}\hat{z} & =  \al^2(\dex + \Dex^c)(\dez + \Dez^c) + \al (-xz + \mu   e
- \sig\dex\dez) + xz \notag\\
& = (1-\al) xz + \al\mu e +
\al(\al-\sig)\dex\dez    +
\label{eq:next-residual-complementar-2a} \\
& \quad + \al^2\left(\dex\Dez^c + \dez\Dex^c + \Dex^c\Dez^c\right).
\label{eq:next-residual-complementar-2b}
\end{align}
\end{subequations}

Note que como $\Decow = \mu\Dew^\mu + \sig\Dew^\sig$, então
\begin{align*}
\dex\Dez^c &=  \mu\dex\Dez^\mu + \sig\dex\Dez^\sig,\\
\dez\Dex^c &=  \mu\dez\Dex^\mu + \sig\dez\Dex^\sig,\\
\intertext{e}
\Dex^c\Dez^c& =   (\mu\Dey^\mu + \sig\Dey^\sig) (\mu\Dez^\mu
			+ \sig\Dez^\sig) \\ 
			& = \mu^2\Dex^\mu\Dez^\mu + \mu\sig \left(\Dex^\mu\Dez^\sig +
			\Dex^\sig\Dez^\mu \right) + \sig^2\Dex^\sig\Dez^\sig.
\end{align*}

Consequentemente a equação \eqref{eq:next-residual-complementar-2b} pode ser
expressa como
\begin{multline}
\label{eq:next-residual-complmentar-3b}
\al^2\left(\mu^2\Dex^\mu\Dez^\mu + \mu\sig \left(\Dex^\mu\Dez^\sig +
			\Dex^\sig\Dez^\mu \right) \right. +  \\ \left. + \mu\left(\dex\Dez^\mu +
			\dez\Dex^\mu \right) +   \sig\left(\dex\Dez^\sig + \dez\Dex^\sig\right) +
			\sig^2\Dex^\sig\Dez^\sig \right).
\end{multline}


Basta agora somar  \eqref{eq:next-residual-complmentar-3b}  e
 \eqref{eq:next-residual-complementar-2a}. Definindo os vetores
$\La(\mu,\sig)$ como em \eqref{eq:Lambda-mu-sigma} e $L_{i,j}$ como em
\eqref{eq:defining-Lij} e substituindo onde for possível, finalmente encontra-se
a equação \eqref{eq:next-residual-complementar} levando em conta que $\rho_C = xz$.
\end{proof} 

% \section{Lema para o Teorema \ref{teo:bound-xz}}
% 
% O Lema \ref{lem:svd-AI} é utilizado para demonstrar o Teorema
% \ref{teo:bound-xz}.
% %XXX: Colocar esse lema no capítulo de resultados técnicos
% \begin{lema}
% \label{lem:svd-AI}
% Sejam  $A \in \Real^{m\times n}$, $m<n$ uma matriz de posto completo e a
%   $\tilde{A} = [A^T\: I_n]$, em que $I_n$ é a matriz identidade de ordem
% $n$. Se $\varsigma_i$, para $i=1,\ldots,m$, é valor singular de $A$ -- e de
% $A^T$ --, e $\pi_i$, para $i=1,\ldots,n$, é valor singular de
% $\tilde{A}$ então
% 
% \begin{subequations}
% \label{eq:sing_value_AI}
% \begin{equation}
% \pi_i^2 = \varsigma_i^2 + 1
% \end{equation}
% para $i=1,\ldots,m$ e 
% \begin{equation}
% \pi_i =   1
% \end{equation}
% para $i=m+1,\ldots,n$.
% \end{subequations}
% Consequentemente, o menor valor singular de $\tilde{A}$, $\pi_{\min}$, 
% e igual a $1$.
% \end{lema}
% 
% \begin{proof}
% Os valores singulares de $A^T$ são as raízes quadradas dos autovalores
% distintos da matriz $A^TA$, isto é, existem $v_i\in \Real^n$, $i=1,\ldots,m$ não
% nulos, tais que
% 
% \[
% A^TAv_i = \varsigma^2_iv_i.
% \] 
% 
% Agora, note que 
% \[\tilde{A}\tilde{A}^T = [A^T \: I_n ]\bbm A\\ I_n \ebm = A^TA +
% I_n
% \] 
% e que
% \[
% \tilde{A}\tilde{A}^Tv_i = A^TAv_i + v_i = (\varsigma^2 + 1)v_i.
% \]
% Portanto, para $i=1,\ldots,m$,  $v_i$  é autovetor de
% $\tilde{A}\tilde{A}^T$ com autovalor correspondente a $(\varsigma_i^2 + 1)$. 
% 
% Além disso, note que o posto de $A^TA$ é $m$ e portanto
% a dimensão do núcleo de $A^TA$ é $(n-m)$. Seja $\mathcal{B} =
% \left\{ u_{m+1},\ldots,u_n\right\} $ uma base para o núcleo de $A^TA$. Para
% $u_i\in\mathcal{B}$, tem-se que 
% \[
% \tilde{A}\tilde{A}^Tu_i = A^TAu_i + u_i = u_i.
% \]
% Com isso, para $i=m+1,\ldots,n$,  $u_i$ também é autovetor de
% $\tilde{A}\tilde{A}^T$ com autovalor correspondente a $1$.  
% 
% 
% Consequentemente, sendo os valores singulares de $\tilde{A}$ a raiz quadrada os 
% autovalores de $\tilde{A}\tilde{A}^T$  e considerando que os autovalores são
% únicos valem as equações \eqref{eq:sing_value_AI}.
% \end{proof}


