%!TEX root = tese.tex
\addchap{Considerações finais}
\label{chap:final_remarks}






Neste trabalho procuramos responder a algumas das questões atuais em Métodos de Pontos Interiores primais duais, tais como a utilização de vários tipos de direções e o modo como se deve combiná-las. Neste sentido,  propomos e implementamos  um Método de Pontos Interiores preditor-corretor com pontos infactíves para \ac{PL}, do tipo seguidor de caminho, fazendo  uso de polinômios reais nas variáveis $(\al,\mu,\sig)$, em que $\al$ é o tamanho do passo, $\mu$ é o parâmetro que define a trajetória central, e $\sig$ modela o peso que uma direção corretora deve ter. 

O método proposto foi denominado Método de Escolha Otimizada de Parâmetros (MEOP) já que os  parâmetros $(\al,\mu,\sig)$ foram tratados como variáveis e foram encontrados  através da solução de um subproblema que \emph{minimiza} uma função de mérito polinomial de grau total máximo 6 nessas três variáveis e que está sujeita a restrições que impõem que o próximo iterando esteja dentro de uma vizinhança da trajetória central. 



Além disso, demonstramos resultados de convergência e complexidade do MEOP. Para tanto, estabelecemos a  Condição~\ref{cond:xzzero-xzstar}, a qual leva em conta o tamanho dos dados do problema a  ser  resolvido, e a distância entre o ponto inicial e uma solução ótima. A importância dessa condição surgiu a partir da análise teórica do MEOP, embora ela também tenha implicações práticas.  Ademais, essa condição vale também para outros métodos de pontos interiores infactíveis, em particular aqueles que necessitam em suas análises de limitantes para $\left\{\norm{(\xk,\zk)}\right\}$.

O ponto inicial foi escolhido  conforme a heurística de \textcite{Mehrotra:1992wr} e todo o conjunto de testes que utilizamos obedece a Condição~\ref{cond:xzzero-xzstar}.  Essa escolha leva a um   resultado de complexidade um pouco pior  em relação aos resultados encontrados por outros autores~\cite{Zhang:1995fu,Zhang:2006ic,Wright:1993je,Wright:1996kj}, embora ainda polinomial.


De fato, a melhor complexidade de ordem de iterações para métodos de pontos interiores preditores-corretores do tipo Mehrotra com ponto inicial infactível é dado por~\textcite{Zhang:1996it}, com ordem de iterações $\Oset(n^{\num{1.5}})$ e taxa de convergência Q-subquadrática. Outras complexidades, com ordem de iterações $\Oset(n^{\num{3}})$ foram dadas por \textcite{Zhang:2006ic,Wright:1996kj}. Nossa análise demonstrou que temos uma ordem de iterações um pouco maior, $\Oset(n^{\num{4}})$, mas ainda polinomial, e temos convergência Q-linear. 

Alguns  comentários são importantes. Tanto quanto sabemos, os algoritmos de \textcite{Zhang:1996it,Zhang:2006ic,Wright:1996kj} não foram implementados. Além disso, \textcite{Gertz:2003ji} utilizam em sua implementação um ponto inicial que também tem alguma relação com o tamanho dos dados do problema, porém, não há qualquer análise de complexidade ou convergência nesse trabalho. Outro detalhe importante é que  grande parte das análises de convergência recentes~\cite{Gondzio:2011ta} supõe que o ponto inicial é factível. 

Embora os resultados teóricos de nosso método  não superem os melhores resultados teóricos existentes, cabe ressaltar que, no nosso caso, a análise feita acima é do algoritmo que foi de fato implementado, incluindo-se aí o tratamento do pontos infactíveis, o ponto inicial de Mehrotra e os critérios de parada detalhados. Com isso, conseguimos um resultado teórico considerável de um algoritmo que tem seu funcionamento na prática demonstrado.



Foram feitos então experimentos computacionais com o MEOP.  Para os testes, utilizamos o conjunto de testes \Netlib-108, escolhido entre os problemas do respostório \Netlib. Comparamos nosso método com o PCx~\cite{Czyzyk:1999hk}, uma implementação do método preditor-corretor de Mehrotra. De fato, relatamos também as modificações que fizemos na escolha do ponto inicial e no critério de parada do PCx, de modo a melhorar seu desempenho. Essas modificações foram utilizadas tanto na nossa implementação, a qual chamamos PCx-EOP quanto a de referência, que denominamos PCx-r. Os testes mostraram que nosso algoritmo é competitivo e robusto, quando comparado com uma implementação já bem testada e madura, como o PCx. 



\minisec{Perpectivas futuras}


O desenvolvimento do Método de Escolha Otimizada de Parâmetros trouxe algumas questões que 
não foram  abordadas nessa tese. Algumas dessas questões podem servir como guia para futuras pesquisas.

Uma ideia interessante é concernente às matrizes $H_P$ e $H_D$  -- veja Equação
\eqref{eq:defining_matrices_H}. Tais matrizes podem ser adaptadas, não só garantido a
não-negatividade dos resíduos primais e duais, mas servindo, de certa maneira  de  fator de escalonamento do sistema o que geraria escalamento da função de mérito. Assim, poderíamos inclusive incluir um escalonamento da parte da complementaridade de KKT e verificar como tal escalonamento impactaria no despenho do MEOP.


Quanto à análise do método, a fim de atacar o problema escolhemos um valor para $\mu\neq 0$ e para $\sig = 0$, de modo a garantir que o conjunto $\Omega$ das restrições do subproblema resolvido em cada iteração fosse não-vazio.  No entanto, quando da solução de problemas práticos, em grande parte das iterações o método utilizado encontrou ótimo para $\mu=0$ e $\sig\neq 0$. Embora a demonstração continue válida, a utilização de outros valores para $\mu$ e $\sig$ pode fazer com que a complexidade seja melhorada. Além disso, outro caminho na análise seria tentar verificar se é possível demonstrar convergências de ordem maior, como quadrática ou subquadrática. 

Uma investigação  que pretendemos fazer é quanto à Condição~\ref{cond:xzzero-xzstar}. Essa condição permite que estudemos relações entre o ponto inicial, a solução ótima e o tamanho dos dados. Embora tenhamos escolhido o ponto inicial de Mehrotra, outro ponto inicial que satisfaça essa Condição tem garantia de convergência do MEOP. Ademais, conforme dissemos, poderemos utilizar essa condição em outras análises de convergência. Daí testes podem surgir com outros pontos iniciais no PCx-EOP ou em outras implementações de pontos interiores.



Em resumo, podemos dizer que através das ideias desenvolvidas nesta tese alcançamos os objetivos a que nos propusemos e, mesmo assim, tais ideias  nos deram muitos caminhos a serem percorridos. Isso mostra que tanto  a teoria quanto a implementação de Métodos de Pontos Interiores ainda são um vasto campo de pesquisa a ser explorado. De fato, poucos trabalhos têm desenvolvido esses dois vieses de forma conjunta, o que foi feito  em nossa pesquisa. Tal fato, por si só, é suficiente para demonstrar a contribuição desta tese para o estado da arte na área de pontos interiores. 