%!TEX root = tese.tex
\addchap{Perspectivas Futuras}
\label{chap:final_remarks}

Para a finalização da tese, pretende-se  terminar de demonstrar  e
descrever os resultados de convergência do método, bem como realizar
experimentos numéricos que comparem o método com códigos mais maduros, como
o \texttt{PCx}. Os passos que pretendem ser dados nessas duas fases seguem
descritos abaixo.

\addsec{Resultados de Convergência}
\begin{itemize}
	\item A convergência do método proposto no Pseudo-código
	\ref{alg:optimized-choice-of-parameters} está em fase de demonstração. Seguem
	um roteiro do que pretende ser seguido para finalizar a demonstração.
	\item \minisec{Roteiro para a demonstração}
	\begin{itemize}
  		\item Supor $\bar{\sig} = 0$ e $\bar{\mu}=\eta
  		x^Tz/n$  e encontrar $\al_k\in(0,1]$ tal que 
  		\[\varphi_{k+1} = (1 -\theta(\al_k))\varphi_k.\]
  		em que $\theta(\al_k) \in(0,1)$. Como consequência, a sequência
  		$\{\varphi_k\}$ satisfaz 
  		\[
  		\varphi_{k+1} < \varphi_k.
  		\] 
  		e logo ter-se-á convergência Q-linear.
  		
  		\item Esta é a abordagem usada por  \textcite{Zhang:2006ic} para um método
  		infactível.
  		\item Na prática o método deve ter desempenho melhor do que Q-linear,
  		já que se buscará, em cada iteração $k$,
  		o minimizador global
  		de $\varphi_k$, dado por $(\al^*,\mu^*,\sig^*)$ e portanto
  		$\varphi_k(\al^*,\mu^*,\sig^*)\leq \varphi_k(\al_k,\bar{\mu},\bar{\sig})$.

  		
  	\end{itemize}
  	
  	  		\item Pretende-se também encontrar estimativas para um limitante superior
  	  		para $\norm{(x^*,z^*)}_\infty$. Este limitante é mencionado nas
  	  		demonstrações de convergência de métodos
  		de pontos interiores infactíveis e indica como escolher um ponto inicial
  		que dependa desse limitante e é utilizado na garantia da convergência. Não
  		há porém -- pelo menos de nosso conhecimento --, estimativas para o esse
  		limitante.


\item As matrizes $H_P$ e $H_D$ (veja Equação
\eqref{eq:defining_matrices_H}) podem ser adaptadas, não só garantido a
não-negatividade dos resíduos primais e duais, mas como fator de escala.
Esse escalamento pode ser feito tal que se
$\varphi_k<\eps$, para algum $k$, então o critério de parada do 
\texttt{PCx} (veja Equação \eqref{eq:termination-criteria-pcx}) também será
satisfeito.
\end{itemize}


\addsec{Experimentos Numéricos}
\begin{itemize}
\item Descrever a heurística utilizada para resolver o subproblema de otimização
global de polinômios. Artigo escrito em conjunto com os orientadores está sendo
finalizado para submissão em que tal subproblema também aparece, porém num
método similar. Tal trabalho contempla também uma biblioteca para resolver o
subproblema de otimização de polinômios.
\begin{itemize}
  \item Este método similar, já em fase final de implementação,
  demonstra-se competitivo com o \texttt{PCx} nos testes preliminares
 \end{itemize}
\item Implementação do método proposto, utilizando-se da biblioteca supra
citada para resolver os subproblemas de otimização de polinômios.
\item Realizar  testes
numéricos com conjunto de teste da \texttt{Netlib} e complementares, com
comparação ao \texttt{PCx} e com o método desenvolvido pelo grupo.
\end{itemize}


