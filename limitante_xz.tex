
 \section{Limitante para $\norm{(x^*,z^*)}_\infty$}


Nas análises de convergência de \ac{MPI} infactíveis em geral há  a
necessidade de se escolher um ponto inicial que reflita de alguma maneira o
\emph{tamanho} do vetor $(x^*,z^*)$~\cite{Zhang:2006ic,Wright:1994jd}.
Nessas análises, o tal medida  é estimada através de um
limitante para $\norm{(x^*,z^*)}_\infty$. Embora esse limitante seja utilizado nas
demonstrações, não há qualquer indicativo de como é possível estimá-lo.


Para darmos uma estimativa para tal limitante, vamos abordar primeiramente o
problema primal, isto é,

\begin{equation} %(P) \quad
	\begin{array}{lc}
\displaystyle \minimizar_{x} & c^Tx \\
\text{sujeito a} &\begin{cases} Ax = b \\
				 x \geq 0	
				 \end{cases}.
\end{array}  
\label{eq:primal-bound}
\end{equation}

Considere-se que um conjunto factível não-vazio de um
\ac{PL} é um conjunto poliedral composto de faces e vértices ou pontos
extremos que depende apenas de $A$ e de $b$.  Além disso, como
$A\in\Real^{m\times n}$, $m<n$ e posto de $A$ é completo, então podemos definir,
sem perda de generalidade, um vértice através do vetor $x  = \bbm x _B \\
x _N\ebm$ em que $x _B\in\Real^m$ é chamada solução básica  e
$x _N\in\Real^{n-m}$ é chamada solução não-básica, com $x _N = 0$. Caso as
$n$ primeiras componentes de $x $ não sejam correspondentes à solução básica
em questão, utiliza-se uma transformação ortogonal de troca de linhas.  Note-se
também que se existe uma solução ótima para o \ac{PL}, então existe um vértice que também é ótimo~\cite{Bazaraa:2009uu}.

O próximo lema utiliza a Decomposição em Valor Singular (SVD) de $A$ para
encontrar um limitante superior para todo solução básica factível.



\begin{lema}
\label{lemma:bound-xb}
Seja  $\Pset$ o conjunto das soluções factíveis primais. Para todo ponto extremo
$x \in\Pset$, existe um escalar $\zeta_x>0$ tal que
\begin{equation}
\label{eq:inequalityzetax}
\norm{x }_\infty \leq \zeta_x.
\end{equation}
\end{lema}


\begin{proof}
Suponha que   $x$ seja ponto extremo não nulo. Sem
perda de generalidade, podemos escrever a matriz $A$ como $A = [B\quad N]$, em
que o bloco $B\in\Real^{m\times m}$  é não-singular e $N\in\Real^{m\times
(n-m)}$. Além disso,  $x = \bbm x _B
\\
x _N\ebm$, com $x _N = 0$. 


Com isso, temos que \[b = Ax = Bx_B \text {\quad
e \quad 	} \norm{x} =\norm{x _B}.\]

Seja $\varsigma_m$ o menor valor-singular de $B$. Neste caso, o menor valor
singular e $B$ é menor que 

Seja SVD completa de $A$ dada por $A = U\Sigma V^T$, em que
$U\in\Real^{m \times m}$ e $V\in\Real^{n\times n}$ são matrizes ortonormais e
$\Sigma\in\Real^{m\times n}$ é matriz diagonal tal que 	
 \[
 \Sigma = \left[ C\quad \bm{0}\right] ,
 \]
  $C=\diag(\varsigma_1,\ldots,\varsigma_m)\in\Real^{m\times m}$ e
  $\varsigma_1\geq\varsigma_2\geq \cdots \geq \varsigma_m >0$ e matriz
  $\bm{0}\in\Real^{m\times (n-m)}$ é composta apenas por zeros.
  Os escalares $\varsigma_i$, para $i=1,\ldots,m$ são os  valores singulares de
  $A$.

Neste caso podemos escrever 
\[
	 b = Ax \Longleftrightarrow  U^Tb =  U^TAx  \Longleftrightarrow  U^Tb = U^TU\Sigma
	 V^Tx \Longleftrightarrow  \Sigma V^Tx =  U^Tb. 
\]


Note agora que 
\[V = \bbm \hat{V} \\  \tilde{V}\ebm,\]
em que $\hat{V}\in\Real^{n\times m}$  é composta pelas $m$
primeiras linhas ortonormais de $V$ e $\tilde{V}\in\Real^{n\times (n-m)}$,
composta pelas linhas  restantes de $V$. Com isso

\[
 \left[ C\quad \bm{0}\right] \bbm \hat{V}^T &  \tilde{V}^T\ebm \bbm x _B \\
x _N\ebm = U^Tb.
\]

Assim, 
\begin{equation*}
% 		\label{eq:sigmaVxbar} 
\left[ C\quad \bm{0}\right]  \hat{V}^T x_B +  \underbrace{\tilde{V}^Tx_N}_{=
\bm{0}} = U^Tb,
 \end{equation*}
ou ainda
\begin{equation}
		\label{eq:sigmaVxbar} 
\left[ C\quad \bm{0}\right]  \hat{V}^T x_B = U^Tb.
 \end{equation}


Assim, usando \eqref{eq:sigmaVxbar},
temos que

\[
C\hat{V}^Tx_B + \bm{0}  = U^Tb \Longleftrightarrow \hat{V}^Tx_B =  C^{-1} U^Tb.
\]
Note ainda que por conta da ortogonalidade, valem as  igualdades
$\norm{\hat{V}^Tx } = \norm{x }$ e $\norm{ U^Tb} = \norm{b}$. Logo,

\[
\norm{\hat{V}^Tx } = \norm{C^{-1} U^Tb}  \Longleftrightarrow \norm{x } =
\norm{C^{-1}U^Tb}.
\]

Nessas condições é possível obter as  desigualdades
\begin{equation}
\label{eq:xbarnormrealtions}
\norm{x }_\infty \leq \norm{x } = \norm{C^{-1}U^Tb} \leq
\norm{C^{-1}}\norm{b},
\end{equation}
utilizando na primeira desigualdade a relação entre normas de vetores e na
última consistência de normas e ortogonalidade de $U$.

Note que como $C^{-1}$ é diagonal,  $\norm{C^{-1}} =
\varsigma_m^{-1}$. Desta forma, definindo $\zeta_x \geq
\varsigma_m^{-1}\norm{b}$ e usando \eqref{eq:xbarnormrealtions}, obtemos a desigualdade
\eqref{eq:inequalityzetax}, como queríamos.

\end{proof}
% 
% 
%  
%  
%  
% 
% \begin{teo}
% \label{teo:bound-xz}
% Sejam $\Sset_P$ o conjunto das soluções ótimas primais e
% $\Sset_D$ o conjunto das soluções ótimas duais para o par primal-dual
% \eqref{eq:primal} e \eqref{eq:dual}. Se $\Sset_P$ e $\Sset_D$ forem limitados
% então para toda solução primal  $x^*\in\Sset_P$ e correspondente solução dual
% $(y^*,z^*)\in\Sset_D$ existe $\zeta>0$ tal que
% \begin{equation}
% \label{eq:bound_x*z*}
% \norm{(x^*,z^*)}_\infty \leq \zeta
% \end{equation}
% \end{teo}
% 
% \begin{proof}
%  Para construir um limitante para $\norm{x^*}_\infty$, suponha que $Qx = (x_B,
%  x_N)$ seja tal que $x_B\in\Real^m$ é uma solução básica e $x_N\in\Real^{n-m}$ uma solução não-básica de
%  \eqref{eq:primal}.
%   Seja $U\Sigma V^T$ a SVD de $A$, em que
%  \[
%  \Sigma = \left[ C\quad  0\right] ,
%  \]
%   $C=\diag(\varsigma_{\max},\ldots,\varsigma_{\min})$ e $\varsigma_{\min}$ e
%  $\varsigma_{\max}$ são o menor e o maior valor singular de $A$.
%  Com isso, $Ax= U\Sigma V^Tx $ e logo $\Sigma V^T x = U^Tb$.
% 
%  Seja
%   \[
%  \Sigma^\dagger = \bbm C^{-1}   \\   0 \ebm
%  \]
% a matriz pseudo-inversa de $\Sigma$. Logo
% $\Sigma^\dagger\Sigma V^Tx = \Sigma^\dagger U^Tb$ e portanto
% \[
% \norm{\bbm I_m & 0\\0 & 0 \ebm V^T x} = \norm{\Sigma^\dagger U^Tb}
% \]
% 
% Como $U$ e $V$ são ortogonais, temos que
% $\norm{x_B} \leq \norm{\Sigma^\dagger}\norm{b}$ e logo
% \[
% \norm{x_B}_\infty \leq \norm{x_B}\leq \frac{1}{\varsigma_{\min}} \norm{b}
% \]
% 
% Note que, como por hipótese $\Sset_P$ é limitado, qualquer solução ótima
% $x^*$ de \eqref{eq:primal} é uma combinação convexa de soluções ótimas básicas, i.e.,
% \[
% x^*= \sum_{\ell=1}^p t_\ell\tilde{x}_\ell
% \]
% em que $\sum_{\ell=1}^p t_\ell  = 1$ e para $\ell=1,\ldots,p$
% tem-se $\tilde{x}_\ell = (\tilde{x}_B | \tilde{x}_N)_\ell$  e
% $(\tilde{x}_B)_\ell$ como uma solução ótima básica.
% 
%  Portanto, existe $\zeta_x$
% escalar positivo, tal que
% \[
% \norm{x^*}_\infty \leq \zeta_x
% \]
% em que \[\zeta_x = \sum_{\ell=1}^p t_{\ell}\frac{1}{\varsigma_{\min}}
% \norm{b} = \frac{1}{\varsigma_{\min}} \norm{b}.
% \]
% 
% 
% Por sua vez, o limitante para $\norm{z^*}_\infty$ surge quando considera-se
% que as restrições do problema dual \eqref{eq:dual} podem ser reescritas na
% forma padrão de um \ac{PL}, definindo $\tilde{A} = [A^T \; I]$ e $\tilde{z} = [y
% \; z]$. tal que $\tilde{A}\tilde{z} = c$.
% 
% Assim, sejam $\pi_i$, para $i=1,\ldots,n$, os valores singulares de $\tilde{A}$.
% Com ideias similares às usadas na primeira parte da demonstração deste teorema,
% é possível verificar que a seguinte desigualdade \[
% \norm{\tilde{z}^*}_\infty \leq \frac{1}{\pi_{\min}}
% \norm{c}
% \] é verdadeira.
% 
% 
% Note que $\norm{z^*}_\infty\leq \norm{\tilde{z}^*}_\infty$. Além disso, o Lema
% \ref{lem:svd-AI}  mostra que  $\pi_{\min} = 1$. Portanto,
% \[
% \norm{z^*}_\infty \leq  \zeta_z,
% \]
% em que $\zeta_z = \norm{c}$.
% 
% 
% Para finalizar, basta definir  $\zeta = \max\{\zeta_x,\zeta_z\}$ e o teorema
% está provado.
% \end{proof}


\begin{lema}
\label{lem:svd-AI}
Sejam  $A \in \Real^{m\times n}$, $m<n$ uma matriz de posto completo e a
  $\tilde{A} = [A^T\: I_n]$, em que $I_n$ é a matriz identidade de ordem
$n$. Se $\varsigma_i$, para $i=1,\ldots,m$, é valor singular de $A$ -- e de
$A^T$ --, e $\pi_i$, para $i=1,\ldots,n$, é valor singular de
$\tilde{A}$ então

\begin{subequations}
\label{eq:sing_value_AI}
\begin{equation}
\pi_i^2 = \varsigma_i^2 + 1
\end{equation}
para $i=1,\ldots,m$ e
\begin{equation}
\pi_i =   1
\end{equation}
para $i=m+1,\ldots,n$.
\end{subequations}
Consequentemente, o menor valor singular de $\tilde{A}$, $\pi_{\min}$,
e igual a $1$.
\end{lema}

\begin{proof}
Os valores singulares de $A^T$ são as raízes quadradas dos autovalores
distintos da matriz $A^TA$, isto é, existem $v_i\in \Real^n$, $i=1,\ldots,m$ não
nulos, tais que

\[
A^TAv_i = \varsigma^2_iv_i.
\]

Agora, note que
\[\tilde{A}\tilde{A}^T = [A^T \: I_n ]\bbm A\\ I_n \ebm = A^TA +
I_n
\]
e que
\[
\tilde{A}\tilde{A}^Tv_i = A^TAv_i + v_i = (\varsigma^2 + 1)v_i.
\]
Portanto, para $i=1,\ldots,m$,  $v_i$  é autovetor de
$\tilde{A}\tilde{A}^T$ com autovalor correspondente a $(\varsigma_i^2 + 1)$.

Além disso, note que o posto de $A^TA$ é $m$ e portanto
a dimensão do núcleo de $A^TA$ é $(n-m)$. Seja $\mathcal{B} =
\left\{ u_{m+1},\ldots,u_n\right\} $ uma base para o núcleo de $A^TA$. Para
$u_i\in\mathcal{B}$, tem-se que
\[
\tilde{A}\tilde{A}^Tu_i = A^TAu_i + u_i = u_i.
\]
Com isso, para $i=m+1,\ldots,n$,  $u_i$ também é autovetor de
$\tilde{A}\tilde{A}^T$ com autovalor correspondente a $1$.


Consequentemente, sendo os valores singulares de $\tilde{A}$ a raiz quadrada os
autovalores de $\tilde{A}\tilde{A}^T$  e considerando que os autovalores são
únicos valem as equações \eqref{eq:sing_value_AI}.
\end{proof}

