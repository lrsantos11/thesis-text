\addchap*{Lista de Símbolos}  % \markboth{}{} é utilizado para corrigir o cabeçalho.

\begin{description}
    % FIXME Remover os dois símbolos abaixo e incluir as que serão
    % utilizadas.

	    \item[$\Natural$] Conjunto dos números naturais.

    \item[$\Real$] Conjunto dos números reais.

    \item[$m, n, p, q$] Números naturais.

    \item[$\Real^n$] Espaço euclidiano de  números reais de dimensão $n$.

    \item[$\infty$] Infinito.

    
    \item[$\norm{\cdot}$] Norma-2 de um vetor real.

    \item[$\norm{\cdot}_{2}$] Norma-2 de um vetor real.

    \item[$\norm{\cdot}_{1}$] Norma-1 de um vetor real.

    \item[$\norm{\cdot}_{\infty}$] Norma-$\infty$ de um vetor real.

     \item[Letras romanas minúsculas] Vetores reais. Se $v\in\Real^{n}$, então $v_{i}$ é a $i$-ésima componente de $v$.

    \item[Letras romanas maiúsculas] Matrizes reais. Se $A\in\Real^{m\times n}$, $a_{ij}$ é a componente de $A$ que está na $i$-ésima linha e na $j$-ésima coluna.

    \item[Letras gregas] Escalar ou constante real.

    \item[$k$] Indica índice de iteração de um vetor ou escalar. Em um vetor é expoente, em um escalar é índice.  Para $v\in\Real^{n}$, $v^{k}$ representa vetor na iteração $k$ e para $\al\in\Real$, $\al_{k}$ representa o escalar $\al$ na iteração $k$. 



    \item[$(\al, \be)$] Intervalo aberto de $\al$ até $\be$.

    \item[{$[\al, \be]$}] Intervalo fechado de $\al$ até $\be$.

    \item[$(u,v)$] Quando $u, v\in\Real^n$, representa o vetor de $\Real^{2n}$ com as $n$ primeiras componentes iguais às $n$ componentes de  $u$ e  as $n$ últimas componentes iguais às $n$ componentes de $v$.

    \item[$uv$]  Produto de Hadamard~\cite[p.~455]{Horn:1985tf} entre $u$ e $v$, 
isto é, $uv$ é o vetor em $\Real^n$, tal que cada componente  $(uv)_i  = u_iv_i$, para $i=1,\ldots,n$

    \item[$A^T$] Matriz transposta de $A$.

    \item[$\diag(A)$] Diagonal da matriz $A$. 

    \item[$\diag(x)$] Matriz diagonal com vetor $x$ na diagonal principal. Utilizamos $X = \diag(x)$ para indicar a matriz diagonal $X$ cuja diagonal principal seja $x$.
 
    \item[$e$] Vetor com todas as componentes iguais a um, de dimensão apropriada.

    \item[$t$] Representa sempre uma variável local, podendo ser escalar ou vetor, de acordo com o contexto.


    \item[$\Oset(n)$] Se $\al = \Oset(n)$ então existe constante $\be$ tal que $\al \leq \be n$.

    \item[$a\leftarrow b$] Assinala para a variável $a$ o valor da variável $b$.

\end{description}
