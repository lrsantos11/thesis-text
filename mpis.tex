%!TEX root = tese.tex
\setchapterpreamble[u]{% 
\dictum[N. Karmarkar]{``In the simplex method, the
current solution is modified by introducing a nonzero coefficient for one of the
columns in the constraint matrix.
Our method allows the current solution to be modified by introducing several
columns at once''~\cite{Karmarkar:1984cp}}}


\chapter{Métodos de Pontos Interiores Primais-Duais  para \acl{PL}\label{chap:mpis}}


     

Este capítulo tem o objetivo de apresentar os \acl{MPI} primais-duais seguidores
de caminho. A teoria central que versa sobre \ac{MPI} está bem estabelecida e há
ótimos textos sobre o
assunto~\cite{Vanderbei:2008vp,Wright:Primal-dual-interior-point:1997h}.
Recentemente \textcite{Gondzio:2011ta} fez um resumo histórico no qual resgata as
principais contribuições no tema.  Estas referências servem como texto
introdutório e foi com base nelas e nos trabalhos de 
 \textcite{Colombo:2008wm} e \textcite{Villas-Boas:2000}, que este capítulo foi
 escrito.
    

     

   
   
\section{Um método de  pontos interiores}
O  problema de \ac{PL} na forma padrão, 
\begin{equation} %(P) \quad
	\begin{array}{lc}
\displaystyle \min_{x} & c^Tx \\
\text{sujeito a} &\begin{cases} Ax = b \\
				 x \geq 0	
				 \end{cases}\
\end{array}
\label{eq:primal}
\end{equation}
é chamado de \emph{problema primal}.

Associado a todo \ac{PL} existe um outro \ac{PL} chamado 
\emph{problema dual}, o qual consiste nos mesmos dados arranjados de maneira
diferente. O dual de \eqref{eq:primal} é
 \begin{equation}%(D) \quad
	\begin{array}{lc}
\displaystyle \max_{(y,z)} & b^Ty \\
\text{sujeito a} &\begin{cases} A^Ty + z = b \\
				 z \geq 0, \:y \text{ livre}	
				 \end{cases}\
\end{array}
\label{eq:dual}
\end{equation}
em que $A\in \Real^{m\times n}$ é uma matriz de posto completo, $c,x,z\in
\Real^n$, $y,b\in\Real^m$ e $m<n$. Chamamos o vetor $y$ de
\emph{variáveis duais}, enquanto que  $z$ é o vetor das \emph{folgas duais}.
Note que, como $A$ tem posto completo, existe uma relação unívoca entre
$y$ e $z$. 

Sejam
\begin{equation}
\label{eq:primal-dual-feas-set}
\Pset = \{x : Ax = b, x\geq 0\} \quad\text{ e } \quad
\Dset = \{ (y,z) : A^Ty + z = b, z\geq 0\}
\end{equation}
 os conjuntos de pontos factíveis primais e duais 
respectivamente. Usando essa notação, o par primal-dual de
(\ref{eq:primal}-\ref{eq:dual}) pode ser reescrito como
\begin{equation}
\label{eq:primal-dual-reescrito}
\min_x\; c^Tx \; \text{ s.a. }\; x\in \Pset \qquad\text{ e } \qquad
\max_{(y,z)}
b^Tx \; \text{ s.a. }\; (y,z) \in \Dset.
\end{equation}


Denomina-se $\Fset = \Pset\times\Dset$ o conjuntos dos pontos primais-duais
factíveis.
 Para fins de notação, pode-se dizer que se  $w = (x,y,z)$ é primal-dual factível, então $w\in\Fset$. Define-se também os conjuntos \[ \Qset =
 \{w\in\Real^{2n+m}:(x,z)\geq 0\} \] e \[ \Qset^+ = \{w\in\Real^{2n+m}:(x,z) >
 0\} \] respectivamente  como os
conjuntos dos pontos de $\Real^{2n+m}$ não-negativos e positivos nas variáveis
$x$ e $z$.



Com isso, seja
\begin{equation}
  	\label{eq:F+}
  	\Fset^+ = \Fset \cap \Qset^+
  \end{equation}   o conjunto dos
pontos primais e duais interiores. Como o próprio nome sugere, em Métodos de 
Pontos Interiores dedica-se especialmente a estudar iterados que estejam
contidos em $\Fset^+$, no caso de métodos factíveis, ou iniciando com pontos em $\Qset^{+}$, no caso de métodos infactíveis. 


Neste trabalho, o seguinte pressuposto é assumido como verdadeiro:

\begin{pressup}\label{ass:interior-nonempty}
 O conjunto $\Fset^+$ é não-vazio, isto é, o ambos os conjuntos $\Pset$ e
 $\Dset$ contém algum vetor estritamente positivo.
\end{pressup}




Com efeito,   o Pressuposto \ref{ass:interior-nonempty} tem como
consequência o fato existir  solução para
\eqref{eq:primal} e para \eqref{eq:dual}~\cite[Teorema 3.1]{Guler:1995wf}.
Mais que isso, a partir desse pressuposto, conhecido como \emph{pressuposto do
ponto interior}, concluímos que o conjunto ótimo primal-dual  é
limitado~\cite[Lema 2.2]{Guler:1995tn}.  Caso
esse pressuposto não seja satisfeito, considera-se permitir que o método aceite
iterados infactíveis ou introduz-se perturbações que aumentem o conjunto
$\Fset$.

  


Apresenta-se agora  alguns resultados bem conhecidos sobre a relação entre o par
primal-dual. Para maiores detalhes consulte-se
\textcite{Colombo:2008wm,Wright:Primal-dual-interior-point:1997h,Bazaraa:2009uu,Fang:1993wu},
textos que inspiraram esta apresentação.

\begin{lema}[Dualidade Fraca]\label{lema:weak-duality}
Seja $(x,y,z)\in\Fset$. Então $c^Tx \geq b^Ty$.
\end{lema}

\begin{proof}
Como $x\in\Pset$ e $(y,z)\in \Dset$ então vale
\[
c^Tx -  b^Ty = c^Tx - x^TA^Ty = x^T(c - A^ty) = x^Tz \geq 0.
\]
\end{proof}  
O significado da Dualidade Fraca é que o valor objetivo primal serve de
limitante superior para o valor objetivo dual e vice-versa. A diferença $c^Tx - 
b^Ty $ é chamada \emph{gap de dualidade}.
Quando em um ponto $(x,y,z)$ factível, o gap de dualidade é zero, isto é, quando 
tanto o valor objetivo primal quanto dual alcançam seus limites, então este ponto  é
solução ótima primal-dual. Este resultado é formalizado no lema a seguir.


 \begin{lema}[Dualidade Forte]\label{lema:strong-duality}
Um ponto $x\in\Pset$ é ótimo se e somente se existe um par $(y,z)\in\Dset$ tal
que $c^Tx = b^Ty$.
\end{lema}

Uma condição necessária e suficiente para que o problema \eqref{eq:primal} tenha uma solução
factível é que $\Pset \neq \varnothing$.  Além disso, se $\Pset \neq \varnothing$
e $\Dset\neq \varnothing$ então ambos \eqref{eq:primal} e \eqref{eq:dual} admitem solução ótima
 $(x^*,y^*,z^*)$ e pelo Lema \ref{lema:strong-duality} o valor da função
 objetivo coincide neste ponto. Por outro lado, se um dos conjuntos $\Pset$ ou
 $\Dset$ forem vazios, então o outro conjunto será ilimitado ou vazio. Nestes
casos, os problemas (\ref{eq:primal}-\ref{eq:dual}) não terão solução.

Podemos agora expressar condições de otimalidade para \eqref{eq:primal} e \eqref{eq:dual}. Tais
condições  ajudarão a reconhecer quando estes problemas  possuem solução
ótima e por isso podem nos ajudar a desenvolver algoritmos ou métodos para
encontrar tais soluções. As condições de 
\acrodef{KKT}{Karush-Kuhn-Tucker}
\ac{KKT}  expressam uma condição de otimalidade de primeira ordem para um \ac{PL}
e podem ser escritas como 
\begin{subnumcases}{\label{eq:KKT}}
Ax=b,\label{eq:KKT-fac-primal}\\ 
A^Ty + z =c, \label{eq:KKT-fac-dual}\\
XZe =0,  \label{eq:KKT-complementar}\\
(x,z)\geq 0. \label{eq:KKT-nao-negativ} 
\end{subnumcases}
em que $X = \diag(x)$, $Z = \diag(z)$ e $e = (1,\ldots,1)^T$.

 As duas primeiras
equações de \eqref{eq:KKT} são conhecidas como \emph{factibilidade ou
viabilidade primal e dual} -- com exceção feita à não-negatividade  --,
respectivamente.
A equação \eqref{eq:KKT-complementar}
significa que $x_iz_i = 0$, para todo $i=1,2,\ldots,n$, e é chamada de \emph{complementaridade}. A última equação é chamada
de \emph{não-negatividade}. Note ainda que se $(x,y,z)\in\Fset$, então ele
satisfaz \eqref{eq:KKT-fac-primal}, \eqref{eq:KKT-fac-dual} e \eqref{eq:KKT-nao-negativ}.
Isso significa que uma solução ótima é caracterizada pela factibilidade primal e
dual e pela complementaridade. Para pontos não ótimos, mas factíveis, a
complementaridade pode nos dar uma medida da \emph{distância} destes pontos para a
otimalidade:
\begin{equation}
\label{eq:gap-complementarity}
x^Tz = c^Tx - b^Ty.
\end{equation}
A valor $x^Tz$ é chamado \emph{gap de complementaridade}. Quando este 
converge para zero, então tem-se uma solução ótima. Além disso, a igualdade
entre o gap de complementaridade e o gap de dualidade mostrado em
\eqref{eq:gap-complementarity} só vale quando o ponto é factível. 


\begin{obs}
 Para fins  de
notação, dados vetores $u$ e $v$ em $\Real^n$,  o vetor $UVe$ poderá
ser representado usando-se o produto de Hadamard~\cite[p.~455]{Horn:1985tf},
isto é, $uv$ é o vetor em $\Real^n$, tal que cada componente  $(uv)_i  = u_iv_i$, para $i=1,\ldots,n$.
Além disso, a substituição $\xi e$ por $\xi$ pode ser feita, quando  $\xi$ for
um escalar e $e = (1,\ldots,1)$, sempre respeitando e adequando as dimensões.
\end{obs}




As equações \eqref{eq:KKT} dão uma condição necessária e suficiente para que
$w^* = (x^*,y^*,z^*)$ seja solução de (\ref{eq:primal}-\ref{eq:dual}), isto é, $x^*$ é
solução de \eqref{eq:primal} e $(y^*,z^*)$ é solução de \eqref{eq:dual}. Consequentemente, um
corolário de \ac{KKT} para \ac{PL} é  o Lema \ref{lema:strong-duality}. 


\subsection{Método Afim-escala\label{sec:affine-scalling}}

Os métodos de pontos interiores primais-duais para \ac{PL}  encontram
$w^*$ resolvendo as equações \eqref{eq:KKT}. Note que a única equação não-linear de
\eqref{eq:KKT} é $XZe=0$ e por conta dela é que aplica-se alguma  variante do
método de Newton, modificando a direção de busca e o tamanho de passo de modo que
\eqref{eq:KKT-nao-negativ} seja  satisfeita \emph{estritamente} em toda iteração. 


Podemos reescrever \eqref{eq:KKT} utilizando uma aplicação
$F:\Real^{2n+m}\to \Real^{2n+m}$ da seguinte maneira:
\begin{subequations}
\label{eq:KKT-Newton}
\begin{gather}
F(w) = \bbm Ax - b \\ A^Ty + z - c \\ xz \ebm = 0\label{eq:KKT-Newton-F},\\
(x,z)\geq 0. \label{eq:KKT-Newton-NN}
\end{gather}
\end{subequations}
Todos os métodos primais-duais geram iterados $w^k = (x^k, y^k,z^k)$ que satisfazem
\eqref{eq:KKT-Newton-NN} estritamente, isto é, $(x^k, z^k) > 0$. Esta
propriedade dá origem ao termo \emph{ponto interior}. Ao respeitar estes
limites, impede-se iterados tais que $F(w^k)=0$, mas que  não satisfaçam
$(x,z)\geq 0$. Estes tipos de soluções são facilmente encontrados, no entanto
não nos dão informações úteis para a solução de \eqref{eq:primal} e
\eqref{eq:dual}. Alguns métodos primais-duais exigem que $w^k\in\Fset^0$, ou
seja, que o ponto seja estritamente factível. No entanto, atualmente os métodos
mais competitivos trabalham com pontos infactíveis, exigindo somente que o ponto
seja estritamente interior (veja Seção~\ref{sec:infeasible_inicial_point})

O método de Newton aplicado a \eqref{eq:KKT-Newton-F} dá uma aproximação linear
em torno do ponto atual, obtendo uma direção de busca $\Dew = (\Dex, \Dey,
\Dez)$ resolvendo o seguinte sistema linear

\[
\nabla F (w)\Dew = -F(w)
\]
em que 
\[\nabla F = \bbm A & 0 & 0 \\ 
0 & A^T & I \\
Z& 0 & X
\ebm\] 
é a matriz Jacobiana de $F$. 

No caso de problemas de \ac{QP} como dado em
\eqref{eq:QP}, a diferença  estaria justamente na matriz Jacobiana, que
seria  dada por 
\[\nabla F = \bbm A & 0 & 0 \\ 
-Q & A^T & I \\
Z& 0 & X
\ebm.\] 

Note que o passo Newton é encontrado 
ao resolver
\begin{equation}
\label{eq:KKT-Newton-step}
\bbm A & 0 & 0 \\ 
0 & A^T & I \\
Z& 0 & X
\ebm
\bbm\Delta x \\ \Delta y  \\
\Delta z\ebm
=
\bbm b - Ax \\
 c- A^Ty - z \\
  -xz
\ebm = \bbm r_P \\ r_D \\ r_C\ebm .
\end{equation}
Se o ponto atual é estritamente
factível, então os resíduos $r_P$ e $r_D$ são nulos.  
 
Um passo na direção $\Dew = (\Dex,\Dey,\Dez)$ é chamado \emph{afim-escala}. Normalmente não é
possível caminhar nesta direção com passo completo, isto é, $w + \Dew$
geralmente viola o limite \eqref{eq:KKT-Newton-NN}. Para evitar essa dificuldade
pode ser feita uma busca linear na direção de Newton tal que o novo iterado
será \[ w + \alpha\Dew \] em que $\alpha \in (0,1]$ é o parâmetro da busca
linear. Mesmo assim, geralmente podemos apenas dar passos pequenos nesta direção
($\alpha \ll 1$) antes de violar a condição $(x,z) > 0$. Consequentemente passos
puramente afim-escala não nos permitem um bom progresso em direção à solução.

%\section{A trajetória central}

 De modo a poder contornar os problemas do método afim-escala, os métodos
 primais-duais sugerem algumas estratégias diferentes do  procedimento padrão de
 Newton:
 \begin{enumerate}[(i)]
\item Enviesar a direção de busca de modo a fazê-la apontar para dentro do
interior do ortante não-negativo $(x,z)\geq 0$, de tal forma que seja possível
mover-se suficientemente ao longe dessa direção antes que algum componente de
$(x,z)$ se torne negativo.
\item Manter os componentes de $(x,z)$  longe do limite do ortante não negativo.
Direções de busca calculadas por meio de pontos próximos a este limite tendem a
ser calculadas com erros de arredondamento relativamente grandes e pouco
progresso pode ser feito por elas.
 \end{enumerate}
 
 
 \subsection{O método de barreira logarítmica}
\label{subsec:barrier-problem}


Uma maneira de fazer  as duas proposições acima valerem é usar  uma
função barreira logarítmica no \ac{PL}. Dado o \ac{PL} na forma padrão \eqref{eq:primal}, é
possível definir um \emph{problema de barreira} correspondente 
\begin{equation}
\begin{array}{cc}
\displaystyle \min_{x} & \displaystyle c^Tx - \tau\sum_{i=1}^n \ln x \\[6mm]
\text{s.a.} & x\in\Pset^0
\end{array}\tag{$P_\tau$},
\label{eq:primal-barrier}
\end{equation}
em que $\tau > 0$ é um escalar, normalmente pequeno, que serve de parâmetro para
uma família de problemas $(P_\tau)$, e que é é chamado de
\emph{parâmetro de barreira}.

A presença da barreira logarítmica na função objetivo de \eqref{eq:primal-barrier} força o
iterado a ficar no interior da região factível, já que há uma penalização muito
pesada quando os pontos estão perto do limite. Por outro lado a influência da
função barreira pode ser controlada através do parâmetro $\tau$. O peso na
barreira regula a distância do iterado para o limite, ou seja, quando $\tau \to
0$,  o problema \eqref{eq:primal-barrier} cada vez mais se parece com o problema \eqref{eq:primal}. Esta
estratégia só é possível se $\Pset^0 \neq \varnothing$. Além disso, se $\Pset$
for limitado então tanto \eqref{eq:primal} quanto \eqref{eq:primal-barrier} admitem solução ótima.

Como a função objetivo de \eqref{eq:primal-barrier} é estritamente convexa, o minimizador desta
função, se existir, pode ser completamente caracterizado pelas condições
\ac{KKT}:

\[
\begin{cases}
Ax  =b,\\ 
\tau X^{-1}e + A^Ty =c,\\
(x,z) > 0. 
\end{cases}
\]
Fazendo $Ze=\tau X^{-1}e$, ou equivalentemente $xz = \tau e$, obtemos a
formulação padrão primal-dual das chamadas \emph{condições \ac{KKT} perturbadas}:

\begin{subnumcases}{\label{eq:KKT-perturbed}}
Ax  =b,\label{eq:KKT-fac-primal-perturbed}\\ 
A^Ty + z =c, \label{eq:KKT-fac-dual-perturbed}\\
xz =\tau e,  \label{eq:KKT-complementar-perturbed}\\
(x,z)  >0. \label{eq:KKT-nao-negativ-perturbed} 
\end{subnumcases}

Se as condições \ac{KKT} perturbadas admitem solução para algum $\hat{\tau} >0$,
então admitem solução para qualquer $\tau>0$. O sistema \eqref{eq:KKT-perturbed}
se aproxima mais e mais de \eqref{eq:KKT} quando $\tau\to 0$ e determina uma
única curva suave, contínua e parametrizada pela variável $\tau$, definida por
$\Cset = \{(x(\tau),y(\tau),z(\tau)):
\tau > 0 \}$. Para cada $\tau$ o ponto da curva é completamente caracterizado
como sendo a solução única do sistema \eqref{eq:KKT-perturbed}.
Além disso, quando  $\tau\to 0$, $\Cset$ converge  para uma solução ótima
prima-dual do \ac{PL}. Essa curva é chamada em \ac{MPI} de \emph{trajetória
central}.  Esta trajetória nos guia para uma solução em que os pares $x_iz_i$
são estritamente positivos e decrescem a zero na mesma taxa de
decrescimento de $\tau$. Estudos sobre a trajetória central pode ser encontrados
em \textcite{Bayer:1989av,Bayer:1989ud,Sonnevend:1986ua,Meggido:Pathways-to-the-optimal:1988u}, entre outros. 

Usando o fato de que para algum $\tau>0$ o ponto $(x(\tau),y(\tau),z(\tau))$ é
primal-dual factível, podemos definir o gap de dualidade $g(\tau)$ para \eqref{eq:primal-barrier}
de forma semelhante à Equação \eqref{eq:gap-complementarity} como função do
parâmetro de barreira:

\begin{equation}
\label{eq:gap-duality-mu}
g(\tau) = c^Tx(\tau) - b^Ty(\tau) =
x(\tau)^T(\tau),
\end{equation}
isto é, para cada valor de $\tau$, o gap de dualidade corresponde ao gap de
complementaridade e logo reduzir um significa reduzir o outro. Além disso, como
$XZe - \tau e = 0$ por \eqref{eq:KKT-complementar-perturbed}, então $x_iz_i =
\tau, i=1,\dotsc,n$ e logo
 \begin{equation}
\label{eq:gap-duality-n-mu}
g(\tau) = x(\tau)^Tz(\tau)= \sum_{i=1}^nx_i(\tau)z_i(\tau) = n\tau.
\end{equation}
Isso significa que quando $\tau\to 0$ temos $g(\tau) \to 0$. As equações
\eqref{eq:gap-duality-mu} e \eqref{eq:gap-duality-n-mu} em conjunto com o fato
de que $c^T x (\tau) \geq c^Tx^* = b^T y^* \geq b^Ty(\tau)$, implicam que

 \[ c^Tx(\tau) \xrightarrow{\tau \to 0} c^Tx^* \quad \text{ e } \quad b^Ty(\tau)
 \xrightarrow{\tau \to 0} b^Ty^*.
\] Assim, os valores objetivos do problema perturbado convergem para aqueles
obtidos por uma solução ótima $(x^*,y^*,z^*)$ do problema original. Mais que isso, vale
o seguinte resultado~\cite{Meggido:Pathways-to-the-optimal:1988u}:
 
 
 \begin{teo}\label{teo:xmu-to-xstar}
 Se um problema de \ac{PL} for primal e dual factível e 
 $A$ for de posto completo, então
  \[
 x(\tau) \to x^* \quad \text{ e } \quad 
 (y(\tau),z(\tau))  \to (y^*,z^*),
\]
sempre que   $\tau\to0$.
 \end{teo}
 
 Este teorema significa que, sob algumas condições, a trajetória central
 converge para a solução ótima dos problemas \eqref{eq:primal} e
 \eqref{eq:dual}. Em consequência disto, a trajetória central  pode ser um bom
 guia para encontrar o conjunto ótimo primal-dual. Métodos que se baseiam na
 trajetória central são chamados \acl{MPI}
\emph{seguidores de caminho}.
 


A solução que se encontra por seguir a trajetória central é caracterizada pela
\emph{complementaridade estrita}. Isso é descrito pelo seguinte resultado.
\begin{teo}[Complementaridade estrita]\label{thm:strict_complementarity}
Se \eqref{eq:primal} e \eqref{eq:dual} forem factíveis, então existe um ponto
$x^*\in\Pset$ e um par $(y^*,z^*)\in\Dset$ tais que 
\[
(x^*)^Tz^* = 0 \text{\quad e \quad} x^*_i+ z^*_i >0, \text{ para }i=1,\ldots,n.
\]
\end{teo}

 

Uma solução $(x^*,z^*)$ que satisfaça o teorema acima é chamada estritamente
complementar. Nesses termos pode-se definir o conceito de partição ótima.
Seguindo as ideias de \textcite{Jansen:1997vy}, definimos o suporte do vetor
$v\in\Real^n$ como
\[
\supp(v) = \{i:v_i>0, i=1,\ldots,n\}.
\]
e a partição do conjunto de índices $\{1,\ldots,n\}$ através da definição dos
conjuntos 
\[
\Bset = \supp(x^*) \text{\quad e \quad } \Mset = \supp(z^*).
\]
Do Teorema \ref{thm:strict_complementarity} segue que essa é uma partição bem
definida, no sentido de que as soluções que possuem a propriedade da
complementaridade estrita satisfazem tanto $\Bset\cap \Mset = \varnothing$
quanto $\Bset\cup\Mset = {1,\ldots,n}$. A noção de complementaridade estrita e
de partição ótima são ideias recorrentes em análises de \ac{MPI}.

Um  fato que vale a pena chamar atenção é que nos casos em que o \ac{PL}
tenha múltiplas soluções, um \ac{MPI} para em uma vizinhança do 
centro analítico da face ótima ao invés de em um vértice, como no método
simplex. Isso significa que de certa maneira, através do conceito de partição
ótima,  pode-se interpretar essa situação como sendo a determinação de
todo o conjunto de soluções ótimas. Em contraste a isso, a escolha do vértice ótimo obtido pelo
método simplex é arbitrária e depende de alguns fatores como regras de
pivoteamento. 

Não raramente, ter uma solução básica que identifica um vértice é equivalente a
encontrar uma solução \emph{exata}. Entretanto, deve-se discutir o que significa de
solução ``exata''. Em vários casos, não é necessária uma precisão adicional de
ter-se uma solução em um vértice, ao invés da solução que se encontra no centro
analítico da face ótima.  Do ponto de vista da programação inteira, temos uma
exceção a esse caso, já que as solução inteiras encontram-se nos vértices do
envoltório convexo de pontos inteiros factíveis. A diferença entre ter ou não
uma base ótima ou uma partição ótima é de importante consequência para o uso de
soluções na análise de sensibilidade \cite{Jansen:1997vy,Yildirim:2001gp}


\textcite{Vavasis:1996bw}
discutiram propriedades da trajetória central, demonstrando que a mesma é caracterizada por
$\mathcal{O}(n^2)$ curvas de alto grau e segmentos nos quais a trajetória é
relativamente reta. Mais que isso, em uma vizinhança
suficientemente pequena do ótimo a trajetória central se torna uma linha reta
e com isto, nesta região, o método apresenta a boa propriedade da convergência quadrática que os métodos do tipo
Newton possuem~\cite{Meggido:Pathways-to-the-optimal:1988u}.
 


Considere o limite de \eqref{eq:primal-barrier} quando $\tau\to \infty$ e
consequentemente encontra-se o ponto do qual a trajetória central inicia. Isso
corresponde a encontrar o ponto $\check{x}$ que minimiza a função de barreira,
isto é, 
\[
\check{x} = \arg \min_{x\in\Pset^+}\left\{-\sum_{i=1}^{n}\ln x_i \right\}.
\]
O ponto $\check{x}$ é o \emph{centro analítico} do politopo factível, e foi
primeiramente estudado por \textcite{Sonnevend:1986ua}. Dada a convexidade estrita
da função de barreira, o conceito de centro analítico está bem definido. Como o
centro analítico minimiza o problema de barreira, ele é o ponto que se encontra
mais distante da fronteira do politopo. Entretanto, existe um problema em
definir a trajetória central em termos de centro analítico: a trajetória central
é afetada pela presença de restrições redundantes. Isso acontece porque tem-se
aqui um conceito exclusivamente analítico, o qual não explora considerações
geométricas. Para vencer essa desvantagem, outros tipos de centro -- centro de
gravidade, centro do elipsoide de volume máximo que pode ser inscrito em
$\Pset$, centro volumétrico  -- podem ser definidos, mas usualmente são muito
custosos de se calcular \cite{Gonzaga:1992uj}.


Tal desvantagem da trajetória central foi  apresentada como potencialmente
geradora de consequências extremas por \textcite{Deza:2006hm}, que conseguiram
replicar o comportamento do método simplex no cubo de Klee-Minty num contexto de
pontos interiores. Isso foi conseguido por patologicamente se adicionar um
número exponencial de restrições redundantes e paralelas às faces do cubo, tal que a trajetória
central torna-se altamente destorcida e está presente em vizinhanças
arbitrariamente pequenas de todos os vértices do cubo. 

Menciona-se aqui o fato de que técnicas de pré-processamento são usualmente
implementadas em códigos maduros de modo a remover tanto quanto possível essas
restrições redundantes, mas embora elas sejam implementadas com heurísticas de
sucesso, geralmente não são ótimas.







 
\section{Métodos seguidores de caminho\label{sec:path-following-methods}}



\subsection{Modelo geral para métodos seguidores de caminho}
 A maioria dos métodos primais-duais seguidores de caminho dão passos de Newton
 em direção a pontos de $\Cset$ para os quais $\tau>0$, ao invés de passos
 puramente Newton para $F$. Como vimos,  o sistema \eqref{eq:KKT} é resolvido
 por fazer os pares complementares se alinhem (veja Equação
 \eqref{eq:KKT-complementar-perturbed}) ao mesmo tempo em que $(x,z) >0$. Como
 estes passos são enviesados em direção ao ortante positivo definido por
 $(x,z)>0$, usualmente é possível tomar passos maiores que um passo de Newton
 para $F$ antes de violar a restrição de positividade. Para descrever esta busca
 enviesada, introduzimos um \emph{parâmetro de centralização} $\eta\in[0,1]$, na
 equação \eqref{eq:KKT-complementar-perturbed}, que faz com que em cada iteração
 $\tau$ decresça monotonicamente.

 
 As equações para  encontrar a direção de busca tornam-se
  \begin{equation}
\label{eq:KKT-Newton-step-mu-tau}
\bbm A & 0 & 0 \\  
0 & A^T & I \\
Z& 0 & X
\ebm
\bbm\Delta x \\ \Delta y  \\
\Delta z\ebm
=
\bbm r_P \\
 r_D \\
 r_C 
  + \eta \tau e
\ebm = 
\bbm r_P \\ r_D \\ r_{\tau}\ebm,
\end{equation} 
 em que  $\Dew = (\Delta x, \Delta y, \Delta z)$ é um passo de Newton apontando para
$(x(\eta\tau),y(\eta\tau),z(\eta\tau))\in\Cset$ e o par $x_iz_i$ é
igual à $\eta\tau$.  Mesmo nesse caso, a direção de busca encontrada $\Dew$ nem
sempre pode ser utilizada e o novo iterado é calculado novamente usando um teste da razão, isto é, escolhendo $\al_k$, tal
que
\begin{equation}
\label{eq:updating-iterate}
 w^{k+1} = w^k+ \al_k \Dew
\end{equation}
e $(x^{k+1}, z^{k+1})>0$. 

\minisec{O parâmetro $\eta$}
 Por um lado, se $\eta=1$ então
\eqref{eq:KKT-Newton-step-mu-tau} define a \emph{direção de centragem} na qual o
passo de Newton é dado em direção à trajetória central. Embora esta direção se
afaste dos limites do ortante não-negativo, ela pouco ou nada reduz o valor de
$\tau$. Entretanto, ao se mover mais perto de $\Cset$  estes passos formam
uma boa base para um progresso substancial na próxima iteração. Isto porque como
a próxima iteração está mais perto da trajetória central, será possível dar um
passo relativamente grande sem que se deixe o ortante não-negativo. Por outro
lado, se $\eta =0$ temos a direção afim-escala pura como em
\eqref{eq:KKT-Newton-step}. A maioria dos métodos usa valores intermediários de
$\eta$ no intervalo aberto $(0,1)$, de modo a tentar alcançar de maneira
eficiente o duplo objetivo de reduzir $\tau$ e melhorar a centralidade. Esta
escolha é dependente do método.



Vamos definir agora uma maneira de escolher $\tau$ utilizando a equação
\eqref{eq:gap-duality-n-mu}. Considere  dado  um ponto estritamente
factível, isto é, $w^0\in \Fset^+$. Então o valor do parâmetro de
barreira  será dado por 
\begin{equation}
\label{eq:tau_k}
\tau_k = \frac{(x^k)^Tz^k}{n}. 
\end{equation}
Olhando por este ângulo, $\tau_k$ representa a média dos valores dos produtos
$x_i^k z_i^k$. Além disso, com o avanço das iterações, o parâmetro de
centralização $\eta$ faz com  que o problema perturbado que estamos resolvendo
\eqref{eq:KKT-Newton-step-mu-tau} se aproxime cada vez mais do problema original \eqref{eq:KKT-Newton}.

 

Com estes conceitos e ideias em  mão, podemos definir um modelo geral para
métodos primais-duais, dado no Pseudo-Código~\ref{alg:modelo-geral}.
\begin{algorithm}
\caption{Modelo geral para um método primal-dual seguidor de caminho.}
\label{alg:modelo-geral}
\begin{algorithmic}[0] 
\Require  $w^0\in\Fset^+$ 
\State $k\gets 0$
	\Repeat
		\State Resolva \eqref{eq:KKT-Newton-step-mu-tau}  para algum $\eta\in(0,1)$.
		\State Encontre $\al_k$, tamanho de passo factível máximo  na direção $\Dew^k$.
		\State Atualize o iterado conforme \eqref{eq:updating-iterate}.
		\State $k\gets k+1$
\Until{O critério de parada ser satisfeito.}
\end{algorithmic}
\end{algorithm}



\subsection{Vizinhanças da Trajetória Central}
\label{subsec:neighbouhoods}

Um método seguidor de caminho segue  a trajetória central $\Cset$ no interior
da região factível em direção a uma solução ótima. No entanto manter um
iterado exatamente sobre $\Cset$ é um objetivo muito difícil, senão impossível.
Isto porque encontrar um ponto que resolve a condição de complementaridade
perturbada \eqref{eq:KKT-complementar-perturbed} para um $\tau$ específico é um
problema tão difícil quanto resolver o próprio \ac{PL}.


Assim, já que computacionalmente é pouco eficiente caminhar ao longo de uma
direção não-linear, um método seguidor de caminho faz com que os iterados
fiquem em torno de $\Cset$. Isto significa restringir os iterados a uma
\emph{vizinhança} da trajetória central e durante o progresso das iterações
segui-la até uma solução do \ac{PL}. Podemos definir vários tipos de
vizinhanças, no entanto sempre excluimos os pontos $(x,z)$ que estão muito perto
do limite do ortante não-negativo. Com isto  direções de busca calculadas de
qualquer ponto desta vizinhança progridem, mesmo que minimamente, em direção ao
conjunto solução e ao mesmo tempo são obtidas de maneira mais fácil.

As duas vizinhanças de $\Cset$ mais comuns são  $\Nset_2(\theta)$, que tem como base a norma-2 e é definida por
\[
\Nset_2(\theta) = \left\{w\in\Fset^0 : \norm{XZe - \tau e}_2 \leq \theta\tau  
\right\},
\]
em que $\theta \in (0,1)$, e a vizinhança 
$\Nset_{-\infty}(\ga)$, baseada na norma-$\infty$, dada por 
\begin{equation}
\label{eq:infty-neig}
\Nset_{-\infty}(\ga) = \left\{w\in\Fset^0 : x_iz_i\geq \ga\tau\text{, para
todo }i =1,\dotsc,n \right\},
\end{equation}
para algum $\ga\in(0,1)$\footnote{Tipicamente são usados os seguintes valores 
para os parâmetros: $\theta = 0.5$ e $\ga = 10^{-3}$~\cite[pg.
9]{Wright:Primal-dual-interior-point:1997h}.}.

A vizinhança $\Nset_2(\theta)$ é mais restrita já que certos pontos de $\Fset^+$
não pertencem à $\Nset_2(\theta)$, não importando quão perto façamos $\theta$ se
aproximar de seu limitante superior 1. Por outro lado, se um ponto pertence à
$\Nset_{-\infty}(\ga)$, então cada par $x_iz_i$ deve ser pelo menos um múltiplo
$\ga$ de seus valores médios $\tau$. Com efeito, se escolhermos $\ga$ perto de
zero, então quase toda região $\Fset$ estará contida em $\Nset_{-\infty}(\ga)$. 

Mantendo todos os iterados dentro de uma dessas duas vizinhanças, métodos
seguidores de caminho reduzem  os produtos $x_iz_i$ a zero mais ou menos na
mesma taxa.


\ac{MPI} seguidores de caminho são  métodos de homotopia
continuada~\cite{Nazareth:1986jg}
similares aos métodos de homotopia para equações não-lineares gerais. Estes
definem uma trajetória que
deve ser seguida para encontrar a solução. Tradicionalmente, métodos de
homotopia mantem-se em uma vizinhança tubular da trajetória, realizando mudanças
incrementais no parâmetro e perseguindo a trajetória homotópica em busca da
solução. Para \ac{MPI} primais-duais, estas vizinhança são cônicas, ao invés de
tubulares, e tendem a ser amplas e relaxadas  para valores grandes da medida de
dualidade $\tau$, porém se tornam mais estreitas quando $\tau\to0$, por conta da
positividade exigida de $(x,z)$.


 Direções de busca que se baseiam na vizinhança $\Nset_2(\theta)$ podem ser
 utilizadas com $\al=1$, e o parâmetro de barreira decresce pouco em cada
 iteração, dando lugar aos chamados \emph{métodos de passo-curto}. Esta
 propriedade impõe e ao mesmo tempo produz os melhores resultados de
 complexidade existentes para \ac{PL}: algoritmos de passo curto tem
 complexidade $\Oset(\sqrt{n}\ln (1/\eps))$. Os métodos de passo curto
 são os mais simples dos \ac{MPI} pois fixam $\al_k\equiv  1$ e $\eta_k \equiv \eta$
 dependente de $\theta$ e foram introduzidos por
 \textcite{Kojima:1989fw} e \textcite{Monteiro:1989go}. Como a redução do
 parâmetro de barreira é muito pequena, estes tipos de métodos são pouco usados
 na prática.
   
 
 Métodos baseados na vizinhança $\Nset_{-\infty}(\ga)$ permitem iterados que
 seguem a trajetória central de maneira mais relaxada  pois obtém-se mais
 liberdade para manobrar e é possível se aproximar da fronteira da região
 factível. De fato, para $\ga$ pequeno quase todo o conjunto $\Fset^0$ está
 contido em $\Nset_{-\infty}(\ga)$. Por outro lado, direções calculadas a partir
 de pontos desta vizinhança têm propriedades mais fracas e o teste da razão é
 necessário para assegurar a positividade de $(x,z)$. Portanto, métodos que
 levam em conta a vizinhança larga são menos conservadores que os métodos de
 passo-curto e podem decrescer o parâmetro de barreira mais rapidamente. São
 chamados de \emph{métodos de passo-longo}  e implementações eficientes de
 \ac{MPI} baseiam-se em alguma variação de métodos de passo-longo.
 
 \subsubsection{Vizinhança Simétrica}
\label{sec:symmetric-neig}  
 A vizinhança $\Nset_{-\infty}(\ga)$, como vimos,  produz uma boa base para
 métodos práticos, pois permite o parâmetro de barreira reduzir rapidamente.
 Mesmo assim, permite que os iterados produzam produtos $x_iz_i$ muito
 diferentes, já que não impõe uma limite superior para a complementaridade.
 
 Essa dificuldade, quando ultrapassada, permite resultados muito melhores na
 prática, já que os pares $x_iz_i$, quando mal escalados, influenciam o mau
 comportamento do método de Newton.
 
 A razão 
\begin{equation}
\label{eq:ratio-complementarity}
\varrho(xz) = \frac{\min\{xz\}}{\max\{xz\}}
\end{equation}
dá uma medida de proporção entre o maior e o menor par de complementaridade.
 Claramente, $\varrho(xz)\in(0,1)$ e quando o iterado é perfeitamente
centrado, este valor é igual a 1. Uma
análise da medida \eqref{eq:ratio-complementarity} pode ser encontrada em
\textcite{Jansen:1jv}.
 

\textcite{Gondzio:1996uw} e posteriormente \textcite{Colombo:2008wm,Colombo:2008ia}
definiram uma vizinhança que tenta fazer com que em cada iteração tenha-se uma
certa \emph{centralidade} dos iterados.  Por centralidade entenda-se a
dispersão dos pares $x_iz_i$. Quando a discrepância nos
valores dos pares complementares é muito grande, e portanto têm-se pouca
centralidade, as direções de busca não são adequadas, não só com valores
pequenos mas também com valores grandes de $x_iz_i$. Com efeito, \textcite[pp.
26]{Colombo:2008wm} propõe que a noção de dispersão dos produtos
complementares não é adequadamente  representada em um ambiente computacional
por nenhuma das  vizinhanças  $\Nset_{-\infty}$.



Com o propósito de corrigir estes problemas foi proposta a \emph{vizinhança
simétrica} $\Nset_s$, na qual os pares complementares não só satisfazem um limite inferior
mas também um limite superior e é definida como
\begin{equation}
\label{eq:symmetric-neig}
\Nset_s(\ga) = \left\{w\in\Fset^0 : \ga\tau \leq x_iz_i \leq
\frac{1}{\ga}\tau, i=1,\dotsc,n \right\},
\end{equation}
em que $\tau = x^Tz/n$ e $\ga\in(0,1)$.

A vizinhança simétrica \eqref{eq:symmetric-neig}  pode ser considerada uma
extensão de $\Nset_{-\infty}$. No entanto, ela não deixa que os
produtos complementares se tornem muito grandes em relação à média. Além disso, $\Nset_s$ promove uma diminuição dos
pares complementares que são muito grandes, permitindo uma melhor centralidade. 


\textcite{Colombo:2008wm}
determinou o valor de $n$ para o qual a vizinhança simétrica impõe um limite
superior mais estreito:
\[
n> \frac{1+\ga}{\ga}
\]
Para um $\ga=0.1$, o limite é estreito sempre que $n>11$. 

Além disso, a complexidade do pior caso  para um método de passo-longo factível
que utiliza a vizinhança simétrica $\Nset_s$ é $\Oset(n\ln(1/\eps))$~\cite{Colombo:2008ia}.
Isso significa que  os limites superiores que diferenciam a vizinhança simétrica
de $\Nset_{-\infty}$ não produzem perdas teóricas, ao mesmo tempo em que,
contribuem com  a possibilidade de deixar os iterados mais centrados.
  
  
De fato, \textcite{Zhang:1993gn}, ao provarem a convergência superlinear de um método
primal-dual, utilizam uma vizinhança com parâmetros simétricos para melhorar os
limites da convergência. No entanto, à época imaginavam que um limite superior,
como o dado por $\gamma\tau$,  não teria significância prática, previsão não se
concretizou~\cite{Gondzio:1996uw,Colombo:2008wm,Colombo:2008ia}.

\subsection{Ponto inicial infactível}
\label{sec:infeasible_inicial_point}

Note que até agora, estamos assumindo que o método inicia  um ponto estritamente
factível $w^0 \in \Fset^0$. Neste caso, $r_P = r_D = 0$, no lado
direito de \eqref{eq:KKT-Newton-step-mu-tau}, e assim a direção de busca
encontrada através desta equação garante que
\begin{equation}
\label{eq:fact-direction}
A\Dex = 0 \quad \text{ e } \quad A^T\Dey + \Dez = 0.
\end{equation}
Consequentemente a factibilidade de todos os iterados é garantida, pois para
$\al =1$ valem as seguintes igualdades:
\begin{gather*}
A(x + \Dex) = Ax + A\Dex = b, \\
A^T(y+\Dey) + (z + \Dez) = (A^Ty + z) + (A^T\Dey + \Dez) = c, \text{ e }\\
\Dex^T \Dez  = - \Dex^T (A^T\Dey) = - (A\Dex)^T\Dey = 0.
\end{gather*}
Métodos que assumem esta hipótese são os chamados \emph{\ac{MPI} factíveis}. 

Ao avaliar o gap de complementaridade quando tomando um passo de
tamanho $\al$ na direção $(\Dex, \Dey, \Dez) $ obtemos
\[
x(\al)^Tz(\al) = x^Tz + \al\left(z^T\Dex +  x^T\Dez + \al^2\Dex^T \Dez  
\right) = \left(1 - \al(1 - \eta) \right)x^Tz,
\]
em que usamos o fato de que $\Dex^T \Dez = 0$, $x^Tz = n\tau$ e que $z^T\Dex
+ x^T\Dez = - x^Tz + n\tau\eta $. Assim, dividindo por $n$ obtemos 
\begin{equation}
\label{eq:progress-opt-feasible}
\tau(\al) = x(\al)^Tz(\al)/n = (1 - \al(1-\eta))\tau.
\end{equation}

A equação \eqref{eq:progress-opt-feasible} mostra que  o progresso na
otimização depende dos parâmetros $\tau$ e $\eta$ bem como do tamanho do passo
$\al$. Este fato motiva a escolha adequada destes valores de modo a obter um
iterado que progrida em direção ao ótimo de maneira mais rápida. 




Nem sempre, porém,  obter um ponto inicial factível é uma tarefa trivial.
É possível, por exemplo, encontrar um ponto inicial factível reformulando o
problema  ou utilizando um método do tipo \emph{Big M}, mas estas
reformulações podem causar distorções, instabilidades numéricas ou aumento de
colunas densas.  Além disso, a região factível pode ter o interior vazio, o que
torna inválida a teoria apresentada até aqui. Uma abordagem totalmente diferente
é baseada na formulação
auto-dual~\cite{Ye:1994gq}, mas, nesse caso, para tornar o problema sempre
factível, aumenta-se o tamanho do problema, o que de certa forma aumenta o tempo computacional por exigir dois
\emph{backsolves} adicionais no cálculo de próximo iterado, devido à
introdução de colunas extra.

\subsubsection{Métodos Infactíveis}
Uma alternativa que  permite que essas dificuldades sejam contornadas é utilizar
\emph{\ac{MPI} infactíveis}. Métodos implementados que têm importância
prática fazem uso de pontos
infactíveis~\cite{Gondzio:1996uw,Gertz:2003ji,Czyzyk:1999hk}. Em geral, métodos
infactíveis exigem apenas que para o ponto inicial tenha-se $w^0\in\Qset^+$,
isto é, $(x^0,z^0)$ esteja no ortante positivo. Nesse caso  $r_P^0$ e $r_D^0$
são não-nulos.


A direção de busca ainda é feita segundo \eqref{eq:KKT-Newton-step-mu-tau}, e
portanto tem-se um passo de Newton em direção ao ponto
$(x({\eta\tau}),y({\eta\tau}),z({\eta\tau}))\in\Cset$. Pode-se provar
que se fosse possível dar um passo completo  $(\al=1)$,  a
infactibilidade seria eliminada. No entanto, na prática,  o limite da sequência
de iterados é o ponto ótimo e somente quando o ponto atual está em uma
vizinhança apropriada deste ótimo é que se dá um passo que zera os resíduos.

\textcite{Kojima:1993fe}
estabeleceram resultados de convergência para um \ac{MPI} infactível bem como uma regra para o
tamanho do passo que garante a convergência global do método. Isto é obtido ao
utilizar-se uma vizinhança similar à $\Nset_{-\infty}(\ga)$ (veja equação \eqref{eq:infty-neig}), dada por 
\begin{equation}
\label{eq:infeasible-infty-neig}
\Nset_{-\infty}(\ga,\be) = \left\{(x,y,z) \in\Qset^+ :\frac{\norm{(r_P,r_D)}}{\norm{(r^0_P,r^0_D)}} \leq
\be\frac{\tau}{\tau_0}, 
x_iz_i\geq \ga\tau,\forall i =1,\dotsc,n \right\}, 
\end{equation}
em que $\ga \in(0,1)$ e $\be\geq 1$ são
parâmetros, $\tau_0$ é dado por \eqref{eq:tau_k} e  $r^0_P$, $r^0_D$ são os
resíduos iniciais primais e duais respectivamente.

 
Embora $\Nset_{-\infty}(\ga,\be)$ não exija a factibilidade em cada iteração,
 existe um limitante superior para a infactibilidade,  que tem como
componente principal a medida de  complementaridade $\tau$.
Assim, ao reduzirmos $\tau$ forçamos os resíduos para zero,  reduzindo
a complementaridade e a infactibilidade na mesma taxa. Com efeito, se
$(x(\al),y(\al),z(\al)) = (x,y,z) + \al\De(x,y,z)$, então 
\[
r_P(\al) = (1-\al)r_P \text{ e } 
r_D(\al) = (1-\al)r_D
\]
mostrando que a infactibilidade se reduz linearmente com $\al$.  Assim, se $\al
= 1$, então os resíduos $r_P$ e $r_D$ tornam-se nulos, portanto a factibilidade é
restaurada e  subsequentemente o método infactível torna-se igual a um método
factível, já que todos os próximos iterados estarão em $\Fset^0$.
\textcite{Kojima:1993fe} provaram a convergência global de um método com passos
infactíveis, enquanto \textcite{Zhang:2006ic} provou  a complexidade  de $\Oset(
n^2 \ln ( 1/\eps ) )$ para um método desse tipo. Para uma uma explanação
completa de métodos infactíveis veja \textcite[cap. 6]{Wright:Primal-dual-interior-point:1997h}. 



 
 

 


\section{Métodos de Pontos Interiores na Prática}

\subsection{Método Preditor-Corretor de Mehrotra}
\label{sec:mehrotra-pc-method}
Do ponto de vista prático, a maioria dos códigos de implementação de \ac{MPI}
baseia-se em alguma variação de um método do tipo  preditor-corretor
 devido a \textcite{Mehrotra:1992wr}. O \acrodef{MPC-M}{Método Preditor Corretor de Mehrotra} \ac{MPC-M} baseia-se no modelo geral
para Métodos Seguidores de Caminho, dado no Pseudo-código
\ref{alg:modelo-geral}, mas altera a busca de direção puramente Newton com
correções que são muito baratas computacionalmente mas que auxiliam a obter uma
direção melhor. Além disso, um escolha adaptativa do parâmetro de centragem
$\eta$ é feita em cada iteração. Atualmente, nos códigos mais utilizados, tanto
acadêmicos como \texttt{BPMPD}, \texttt{HOPDM}, \texttt{OOPS}, \texttt{OOQP},
\texttt{PCx}, bem como  comerciais tais quais \texttt{Cplex}, \texttt{Mosek} e
\texttt{Xpress}, algumas heurísticas e ideias de otimização também foram
incorporadas, de modo que todos são variantes do \ac{MPC-M}.

Este método usa  aproximações de ordem maior da trajetória central $\Cset$ --
abordagem começada  por \textcite{Meggido:Pathways-to-the-optimal:1988u} e mais
tarde desenvolvidas por
\textcite{Monteiro:1990vn} -- e o uso de pontos infactíveis. Segundo \textcite[pg
194]{Wright:Primal-dual-interior-point:1997h},  a contribuição principal
de Mehrotra foi combinar estas ideias existentes da maneira certa e adicionar
heurísticas engenhosas para escolher o parâmetro de centragem (adaptativamente), o tamanho
do passo e o ponto inicial.

Até aqui o tamanho de passo foi sempre considerado igual tanto para o problema
primal, quanto para o problema dual. No entanto, a quase totalidade das
implementações do método de Mehrotra usa passos diferentes para a variáveis
primais e duais. Esta ideia não tem origem em Mehrotra, mas tem sido usada em
quase todos os desenvolvimentos práticos de métodos primais-duais e  contribui
para reduzir em até 10\% no número de iterações, tendo como base de testes o
conjunto  \texttt{Netlib}~\cite[pg.
195]{Wright:Primal-dual-interior-point:1997h}.

Como se quer que $(x,z)$ permaneça no ortante positivo, emprega-se uma busca
linear a fim de encontrar os tamanhos de passo primal $\al_P$ e dual $\al_D$  tais que
$x+\al_P\Dex >0$ e $z+\al_D\Dez >0$. Para isso, podemos fazer um teste da
razão, isto é,
\begin{equation}
\label{eq:size-step}
\al_P = \al_0\min \left\{- \frac{x_i}{\Dex_i} : \Dex_i <
0\right\}, \qquad \al_D = \al_0\min \left\{- \frac{z_i}{\Dez_i} : \Dez_i <
0\right\}
\end{equation} 
em que $\al_0$ é um fator que garante positividade estrita e em geral vale
$0,9995$. Além disso, embora use-se sempre  algum tipo de vizinhança, muitas
vezes permite-se que o ponto esteja fora da mesma e por isso, perde-se a
propriedade da convergência global em favor da eficiência computacional. 


O método de Mehrotra gera a sequência de iterados infactíveis
$(x,y,z)$, tais que $(x,z)>0$ e a direção de busca consiste em três componentes:
\begin{enumerate}[(i)]
  \item Uma  \emph{direção preditora}; 
  
  \item Um termo de centralização,
  cujo tamanho é controlado pela escolha de forma adaptativa do parâmetro $\eta$;
   \item Uma direção  \emph{corretora}.
\end{enumerate}




Para compreendermos melhor o que essas ideias significam, note que no sistema de
Newton \eqref{eq:KKT-Newton-step-mu-tau}, o lado direito pode ser separado da seguinte forma:
\[
\bbm b-Ax \\ c - A^Ty - z \\ -xz + \eta\tau \ebm = 
 \bbm b-Ax \\ c - A^Ty - z \\ -xz  \ebm +  \bbm 0 \\ 0 \\ 
 \eta\tau \ebm.
 \] 


Assim, a direção preditora $(\dex, \dey, \dez)$ é obtida resolvendo-se o
sistema de Newton com o lado direto dado por
 \begin{equation}
 \label{eq:afinne-scale-step}
 \bbm b-Ax\\ c - A^Ty - z \\ -xz  \ebm =  \bbm r_P\\ r_D \\ r_C  \ebm 
 \end{equation} 
 isto é,  $(\dex, \dey, \dez)$  corresponde a direção afim-escala dada em
\eqref{eq:KKT-Newton-step}. Como visto, essa direção otimiza fortemente em
direção aos pontos para os quais a complementaridade é nula. Entretanto, como o
alvo e $XZe = 0$, essas direções podem ser excessivamente influenciadas por
pontos que têm complementaridade pequena, mas que não são ótimos


%XXX: Distraídas? 


\subsubsection{Direção com correção de Segunda Ordem}

O \ac{MPC-M} explora uma direção corretora de centralidade de modo a
tentar remediar pontos que estão mal centrados. Esta direção caminha para
perto da trajetória central e logo reduz o espalhamento dos produtos
complementares em relação à média, embora não busque otimalidade.

Uma ferramenta importante para isso foi a avaliação dinâmica do parâmetro de
centralização $\eta$. Para começar, queremos medir a eficácia da direção
afim-escala, e por isso definimos $\tau_{\text{af}}$ como sendo a média dos valores
hipotéticos para os pares complementares resultantes de um passo na direção
afim-escala, dado por 
\begin{equation}
\label{eq:tau-affine}
 \tau_{\text{af}} = \frac{g_{\text{af}}}{n} = \frac{(x+\al_P^{\text{af}}\dex)^T
(z+\al_D^{\text{af}}\dez)}{n}.
\end{equation}
em que $g_{\text{af}}$ é o gap de complementaridade previsto para a direção afim-escala.
Note que $\tau_{\text{af}}/\tau\in(0,1)$. Se $\tau_{\text{af}}\ll\tau$ então a direção afim-escala é
boa e permite uma significante redução do gap de complementaridade. Por outro
lado, se a razão $\tau_{\text{af}}/\tau$ é próxima de 1, então pouco progresso está sendo
feito na direção  $(\dex, \dey, \dez)$ e um parâmetro $\eta$ perto de 1 é
recomendado. Essa escolha faz com que ocorra uma forte centralização, isto é,
move-se para mais perto de $\Cset$, e logo, na próxima iteração, o método está
melhor posicionado para reduzir o gap de complementaridade.

 \textcite{Mehrotra:1992wr} utiliza a seguinte heurística na escolha do parâmetro
 de centralização foi usada:
\begin{equation}\label{eq:Mehrotra-centralizer-parameter}
\eta = \left(\frac{\tau_{\text{af}}}{\tau}\right)^3
\end{equation}
Essa escolha sugere um parâmetro de barreira dado por 
\begin{equation}\label{eq:Mehrotra-barrier-parameter}
\tau = \left(\frac{g_{\text{af}}}{x^Tz}\right)^2\tau_{\text{af}} =
\left(\frac{g_{\text{af}}}{x^Tz}\right)^3\frac{x^Tz}{n}.
\end{equation}
Assim, se o passo preditor progride bem, $\eta$ é escolhido pequeno e têm-se
pouca centralização. Caso contrário, $\eta$ é escolhido próximo a $\frac{x^Tz}{n}$ e busca-se
uma forte centralização. Mais geralmente, o parâmetro de
centralização pode ser escolhido como
\[
\eta = \left(\frac{\tau_{\text{af}}}{\tau}\right)^p = \left(\frac{g_{\text{af}}}{x^Tz}\right)^p.
\]

Além disso, \textcite{Mehrotra:1992wr} estudou o efeito de diferentes valores de
$p= 1,2,3,4$ em um subconjunto da de problemas da \texttt{Netlib} e conclui que não havia
 muita diferença para $p$ entre 2 e 4.

Para calcular a direção centralizadora, bastaria resolver o sistema de Newton
com lado direito dado por $(0,0,\eta \tau e)$. No entanto, será mais eficiente
calcular esse termo em conjunto com a direção corretora. 

A direção afim-escala corresponde à aproximação linear da trajetória que liga o
ponto atual ao conjunto ótimo.  Esta linearização produz um erro que pode ser
determinado. De fato, assumindo que daremos um passo completo na direção
afim-escala, temos que o vetor dos pares complementares é  dado por \[
\hat{x}\hat{z} = (x+\dex)(z + \dez) = xz + x\dez + z\dex + \dex
\dez =  \dex\dez \] já que ao usarmos o lado direito
\eqref{eq:afinne-scale-step} para resolver o sistema de Newton, temos que
$x\dez  + z\dex  = -xz$. Isso significa que, quando um passo completo é
dado, os produtos $\hat{x}_i\hat{z}_i$ se transformam em $\dex_i\dez_i $, ao
invés de se anularem como era esperado. Este é exatamente o erro da aproximação linear e
um passo corretor $(\Decox, \Decoy, \Decoz)$ é dado para tentar compensá-lo.

Pode-se ainda ver que, idealmente, deseja-se que o próximo ponto 
satisfaça  \[ (x+\Dex)(z+\Dez) =\eta\tau , \] o que é equivalente a resolver
o sistema não linear \[ z\Dex + x\Dez = -xz + \eta\tau 
 - \Dex\Dez . \] Comparando esta equação com
 \eqref{eq:KKT-Newton-step-mu-tau}, vê-se que na direção afim-escala falta
 exatamente o termo $\Dex\Dez$. Para corrigir isso, o sistema de Newton é
 resolvido com o seguinte lado direito
  \begin{equation}
 \label{eq:corrector-centralizer-step}
 \bbm 0\\ 0 \\ - \dex \dez  + \eta\tau  \ebm, 
 \end{equation}
 encontrando a direção $(\Decox, \Decoy, \Decoz)$. Note que tal direção
 combina a centralização e o termo de segunda ordem   que é o 
 erro da linearização.
 
 Uma vez que as direções preditora e corretora foram calculadas, elas são
 simplesmente adicionadas de maneira a ter a direção desejada:
\begin{equation}
\label{eq:Mehrotra-direction}
 (\Dew, \Dey, \Dez) =  (\dex, \dey, \dez) +  (\Decox, \Decoy,
 \Decoz),
\end{equation}
 e o próximo iterado será dado por
\begin{equation}
\label{eq:Mehrotra-updating}
 (x^{k+1}, y^{k+1},  z^{k+1}) =  (x^{k}, y^{k},  z^{k}) + (\al_P\Dex,\al_D \De
 y, \al_D\Dez),
\end{equation}
 em que $\al_P$ e $\al_D$ são os tamanhos de passo primal e dual dado por
 \eqref{eq:size-step}. 
 
 Resumimos o método preditor-corretor de Mehrotra no Pseudo-Código
 \ref{alg:mehrotra}.
 \begin{algorithm}
\caption{Método preditor-corretor de Mehrotra.}
\label{alg:mehrotra}
\begin{algorithmic}[0] 
\Require $(x^0,y^0,z^0)$ inicial tal que $(x^0,z^0) > 0$ 

	\Repeat
		\State Resolva o sistema de Newton com o lado direito dado por
		\eqref{eq:afinne-scale-step}. 
		 \State Encontre tamanho de passo $\al_P^{\text{af}}$ e $\al_D^{\text{af}}$
		 para a direção afim-escala \State Encontre $\tau$, dado por \eqref{eq:Mehrotra-barrier-parameter} e
		 calcule $\eta$ de acordo com \eqref{eq:Mehrotra-centralizer-parameter}.
		 \State Resolva o sistema de Newton com o lado direito dado por
		  \eqref{eq:corrector-centralizer-step}.
		  \State Encontre tamanhos de passos $\al_P$ e $\al_D$  tais que possa se dar
		  um passo factível na direção  \[(\dex, \dey, \dez) +  (\Decox, \Decoy,
		  \Decoz)\] 
		  \State Atualize o iterado conforme 
		  \eqref{eq:Mehrotra-updating}.
		   \State $k\gets k+1$
\Until{O critério de parada ser satisfeito.}

\end{algorithmic}
\end{algorithm}
 
 
 A principal vantagem do \ac{MPC-M} é que, na prática, ele produz tamanho de
 passos maiores antes de violar as restrições de não-negatividade. Isso normalmente
 significa que se está economizando número de iterações: de fato Mehrotra relata
 que temos economia da ordem de 35\%-50\% quando comparada com outras
 estratégias~\cite{Mehrotra:1992wr}. Nesse
 caso, o custo é apenas de um \emph{backsolve} a mais, já que a matriz do
 sistema de Newton é sempre a mesma. Assim, se o custo da fatoração é elevado,
 economia nos custos computacionais também é alcançada por essa técnica. De 
 fato, o método de Mehrotra é vantajoso em relação a todas as implementações
 para programação linear que usam métodos diretos para calcular as direções de
 Newton~\cite[pg. 40]{Colombo:2008wm}.
 
 
 \subsection{Método das múltiplas correções de centralização}
 
O \acl{MPC-M} é baseado na hipótese de que um passo
completo na direção corretora ocorre, o que raramente é possível. Mais que
isso,  tentar corrigir a complementaridade fazendo todos os seus produtos
iguais ao valor $\tau$ é um pedido muito forte e algumas vezes pouco eficaz. 
É possível tentar remediar estes problemas através de uma extensão do \ac{MPC-M}
que utiliza mais correções lineares de uma forma inteligente.

O \emph{Método de Múltiplas Correções} deve-se a
\textcite{Gondzio:1996uw} e é uma tentativa de melhorar a centralização do iterado
atual. Esta abordagem busca produzir produtos $x_iz_i$ mais homogêneos e forçar
um aumento do tamanho do passo corrigindo o parâmetro de centralização de
Mehrotra.

Implicitamente \textcite{Gondzio:1996uw} utiliza a Vizinhança Simétrica dada 
em \eqref{eq:symmetric-neig}, de modo a ter produtos perto o suficiente da
trajetória central. Nesta descrição o passo será dado na direção
\[
\Dew = 	\DeMew + \DeMuw
\]
em que $\DeMew$ é a direção de Mehrotra \eqref{eq:Mehrotra-direction}, à qual
uma ou mais correções $\DeMuw$ podem ser aplicadas. Outras escolhas da direção preditora 
$\DeMew$ podem ser
feitas~\cite{Colombo:2008ia}.

Dada a direção preditora-corretora $\DeMew$, seja \[ \tilde{\al_P} =
\min(\al_P+\de, 1) \quad\text{ e }\quad\tilde{\al_D} = \min(\al_D+\de, 1) \]
para algum $\de\in(0,1)$ fixo, entendido como uma quantidade desejável de
aumento do passo. Calcula-se o ponto de tentativa \[
\tilde{x} = x +  \tilde{\al_P} \DeMex,\quad\tilde{z} = z +  \tilde{\al_D}
\DeMez \] e os respectivos produtos complementares $\tilde{t} =
\tilde{x}\tilde{z}\in\Real^n$. Note que por aumentar o tamanho do passo, o
ponto de tentativa é infactível (exterior) mas em compensação é utilizado apenas
para buscar corrigir a centralização,  logo não é necessário preocupar-se
com este fato.

Obviamente os produtos $\tilde{t}$ dificilmente serão iguais a $\tau$ como
desejaríamos. Alguns componentes são bem menores que $\tau$ -- inclusive
podendo ser negativos -- e outros muito maiores. Impondo que
$\tilde{t}\in\Nset_s(\ga)$, colocamos limites superiores e inferiores para cada
par complementar, também checamos as componentes que não satisfazem a
vizinhança simétrica. 

De fato, o que faremos é  mover produtos pequenos
($\tilde{t}_i\leq \ga\tau$) para o limite inferior $\ga\tau$ e produtos grandes
($\tilde{t}_i\geq \ga^{-1}\tau$) para o limite superior $\ga^{-1}\tau$. Os
produtos que estiverem na vizinhança simétrica já satisfazem estes limites e
portante estão razoavelmente pouco dispersos em relação à média. Na prática
estamos movendo os iterados para dentro de $\Nset_s(\ga)$.


Nesse caso, o termo $\DeMuw$ será encontrado resolvendo o sistema de Newton
com o lado direito dado por
\begin{equation}
\label{eq:multiply-correctors-rhs}
r = \bbm 0\\ 0\\ \xi\ebm
\end{equation}
em que o alvo $\xi$ é definido como
\[
\xi_i = 
\begin{cases}
\ga\tau - \tilde{t}_i, & \text{se } \tilde{t}_i\leq \ga\tau,\\
\frac{1}{\ga}\tau -
\tilde{t}_i, & \text{se } \tilde{t}_i\geq \ga^{-1}\tau,\\ 
0, & \text{caso contrário}.
\end{cases}
\]

Ressalte-se que o ponto que estamos buscando não pertence à trajetória central
mas à vizinhança simétrica e que nem todos os produtos complementares sofrem
correção. Assim, o alvo dado por \eqref{eq:multiply-correctors-rhs} é
realmente alcançado. 

Dada uma nova direção, poderemos repetir o processo de atualizá-la tantas vezes
quantas forem convenientes. Basta fazer-se $\DeMe  \leftarrow \De$ e procurar
novamente a direção $\DeMu$. No entanto, o número máximo de correções de
centralização permitida depende do problema. Tal número é obtido em
\cite{Gondzio:1996uw} de modo heurístico. Note que a direção corretora
obtida é aceita contanto que o tamanho do passo tenha aumentado de uma fração
desejada.


Em códigos como o \texttt{PCx}~\cite{Czyzyk:1999hk}, o
número máximo de correções é calculado considerando-se o esforço relativo entre o
número de operações necessárias para calcular a decomposição da matriz
resultante do sistema de Newton e o número de operações necessárias para fazer o
\emph{backsolve}. Além disso estima-se a redução do número de iterações. Em
\cite{Gondzio:1996uw}  o custo da fatoração não foi
considerado. 
    

As experiências computacionais apresentadas em \cite{Gondzio:1996uw}  mostram
que a estratégia das múltiplas correções realmente é efetiva, pois os passos
primais e duais calculados para a direção composta são maiores que aqueles
correspondentes à direção preditora.
Isso resulta em uma redução do número de iterações. O custo de cada
correção neste caso é o mesmo, ao contrário de outros métodos cujo custo aumenta a cada
correção. Praticamente todos os códigos modernos utilizam-se desta estratégia
\cite[Apêndice B]{Wright:Primal-dual-interior-point:1997h}. Uma generalização
desta pode ser encontrada em \cite{Colombo:2008ia}, no qual a direção de
busca é dada por $\Dew = \DeMew + \om\DeMuw$ em que usa-se parâmetro de peso
$\om\in(0,1]$. Este peso é encontrado através de uma busca linear tal que
a solução $\hat{\om}$  maximiza o tamanho de passo $\al$.



\subsection{Ponto Inicial}

Como dissemos, códigos práticos que resolvem \ac{PL} através de \ac{MPI}
utilizam passos infactíveis, exigindo apenas que $(x,z)>0$. Nada
obstante, encontrar tal ponto seja tarefa mais fácil que encontrar
um ponto inicial factível, obviamente deseja-se que a escolha seja feita de forma a
acelerar a convergência do método. 

A escolha de um iterado inicial com boas
propriedades é crítica e é uma tarefa difícil tanto para métodos factíveis
quanto infactíveis. Pede-se que o ponto possua boas propriedades nas
seguintes características: centralidade do ponto e magnitude das
infactibilidades.

\textcite{Mehrotra:1992wr}
introduziu uma heurística para encontrar um ponto inicial que satisfaça essas hipóteses.
Nesta, resolve-se os seguintes problemas de Quadrados Mínimos
\begin{align*}
&\min\: x^Tx\quad \text{s.a.}\quad Ax = b \\
&\min_{(y,z)}\: z^Tz\quad \text{s.a.}\quad A^Ty + z = c 
\end{align*}
que tentam encontrar um ponto $(\tilde{x},\tilde{y},\tilde{z})$ que satisfaça as restrições primais e
duais. As soluções para estes problemas são dada por 
\[
\tilde{x} = A^T(AA^T)^{-1}b, \quad \tilde{y} = (AA^T)^{-1}Ac\quad \text{ e }
\quad \tilde{z} = c - A^T\tilde{y}.
\]
Tal ponto é então transladado para o ortante positivo e o ponto inicial é
dado por
\[
(x^0,y^0,z^0) = (\tilde{x}+ \vartheta_x e,\tilde{y}+\vartheta_ y e,\tilde{z}+\vartheta_z e),
\]
em que $\vartheta_x$, $\vartheta_y$ e $\vartheta_ z$ são escalares positivos tais que
$(x^0,z^0)>0$.

Essa estratégia de Mehrotra tem algumas questões a serem consideradas. Ela
depende do escalamento da matriz, é afetada pela presença de restrições
redundantes que porventura tenham escapado do pré-processamento, e não garante
que o ponto inicial seja bem centrado. Mesmo assim, é uma estratégia comum em
códigos de pontos interiores por ser considerada uma heurística muito boa para
determinar o ponto inicial, além de ter custo computacional equivalente ao de
uma iteração de um \ac{MPI}.

\textcite{Gertz:2003ji},	  em seu \emph{solver} \texttt{OOQP}, propõem uma
heurística que se baseia nos mesmos princípios -- que o ponto inicial seja ao mesmo tempo
``bem centrado'' e ``não muito infactível'' -- agora para usar um \ac{MPI} no
contexto de Programação Quadrática. Descreveremos a seguir essa heurística, no
contexto de \ac{PL}. 

Primeiramente calcula-se a \emph{norma dos dados}, que é definida como a raiz
quadrada da magnitude do maior elemento dos dados do problema, isto é, o maior
elemento em valor absoluto entre a matriz $A$ e os vetores $c$ e  $b$.
Após isso, faz-se $x$ e $z$ iguais a essa norma dos dados e $y$ igual a zero e a
partir desse ponto, resolve-se o sistema de Newton \eqref{eq:KKT-Newton-step} e
dá-se  um passo completo na direção afim-escala, encontrando-se o ponto
$(\tilde{x},\tilde{y},\tilde{z})$. Certamente esse procedimento resulta em um
ponto infactível. Procede-se então de forma similar à heurística de Mehrotra,
isto é, translada-se $(\tilde{x},\tilde{y},\tilde{z})$ para o ortante positivo. 
Para tal, calcula-se valor
o máximo da violação dos limites das variáveis, digamos 
\[ \vartheta =
\max_{i\in I_V}\{-x_i,-z_i,0\}, \] em que $I_V$ é o conjunto de índices de $x$ e $z$
que violaram a não-negatividade e translada-se o ponto com o valor \[
\tilde{\vartheta}  = 100 + 2\vartheta, \] tal que o ponto inicial seja interior
o suficiente. Com efeito, o ponto inicial será \[ (x^0,y^0,z^0) = (\tilde{x}+
\tilde{\vartheta} e,\tilde{y},\tilde{z}+\tilde{\vartheta} e).
\]
Note-se que tal ponto, assim como o ponto de Mehrotra, também tem custo
computacional de uma iteração de \ac{MPI}.

 As estratégias expostas acima podem ser compreendidas
como uma tentativa de encontrar um ponto inicial que esteja perto o suficiente
da trajetória central, mas com  produtos complementares $x_iz_i$ 
maiores que uma fração adequada de $\tau$, e que a razão \[
\frac{\norm{(r_P,r_D)}}{\tau}
\] não seja muito grande. 

Aliás, note-se que se está  utilizando a \emph{métrica} dada pela vizinhança
$\Nset(\ga,\be)$ definida em \eqref{eq:infeasible-infty-neig}. Com isso, essas
heurísticas têm como objetivo impedir que o tamanho do passo $\al$ seja muito
pequeno, ao mesmo tempo que tendem a limitar  o valor de $\norm{(r_P^k
,r_D^k)}/{\tau_k}$. Além disso, ao utilizarmos $\Nset(\ga,\be)$, deseja-se que a
infactibilidade seja levada a zero na mesma taxa que o \emph{gap} de dualidade.
Para uma  análise aprofundada de estratégias de pontos iniciais em
\ac{MPI} veja \cite{Gertz:2004cw,DApuzzo:2009ks}.




  
\subsection{Critério de Parada}

Os \ac{MPI}, ao contrário do  Simplex, por exemplo, encontram a solução apenas
assintoticamente. Por conta do parâmetro de barreira $\tau$, um \ac{MPI} jamais encontra
a solução exata do problema de \ac{PL}. Com isso, é necessário um critério de
parada que possa decidir quando o iterado atual está próximo o suficiente 
do conjunto solução. 


Códigos práticos de pontos-interiores,  geralmente geram uma sequência de
pontos estritamente factíveis $\{(x^k,y^k,z^k)\}$ para os quais $x^k$ e $z^k$
são ambos estritamente positivos. Nenhum desses iterados pode ser uma solução de
fato do \ac{PL} já que para estes, $x^k_iz^k_i>0$, para $k=0,1,2,\ldots$, o que
viola a condição de complementaridade \eqref{eq:KKT-nao-negativ}. Com efeito, os
iterados convergem para um conjunto de soluções sem de fato alcançá-lo, e para
efeitos práticos um iterado avançado  -- $(x^k,y^k,z^k)$ para $k$ suficientemente
grande --  serve muito bem como uma solução aproximada. Para outras aplicações e
para propósitos teóricos, uma solução ótima $(x^*,y^*,z^*)\in\Fset$ pode ser
necessária. Para tanto, nesses códigos,  há uma fase de
\emph{terminação finita}, a qual toma um iterado interior suficientemente avançado
para um ponto em $\Fset$, encontrando, se for o caso, a base ótima. É
necessário também saber quão avançado o iterado precisa ser, tal que
garanta-se que o número de iterações de ponto-interior não seja muito grande. Veja-se o texto de \textcite[Cap. 7]{Wright:Primal-dual-interior-point:1997h} para uma explanação completa sobre terminação finita em \ac{MPI}.

A maioria das implementações de pontos interiores, por estar trabalhando com
precisão finita, garante apenas um certo grau de exatidão ou tolerância, tanto para a factibilidade quanto para a otimalidade. Grande parte
dos códigos utiliza os seguintes critérios~\cite{Gondzio:1996:CVI,Czyzyk:1999hk}:	


\begin{subequations}
\label{eq:termination-criteria}
\begin{align}
	\dfrac{\norm{Ax - b}}{1 + \norm{x}_{\infty}} \leq \tolL,
	\label{eq:termination-criteria-primal}\\ 
	\dfrac{\norm{A^Ty + z - c}}{1 + \norm{z}_{\infty}} \leq
	\tolL,\label{eq:termination-criteria-dual} \\ 
	\dfrac{\abs{c^Tx - b^T y}}{1 + \abs{b^Ty}}\leq
	\tolG.\label{eq:termination-criteria-gap}
\end{align}
\end{subequations}
Os valores de $\tol_{L}$ e $\tol_{G}$ são em geral potências de 10, e  dependem da precisão desejada. Comumente na literatura escolhe-se $\tolL=\tolG=10^{-8}$.  


Os critérios
\eqref{eq:termination-criteria-primal} e \eqref{eq:termination-criteria-dual}
exigem que os resíduos primais e duais sejam menores ou iguais a uma tolerância.
Assim estaríamos, ainda que assintoticamente, com pontos factíveis. O critério
\eqref{eq:termination-criteria-gap} faz uso do Lema
\ref{lema:strong-duality} (Dualidade Forte para \ac{PL}) e é indicador de
otimalidade. Os denominadores em cada critério são construídos de modo a 
permitirem avaliar relativamente tanto viabilidades primal e dual quanto
otimalidade.
  
Um outro conjunto de critérios muito parecido  com
\eqref{eq:termination-criteria}  é
\begin{subequations}
\label{eq:termination-criteria-pcx}
\begin{align}
	\dfrac{\norm{Ax - b}}{1 + \norm{b}} \leq \tol,
	\label{eq:termination-criteria-pcx-primal}\\
	\dfrac{\norm{A^Ty + z - c}}{1 + \norm{c}} \leq \tol,
	\label{eq:termination-criteria-pcx-dual}\\
	\dfrac{x^{T}z}{1 + \abs{c^Tx}}\leq
	\tol,\label{eq:termination-criteria-pcx-gap}
\end{align}
\end{subequations}
em que $\tol = 10^{-8}$. Note que agora, em contraste com
\eqref{eq:termination-criteria}, os denominadores de 
\eqref{eq:termination-criteria-pcx-primal} e
\eqref{eq:termination-criteria-pcx-dual} são fixos, o que de certa maneira
diminui a quantidade de operações, já que não é mais necessário encontrar a
norma  -- ainda que norma-$\infty$ --  dos vetores $x$ e $y$ em cada
iteração. O valor de $x^{T}z$,  é usado em \eqref{eq:termination-criteria-pcx-gap} ao
 invés do \emph{gap} dual  como em \eqref{eq:termination-criteria-gap}, pois além de para pontos factíveis, \emph{gap} dual e \emph{gap} de complementaridade são iguais -- vide Equação \eqref{eq:gap-complementarity} -- é razoável que se $x^{T}z$ convergir para zero, então o \emph{gap} dual também converge para zero.




Todas estas ideias representam o estado da arte no que diz respeito a \ac{MPI}
para problemas de \ac{PL}. Servirão como estrutura e base para os
desenvolvimentos que faremos neste trabalho.
