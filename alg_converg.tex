
\section{Convergência do Algoritmo de Escolha Adiada de Parâmetros}



Vamos provar  a convergência do Algoritmo \ref{alg:optimized-choice-of-parameters}. Para tanto, estratégicamente fixaremos os parâmetros $(\mu,\sig)$. Neste sentido escolheremos para o parâmetro  $\mu$ o valor \[\bar{\mu}=\eta\dfrac{x^Tz}{n},
\] 
em que $\eta\in [\eta_{\min},\eta_{\max} ]$ enquanto o parâmetro $\sig$ será fixado como $\bar{\sig} = 0$. Essas escolhas  podem fazer com que nosso método, resolva  sistemas lineares similares aos dos métodos seguidores de caminho, vistos no Capítulo~\ref{chap:mpis}.  No entanto, para fins de clareza, e sem perda de generalidade, vamos escolher $\eta_{\min}=\eta_{\max}=\eta $ como em  
\cite{Zhang:1995fu}.


 Na prática, porém, esperamos que  o método  tenha desempenho melhor do que a convergência teórica, já que se buscará, em cada iteração $k$,
o minimizador global de $\nextphi$ através da solução do problema \eqref{eq:pop-subproblem}, dado por $(\al^*,\mu^*,\sig^*)$ e portanto	$\nextphi(\al^*,\mu^*,\sig^*)\leq \nextphi(\bar{\al},\bar{\mu},\bar{\sig})$, para  $\bar{\al}$ definido abaixo.


Com essas alterações, podemos rescrever o Algoritmo~\ref{alg:optimized-choice-of-parameters}, como o  Algoritmo Simplificado \ref{alg:optimized-choice-of-parameters-simplified}.


A fim de pode  resolver o o problema primal-dual (\ref{eq:primal}-\ref{eq:dual}), vamos considerar válido o Pressuposto \ref{ass:interior-nonempty}, isto é, que o interior da região factível é não vazio. Além disso, vamos considerar que o \ac{PL} em questão possui ao menos uma solução ótima. 





Pelo Teorema \ref{thm:varphi}, a função de mérito para o próximo ponto   é, nas variáveis  $(\al,\mu,\sig)$,
\begin{equation*}
% \label{eq:merit-function-al-mu-sig}
{\nextphi}(\al,\mu,\sig) =  (1-\al)(\dbvec{\rho_L} +
\dbvec{\rho_C}) + \al\mu + \al(\al-\sig)\dbvec{L_{0,0}} +
\al^2\dbvec{\Lambda(\mu,\sig)} ,
\end{equation*}
em que 
\[
\dbvec{\Lambda(\mu,\sig)} = \mu^2
 \dbvec{L_{2,0}} + \mu \dbvec{L_{1,0}} + 	\mu \sig \dbvec{L_{1,1}} +
 \sig^2 \dbvec{L_{0,2}} + \sig \dbvec{L_{0,1}}.
 \]


Como vimos na Observação \ref{obs:L_02-L20},  $\dbvec{L_{2,0}} = \dbvec{L_{0,2}} = 0$. Ao fixarmos $\bar{\mu} = \eta\frac{x^Tz}{n} = \eta\dbvec{\rho_C} $
 e $\bar{\sig}=0$ e usarmos a Proposição \ref{prop:nu_k},   podemos rescrever a função de mérito para o próximo ponto, dependendo apenas de uma escolha de  $\al$ como
\begin{equation}
	\label{eq:simplified-merit-function-al}
\nextphi(\al)  = (1-\al)(\nu\dbvec{\rho_L}_0 +
\dbvec{\rho_C}) + \al\eta\dbvec{\rho_C} + \al^2\left(\dbvec{L_{0,0}} + \eta\dbvec{\rho_C} \dbvec{L_{1,0}}
\right) .
\end{equation}

Seja 
\begin{equation}
	\label{eq:theta}
\theta(\al) =  \dfrac{\al\left[ \nu\dbvec{\rho_L}_0 + (1-\eta)\dbvec{\rho_C} + \al\left(\dbvec{L_{0,0}} + \eta\dbvec{\rho_C} \dbvec{L_{1,0}}
\right) \right]}{\nu\dbvec{\rho_L}_0 +
\dbvec{\rho_C}}.
\end{equation}
Usando \eqref{eq:theta}, podemos escrever a seguinte relação entre a função de mérito atual e a próxima:
\begin{equation}
	\label{eq:relation-phi-next-phi}
	 			{\nextphi} = (1- \theta(\al))\varphi
\end{equation}

Como precisa-se garantir que  $\nextphi  $  seja não negativo, deve-se escolher um tamanho de passo,  $\al_k$, em cada iteração $k$ ,tal que  $\theta_k = \theta(\al_k)<1$. Além disso, se existir $\theta>0$, tal que $\theta = \liminf \theta_k$, isto é, se a sequência $\{\theta_k\}$ for limitada inferiormente por um valor positivo, então a sequência $\{\varphi_k\}$, gerada pelo Algoritmo \ref{alg:optimized-choice-of-parameters-simplified}, converge para zero Q-linearmente \cite{Ortega:2000vd}.





% Com efeito, observe que como $\eta<1$, então $0<\theta'(0)<1$ e por continuidade, temos que é possível escolher $\al_k\in(0,1]$ tal que 
% \[
% 0 < \theta_k = \theta(\al_k) < 1.
% \]

 Em resumo, para garantirmos a convergência do método, é crucial que exista $\bar{\al}>0$  tal que,  para toda iteração $k$, tenha-se  $\al_k\in(0,\bar{\al}]$, de modo que a equação \eqref{eq:relation-phi-next-phi} seja válida e  e além disso que todos os  ponto  $({x}^{k} ,{y}^{k},{z}^{k})$ pertença a vizinhança $\Nset_{-\infty}(\gamma,\beta)$.

 Essa propriedade garante que  o Algoritmo \ref{alg:optimized-choice-of-parameters-simplified} gere uma sequência $\{\varphi_k\}$ que decresce de maneira significante a cada passo, com $\varphi_k \to 0$. Assim, garantimos que o Algoritmo gera pontos $({x}^{k} ,{y}^{k},{z}^{k})$ que tem as boas propriedades da vizinhança $\Nset_{-\infty}(\gamma,\beta)$ e ainda que $({x}^{k} ,{y}^{k},{z}^{k})$ convirga para  alguma solução ótima $(\xstar ,\ystar,\zstar)$ do problema primal-dual  (\ref{eq:primal}-\ref{eq:dual}).




\section{Ponto Inicial}

Vamos  estabelecer um ponto inicial adequado para utilizarmos em nosso Algoritmo. Para tanto, tomaremos como hipótese que existe uma solução ótima $(\xstar,\ystar,\zstar)$  do problema (\ref{eq:primal}-\ref{eq:dual}) para a qual é válido  
\begin{equation}
	\label{eq:norm-xz-star}
	\norm{(\xstar,\zstar)}_{\infty} \leq \bdxzstar,
\end{equation}
em que 
 \[
 	\bdxzstar = \max\{\abs{A_{ij}},\abs{b_{i}},\abs{c_{j}}, \text{ para } 1\leq i\leq m \text{ e } 1\leq j \leq n  \}
 \]
é o maior valor absoluto de todos os dados de entrada do problema. 

%FIXME comentar a respeito dessa hipótese, depois dos resultados numéricos
Definimos nosso ponto inicial como

\begin{equation}
	\label{eq:initial-point}
	(\xzero ,\xzero,\zzero ) = (\bdxzstar e, 0, \bdxzstar e ).
\end{equation}
Naturalmente nesses termos, $\xstar_{i}\leq   \bdxzstar$, bem como $\zstar_{i}\leq \bdxzstar $, para $i=1,\ldots,n$ e por isso
\begin{equation}
	\label{eq:x0}
0\leq x^{0} - \xstar \leq \bdxzstar e \quad \text{e} \quad 0\leq \zzero  - \zstar \leq \bdxzstar e. 
\end{equation}

\section{Resultados técnicos}



A fim de demonstrarmos a convergência e a complexidade do Algoritmo proposto, na presente seção provaremos alguns lemas técnicos que serão utilizados para estabelecer nossos principais resultados de convergência.

Antes disso, faz-se necessário  escolher  $\gamma\in(0,1)$  e $\be\geq 1$ adequados, para construir a vizinhança $\Nset_{-\infty}(\ga,\be)$. Para o primeiro parâmetro, várias escolhas são possíveis como, por exemplo, a de 
\textcite{Colombo:2008ia} que utilizam $\ga = 1/10$. No entanto, vamos escolher $\ga$ tal que  
\[
\gamma \leq \frac{\min(\xzero \zzero )}{(\xzero )^T\zzero /n},
\]
o que garante que o ponto inicial satisfaz a desigualdade~\eqref{eq:symmetric-polynomials-b}.

Com relação à $\be$,  usando um raciocínio parecido, para qualquer  $\be\geq 1$, o ponto inicial sempre satisfaz \eqref{eq:symmetric-polynomials-a}. Quanto maior for $\be$, mais acelerada é a redução  das médias dos resíduos lineares, dadas por $\dbvec{\rho_L}$, em relação à redução da média dos resíduos complementares $\dbvec{\rho_C}$. 


A seguinte observação  é útil em vários momentos da análise que segue. Seja $(\tilde{x},\tilde{y},\tilde{z})$ um ponto tal que 
\begin{equation}
	\label{eq:Ax0-ATyz0}
	A\tilde{x}=0 \text{ e } A^T\tilde{y} + \tilde{z} = 0,
\end{equation}
então
\begin{equation}
	\label{eq:xTziszero}
	(\tilde{x})^T\tilde{z} =  - (\tilde{x})^T(A^T\tilde{y})  = -(A\tilde{x})^Ty = 0.
\end{equation}


Esta análise é  baseada naquelas feitas por \textcite[cap. 6]{Wright:Primal-dual-interior-point:1997h},  \textcite{Zhang:2006ic,Zhang:1995fu}. 

O primeiro lema importante mostra a existência de um limitante para $\nu_k\norm{(\xk,\zk)}_{1}$. Sua demonstração pode ser encontrada em~\cite[Lema 6.1]{Wright:Primal-dual-interior-point:1997h}, fonte da qual a transcrevemos com as adaptações pertinentes à nossa notação. Ademais, na demonstração deste e dos outros resultados, omitiremos o índice da iteração $k$ dos vetores, para fins de clareza. 


% \subsection{Limitante para \texorpdfstring{$\nu_k\norm{(\xk,\zk)}_{1}$}{a norma dos iterandos}}


\begin{lema}\label{lemma:boundxz1}
	Suponha que o ponto inicial seja dado por \eqref{eq:initial-point}. Então para qualquer iterado $(\xk,\yk,\zk)$,  temos que 
	\begin{equation}
		\label{eq:bound-xkzk}
		\bdxzstar\nu_k\norm{(\xk,\zk)}_{1} \leq  4\be n \dbvec{\rho_C^{k}}
	\end{equation}
\end{lema}

\begin{proof}
	Seja o ponto auxiliar
	\[
		(\tilde{x},\tilde{y},\tilde{z})  = \nu(\xzero,\yzero,\zzero) + (1-\nu)(\xstar,\ystar,\zstar) - (x,y,z). 
	\]
em que $(\xzero,\yzero,\zzero)$ é ponto inicial dado por \eqref{eq:initial-point} e $(\xstar,\ystar,\zstar)$ é  solução ótima que satisfaz \eqref{eq:norm-xz-star}.
É fácil ver que tal ponto satisfaz \eqref{eq:Ax0-ATyz0} e portanto vale \eqref{eq:xTziszero}, isto é, $(\tilde{x})^{T}\tilde{z} = 0$. 

Logo temos que 
\begin{align}
	0  = \tilde{x}^{T}\tilde{z} & = (\nu\xzero + (1-\nu)\xstar - x)^{T}(\nu\zzero + (1-\nu_k)\zstar - x)\notag\\
	  & = \nu^{2}(\xzero)^T\zzero + (1-\nu)^{2}(\xstar)^{T}\zstar + \nu(1-\nu)((\xzero)^{T}\zstar + (\xstar)^{T}\zzero) \notag\\
	  & \textcolor{white}{=} + x^{T}z - \nu (x^{T}\zzero + (\xzero)^{T}z) - (1-\nu) (x^{T}\zstar + (\xstar)^{T}z).\label{eq:boundxz1}
\end{align}

Como todas as componentes de $(x,z)$ e $(\xstar,\zstar)$ são não-negativas, vale $(x^{T}\zstar + (\xstar)^{T}z) \geq 0$. Além disso, $(\xstar,\ystar,\zstar)$ é uma solução ótima, e por isso $(\xstar)^{T}\zstar = 0$. Usando essas observações e levando em conta que $\nu\in(0,1)$,  pode-se reorganizar  a equação \eqref{eq:boundxz1} como
\begin{equation}
	\label{eq:xTzero+xzeroTz}
	  \nu (x^{T}\zzero + (\xzero)^{T}z) \leq 
  \nu^{2}(\xzero)^T\zzero  +  x^{T}z+  \nu(1-\nu)((\xzero)^{T}\zstar + (\xstar)^{T}\zzero).
\end{equation}

Note agora que 
\begin{equation}
\label{eq:normxz1}
	x^{T}\zzero + (\xzero)^{T}z = \sum_{i=1}^{n} \bdxzstar\abs{x_i} +\sum_{i=1}^{n} \bdxzstar\abs{z_i} = \bdxzstar\norm{x}_{1} + 
	\bdxzstar\norm{z}_{1} = \bdxzstar\norm{(x,z)}_{1},
\end{equation}
e que
\begin{equation}
\label{eq:norm-star-zero-infty-1}
	(\xzero)^{T}\zstar + (\xstar)^{T}\zzero  = \norm{\xzero}_{\infty}\norm{\zstar}_{1} + \norm{\zzero}_{\infty}\norm{\xstar}_{1}  = \norm{(\xzero,\zzero)}_{\infty}\norm{(\xstar,\zstar)}_{1}.
\end{equation}


Além disso, pela escolha do ponto inicial, temos que 
\begin{gather*}
	\norm{(\xzero,\zzero)}_\infty = \bdxzstar \label{eq:initial-conseq-a}\\
	\norm{(\xstar,\zstar)}_1\leq 2n\norm{(\xstar,\zstar)}_{\infty} \leq 2n \bdxzstar \label{eq:initial-conseq-b}\\
	(\xzero)^{T}\zzero = n\dbvec{\rho_C^0} = n\bdxzstar^{2}\label{eq:initial-conseq-c}
\end{gather*}

Substituindo esses valores em \eqref{eq:xTzero+xzeroTz}, utilizando \eqref{eq:normxz1}, \eqref{eq:norm-star-zero-infty-1} e o fato de que $(x,y,z)$ pertence à vizinhança $\Nset_{\infty}(\ga,\be)$ obtemos
\begin{align*}
	\bdxzstar\nu\norm{(x,z)}_{1} &\leq \nu^{2}n\dbvec{\rho_C^0} + n\dbvec{\rho_C} + \nu(1-\nu)(\bdxzstar2n\bdxzstar) \\ 
								&\leq \nu n\dbvec{\rho_C^0}+ n\dbvec{\rho_C} + \nu(2n\bdxzstar^{2}) \\ 
								&\leq \be n\dbvec{\rho_C}+ n\dbvec{\rho_C} + \be\left({\dbvec{\rho_C}}/{\dbvec{\rho_C^{0}}}\right)(2n\bdxzstar^{2}) \\	
								&\leq 4\be n \dbvec{\rho_C},									
\end{align*}
conforme queríamos demonstrar.
\end{proof}





\subsection{Limitantes para a normas das direções} 

A próxima proposição será utilizada nas demonstrações que seguem e é um fato fácil de se verificar.

\begin{prop}\label{prop:norm-uv}
	Seja $D$ uma matriz diagonal não-singular de ordem $n$ e sejam $u,v\in\Real^n$, então 
	\begin{equation}
		\label{eq:prop-uvDuDv}
		\norm{uv}\leq \norm{uv}_1 \leq \norm{Du}\norm{D^{-1}v} \leq \frac{1}{2}\left(\norm{Du}^{2} + \norm{D^{-1}v}^{2}\right)
	\end{equation}
\end{prop}
\begin{proof} A primeira desiguldade é valida por conta da equivalência de normas~\cite{Golub:1996wp}. Quanto às demais, vejamos que 
	\begin{align*}
		\norm{uv}_1^{2} & = \left( \sum_{i=1}^{n}u_i v_i  \right)^{2} \leq  \sum_{i=1}^{n}\abs{u_iv_i}^{2}\\
						& = \norm{uv}^{2} = (uv)^{T}(uv) \\
						& = v^{T}D^{-T}D^{T}u^{T}DD^{-1} uv\\
						& = v^{T}D^{-T}D^{T}u^{T}DuD^{-1}v\\
						& = \norm{Du}^{2}\norm{D^{-1}v}^{2}.
	\end{align*}
e logo a segunda desigualdade de \eqref{eq:prop-uvDuDv} está provada. 

Com relação a terceira desigualdade, note que  
\[
0 \leq \left(\norm{Du}^{1} - \norm{D^{-1}v}^{2}\right)^{2} = \norm{Du}^{2} - 2\norm{Du}\norm{D^{-1}v} + \norm{D^{-1}v}^{2}
\]
e portanto temos a validade de
\[
\norm{Du}\norm{D^{-1}v} \leq \frac{1}{2}\left(\norm{Du}^{1} + \norm{D^{-1}v}^{2}\right). \qedhere
\]
\end{proof}

De agora em diante, definimos a matriz $D$ como sendo
\[
D = X^{-1/2}Z^{1/2}.
\]


\begin{lema}\label{lemma:boundDxDzaff}
	Suponha um ponto inicial escolhido como em \eqref{eq:initial-point}. Então existe uma constante $\omega_1>1$ e  independente de $n$ tal que 
	\begin{equation}
		\norm{D^{k} (\dex)^{k} }\leq \omega_1 n\dbvec{\rho_C^{k}}^{1/2}\text{ e } \quad  \norm{(D^{k})^{-1}(\dez)^{k}}\leq \omega_1 n\dbvec{\rho_C^{k}}^{1/2}.
	\end{equation}
\end{lema}
\begin{proof}
	Seja o ponto auxiliar
	\[
		(\tilde{x},\tilde{y},\tilde{z})  = (\dex,\dey,\dez) + \nu(\xzero - \xstar,\yzero - \ystar,\zzero - \zstar), 
	\]
em que $(\xzero,\yzero,\zzero)$ é ponto inicial dado por \eqref{eq:initial-point} e $(\xstar,\ystar,\zstar)$ é  solução ótima que satisfaz \eqref{eq:norm-xz-star}.
	Também é fácil  ver que tal ponto satisfaz \eqref{eq:Ax0-ATyz0} e logo 
	\begin{equation}
		\label{eq:DxDzaff1}
			 0 = \tilde{x}^{T}\tilde{z} = \left(\dex +  \nu(\xzero - \xstar)\right)^{T}\left(\dez +  \nu(\zzero - \zstar)\right).
	\end{equation}
	

	 Usando este ponto auxiliar na equação \eqref{eq:affine-scaling-system-compl}, obtemos 
	 \[
	 Z(\left(\dex +  \nu(\xzero - \xstar)\right)+ X \left(\dez +  \nu(\zzero - \zstar)\right) = -xz + \nu Z(\xzero - \xstar) + \nu X(\zzero - \zstar).
	 \]
	 Multiplicando toda essa expressão por $(XZ)^{-1/2}$, e notando que $D=(XZ)^{-1/2}Z$ e que $D^{-1}=(XZ)^{-1/2}X$, resulta em 
	\begin{multline}
		\label{eq:DxDzaff2}
		 D\left(\dex +  \nu(\xzero - \xstar)\right)+ D^{-1} \left(\dez +  \nu(\zzero - \zstar)\right) = \\ -(xz)^{1/2} + \nu D(\xzero - \xstar) + \nu D^{-1}(\zzero - \zstar).
		\end{multline}
	 
	 Note que, como vale  \eqref{eq:DxDzaff1}, pode-se utilizar o Teorema de Pitágoras para norma-2 de vetores e logo, tomando a norma-2 ao quadrado do lado esquerdo de \eqref{eq:DxDzaff2} obtemos
	\begin{multline*}
			 	 \norm{D\left(\dex +  \nu(\xzero - \xstar)\right)+ D^{-1} \left(\dez +  \nu(\zzero - \zstar)\right)}^{2} =\\
	 	 	 	 	 \norm{D\left(\dex +  \nu(\xzero - \xstar)\right)}^{2}+ \norm{D^{-1} \left(\dez +  \nu(\zzero - \zstar)\right)}^{2}.
	\end{multline*}
	 Se  usarmos este resultado e a desigualdade triangular, após tomarmos a norma-2 ao quadrado de ambos os lados de \eqref{eq:DxDzaff2}, obtemos 
	 \begin{multline}\label{eq:DxDzaff3}
	 \norm{D\left(\dex +  \nu(\xzero - \xstar)\right)}^{2}+ \norm{D^{-1} \left(\dez +  \nu(\zzero - \zstar)\right)}^{2} \leq \\ 
	 \left\{ \norm{(xz)^{1/2}} + \nu \norm{D(\xzero - \xstar)} + \nu \norm{D^{-1}(\zzero - \zstar)} \right\}^{2}.
	 \end{multline}

Isolando o primeiro termo dessa equação, obtemos
\[
\norm{D\left(\dex +  \nu(\xzero - \xstar)\right)} \leq  
	  \norm{(xz)^{1/2}} + \nu \norm{D(\xzero - \xstar)} + \nu \norm{D^{-1}(\zzero - \zstar)}.
\]
Uma aplicação direta da desigualdade triangular e a adição de um termo  $\nu \norm{D^{-1}(\zzero - \zstar)}$ extra resulta em
\begin{align}
	\norm{D\dex} & =  \norm{D\left(\dex +  \nu(\xzero - \xstar)  -  \nu(\xzero - \xstar)\right)} \notag \\ 
				 & \leq	 \norm{D\left(\dex +  \nu(\xzero - \xstar)\right)} +  \nu\norm{D(\xzero - \xstar)} \notag \\
				 & \leq \norm{(xz)^{1/2}} + 2 \nu \norm{D(\xzero - \xstar)} + 2\nu \norm{D^{-1}(\zzero - \zstar)}.\label{eq:DxDzaff4}
\end{align}

Vamos mostrar a existência de um limitante para cada termo do lado direito de \eqref{eq:DxDzaff4} de magnitude $\Oset(\dbvec{\rho_C}^{1/2})$. Neste caso, o mesmo se aplicará à $\norm{D^{-1}\dez}$ o que finalizará esta demonstração.

Para o primeiro termo, note que 
\begin{equation}
	\label{eq:DxDzaff5}
	\norm{(xz)^{1/2}} = \left(\sum_{i=1}^{n}x_iz_i\right)^{1/2} = (x^{T}z)^{1/2} = n^{1/2}\dbvec{\rho_C}^{1/2} \leq \frac{n}{\ga^{1/2}}\dbvec{\rho_C}^{1/2},
\end{equation}
já que $\ga\in(0,1)$ e $\sqrt{n}\leq n$, para todo $n$ natural.

Para os últimos dois termos, considere primeiramente que pela escolha do ponto inicial, vale \eqref{eq:x0} e por conta disso, 
\[
\norm{\xzero - \xstar} \leq \bdxzstar \text{ bem como } \norm{\zzero - \zstar} \leq \bdxzstar.
\]
Por outro lado, a norma-2 da matriz  $D$ será
\[
 \norm{D} = \max_{i=1,\ldots,n} \abs{D_{ii}} = \norm{De}_{\infty} = \norm{(XZ)^{-1/2}z}_{\infty} \leq \norm{(XZ)^{-1/2}}\norm{z}_{1},
\]
e similarmente
\[
\norm{D^{-1}} \leq  \norm{(XZ)^{-1/2}} \norm{x}_{1}.
\]

Mais que isso, como $(x,y,z)\in\Nset_{-\infty}(\ga,\be)$, temos que 
\begin{equation}
\label{eq:DxDzaff6}
	\norm{(XZ)^{-1/2}} = \max_{i=1,\ldots,n}\frac{1}{(x_iz_i)^{1/2}} \leq \frac{1}{\ga^{1/2}\dbvec{\rho_{C}}^{1/2}}.
\end{equation}

Com essas desigualdades para norma de $D$ e de $D^{-1}$, utilizando consistência de norma de matrizes, segue 
\begin{align*}
 \nu \norm{D(\xzero - \xstar)} + \nu \norm{D^{-1}(\zzero - \zstar)}  & \leq \nu \norm{D}\norm{\xzero - \xstar} + \nu \norm{D^{-1}}\norm{\zzero - \zstar} \notag \\
 					& \leq \nu\bdxzstar\left[  \norm{D} + \norm{D^{-1}}   \right] \notag \\ 
 					& \leq \nu\bdxzstar\norm{(x,z)}_{1}\norm{(XZ)^{-1/2}} .\notag
\end{align*}

Da Equação~\eqref{eq:DxDzaff6}, do Lema~\ref{lemma:boundxz1} e da desigualdade acima, segue que 
\begin{equation} \label{eq:DxDzaff7}
 \nu \norm{D(\xzero - \xstar)} + \nu \norm{D^{-1}(\zzero - \zstar)}  \leq 4\be n \dbvec{\rho_C} \frac{1}{\ga^{1/2}\dbvec{\rho_{C}}^{1/2}} = \frac{4\be}{\ga^{1/2}}n\dbvec{\rho_C}^{1/2}.
\end{equation}

O resultado é obtido por observar \eqref{eq:DxDzaff5} e \eqref{eq:DxDzaff7}, comparar com \eqref{eq:DxDzaff4} e escolher 
\[
\omega_1 = \frac{9\be}{\ga^{1/2}},
\]
Como $\be\geq1$ e $\ga\in(0,1)$, temos que $\omega_1>1$ como requisitado.
\end{proof}


\begin{lema}\label{lemma:boundDxDzc}
	Suponha um ponto inicial escolhido como em \eqref{eq:initial-point}. Então existe uma constante $\omega_2\geq1$ e  independente de $n$ tal que 
	\begin{equation}\label{eq:lemma-boundDxDzcx}
		\norm{D^{k}(\Decox)^{k}}^{2} + \norm{(D^{k})^{-1}(\Decoz)^{k}}^{2} \leq \omega_2 n^{4}\dbvec{\rho_C^{k}}.
	\end{equation}
\end{lema}



\begin{proof}
	Note que por conta de \eqref{eq:corrector-nonlinear}, temos que $(\Decox,\Decoy,\Decoz)$ satisfaz \eqref{eq:Ax0-ATyz0} e logo $(\Decox)^{T}\Decoz = 0$. 

	


	Além disso, temos que
	\[
	z\Decox + x\Decoz = \mu e  - \sig\dex\dez.
	\]

	Multiplicando toda equação anterior por $(xz)^{-1/2}$ e  fixando $\mu = \mubar$ e $\sig = \sigbar$, segue que 
	\[
		D\Decox + D^{-1}\Decoz = (xz)^{-1/2}(\mubar e + \sigbar\dex\dez).
	\] 

	Por um lado temos que 
	\[
		\norm{D\Decox + D^{-1}\Decoz}^{2} = \norm{D\Decox}^{2} + \norm{D^{-1}\Decoz}^{2} + 2(\Decox)^{T}\Decoz = \norm{D\Decox}^{2} + \norm{D^{-1}\Decoz}^{2}.
	\]

Assim, usando consistência de normas e o fato de que  $(x,y,z)\in\Nset_{-\infty}(\ga,\be))$ obtemos
\begin{align}
	\norm{D\Decox}^{2} + \norm{D^{-1}\Decoz}^{2} & = \norm{(xz)^{-1/2}(\mubar e+ \sigbar\dex\dez)}^{2} \notag\\ 
												& \leq \norm{(xz)^{-1/2}}^{2}\norm{(\mubar e+ \sigbar\dex\dez)}^{2} \notag\\ 
												& \leq \min\left({x_iz_i}\right)^{-1}\left(\mubar\norm{ e} +\abs{\sigbar}\norm{\dex\dez}\right)^{2}\notag\\
												& \leq (\ga\dbvec{\rho_C})^{-1}\left(\mubar\sqrt{n} +\abs{\sigbar}\norm{\dex\dez}\right)^{2} \label{eq:DxDzc1}		
\end{align}

Com as desigualdades da~\eqref{eq:prop-uvDuDv} da Proposição~\ref{prop:norm-uv} e o Lema~\ref{lemma:boundDxDzaff}, temos que 
\[
  	\norm{\dex\dez}  \leq \frac{1}{2}\left(\norm{D\dex}^{2} + \norm{D^{-1}\dez}^{2}\right)
  					 \leq \left(\omega_1 n\dbvec{\rho_C}^{1/2}\right)^{2}
  					 = \omega_1^{2} n^{2}\dbvec{\rho_C}.
\]

Utilizando esta desigualdade, fazendo $\mubar =\eta \dbvec{\rho_C}$ como anteriormente e notando que para todo $n$ natural vake $\sqrt{n} \leq n^{2}$, o lado esquerdo de~\eqref{eq:DxDzc1} é tal que
\begin{align*}
	% (\ga\dbvec{\rho_C})^{-1}\left(\mubar\sqrt{n} +\abs{\sigbar}\norm{\dex\dez}\right)^{2}  
	\norm{D\Decox}^{2} + \norm{D^{-1}\Decoz}^{2}
							& \leq (\ga\dbvec{\rho_C})^{-1}\left(\eta\dbvec{\rho_C}\sqrt{n} +\abs{\sigbar} \omega_1^{2} n^{2}\dbvec{\rho_C}\right)^{2}\\
							& \leq (\ga)^{-1}\left(\eta\sqrt{n} +\abs{\sigbar} \omega_1^{2} n^{2}\right)^{2}\dbvec{\rho_C}\\
							& \leq (\ga)^{-1}\left(\eta +\abs{\sigbar} \omega_1^{2} \right)^{2}n^{4}\dbvec{\rho_C}.
\end{align*}

Escolhendo \[\omega_2 = \frac{\left(\eta +\abs{\sigbar} \omega_1^{2} \right)^{2}}{\ga}, \]
obtemos o resultado esperado.

Além disso,  observando que $\omega_1^{2}\geq 1$, se $\eta \geq \sqrt{\gamma}$, para todo $\sigbar$ real, garantimos que $\omega_{2}\geq 1$. 
\end{proof}

\begin{lema}\label{lemma:boundDxDzaff-c}
	Suponha um ponto inicial escolhido como em \eqref{eq:initial-point}; então 
	\begin{equation}\label{eq:lemma-boundDxDzaff-c}
		\norm{\dex\Decoz + \Decox\dez}_{1} \leq  \tfrac{1}{2}\omega_{3}n^{3}\dbvec{\rho_C},
	\end{equation}
em que $\omega_{3} = 4(\omega_1\omega_2^{1/2})\geq 4 $.
\end{lema}

\begin{proof}
	Da equação \eqref{eq:prop-uvDuDv} e dos Lemas~\ref{lemma:boundDxDzaff} e \ref{lemma:boundDxDzc} segue que  
	\begin{align*}	
		\norm{\dex\Decoz}_{1} & \leq \norm{D\dex}\norm{D^{-1}\Decoz}\\
							  & \leq 	\omega_1 n\dbvec{\rho_C}^{1/2}(\omega_2 n^{4}\dbvec{\rho_C})^{1/2}  \\
							  & = (\omega_1\omega_2^{1/2})n^{3}\dbvec{\rho_C}.
	\end{align*}			  
Similarmente, 	$\norm{\Decox\dez}_{1} \leq (\omega_1\omega_2^{1/2})n^{3}\dbvec{\rho_C}$. Com isso temos o resultado desejado, isto é,
\[
		\norm{\dex\Decoz + \Decox\dez}_{1} \leq 2(\omega_1\omega_2^{1/2})n^{3}\dbvec{\rho_C}.\qedhere
\]
\end{proof}

\subsection{Um limitante para \texorpdfstring{$\al_k$}{o tamanho do passo} }

A fim de que o Algoritmo~\ref{alg:optimized-choice-of-parameters-simplified} esteja bem definido, é necessário que exista para cada iteração $k$ uma tripla $(\al_k,\mu_k,\sig_k)$, de modo que seja possível encontrar um próximo ponto $(\nextx,\nexty,\nextz)$. Com efeito, considerando que fixamos os valores de $\mu$ e $\sig$ como $\bar{\mu} $ e  $\bar{\sig}$, basta encontrar um o tamanho de passo ${\al_k}>0$ tal que o próximo ponto $(\nextx,\nexty,\nextz)$ satisfaça as restrições da vizinhança 
$\Nset_{-\infty}(\gamma,\beta)$ e além disso, garanta que  $0 < \theta(\al_k) <1$. 

Para tanto, considere-se novamente que vamos fixar $\mu$ e $\sig$ com  as seguintes escolhas $ \bar{\mu} = \eta\dbvec{\rho_C} $ e $\bar{\sig}=0$.

 Primeiramente, utilizando a Equação \eqref{eq:simplified-merit-function-al}, é possível rescrever a função $g_C^i $, para $i=1,\ldots,n$, que foi dada em \eqref{eq:g-Ci_explicit}, somente dependendo de uma escolha de $\al$. De fato, tal função pode ser escrita escrita nos seguintes termos
\[
\begin{aligned}
{g}_C^i (\al)				& = (1-\al)(\rho_C)_i+ \al\eta\dbvec{\rho_C}+ \al^2\left[(L_{0,0})_i + \eta\dbvec{\rho_C} ({L_{1,0}})_i 
				+ (\eta\dbvec{\rho_C})^2(L_{2,0})_i \right]  + \\
				& \quad -\ga\left[  (1-\al)\dbvec{\rho_C} + \al\eta\dbvec{\rho_C} + \al^2\left(\dbvec{L_{0,0}} + \eta\dbvec{\rho_C} \dbvec{L_{1,0}}.
\right)  \right]
\end{aligned}
\]

Definindo as constantes
\[
\begin{aligned}
\zeta_i & = (L_{0,0})_i - \ga \dbvec{L_{0,0}}, \\
\chi_i  & = \eta\dbvec{\rho_C} \left( ({L_{1,0}})_i - \ga\dbvec{L_{1,0}} \right),   \\
\xi_i	& =  (\eta\dbvec{\rho_C})^2(L_{2,0})_i , \\ 	
\end{aligned}
\]
e usando o fato de que o ponto atual pertence à vizinhança $\Nset_{-\infty}(\gamma,\beta)$ temos
\[
\begin{aligned}
	g_C^i (\al) & = \underbrace{(1-\al)((\rho_C)_i - \ga\dbvec{\rho_C})}_{\geq 0}  + (1-\ga)\eta\dbvec{\rho_C} \al+  (\zeta_i + 				\chi_i + \xi_i)\al^2  \\
				& \geq (1-\ga)\eta\dbvec{\rho_C} \al +  (\zeta_i + 				\chi_i + \xi_i)\al^2 \\ 
				& \geq (1-\ga)\eta\dbvec{\rho_C} \al -  (\abs{\zeta_i} + \abs{\chi_i} + \abs{\xi_i})\al^2 \\
				& = \al \left[	(1-\ga)\eta\dbvec{\rho_C}  -  (\abs{\zeta_i} + \abs{\chi_i} + \abs{\xi_i})\al	\right] \\
				& = h^i(\al).
\end{aligned}
\]
em que $h^i$ é uma quadrática côncava em função de $\al$ com uma raiz nula e uma positiva. Note que, para $\al>0$ e $i=1,\ldots,n$, se $h^i(\al)\geq0$ então $g_C^i(\al)\geq 0$. 

Com efeito, a única raiz positiva de $h^i$ é dada por
\[
\al_{C}^i = \dfrac{(1-\ga)\eta\dbvec{\rho_C}}{\abs{\zeta_i} + \abs{\chi_i} + \abs{\xi_i}}
\]
e $h^i(\al)\geq$ sempre que $\al\in[0,\al_{C}^i].$ 


Assim, seja
\[
\al_C = \min_i\{\al_C^i\}.
\] 
Para $i=1,\ldots,n$, teremos $g_C^i(\al)\geq 0$, quando  
$\al\in[0,\al_C]$. 







Por outro lado, usando as mesmas substituições de $\bar{\mu}$ e $\bar{\sig}$, a função $g_L$, dada em \eqref{eq:g-L_explicit} torna-se uma função que depende apenas de $\al$, nos seguintes termos 
\[
g_L(\al) =     (1-\al)\left(\dbvec{\rho_C} -  \be_L \nu   \right) +  \al\eta\dbvec{\rho_C} + 
   \al^2\left( \dbvec{L_{0,0}} + \eta\dbvec{\rho_C}  \dbvec{L_{1,0}}   \right ) ,
	\]
Usando novamente o fato de que o ponto atual pertence à  vizinhança $\Nset_{-\infty}(\gamma,\beta)$ e definindo as seguintes constantes 
\[
\begin{aligned}
\zeta & =  \dbvec{L_{0,0}}, \\
\chi  & = \eta\dbvec{\rho_C} \dbvec{L_{1,0}},
\end{aligned}
\]
segue que 
 \[
\begin{aligned}
{g}_L(\al) & =     (1-\al)\underbrace{\left(\dbvec{\rho_C} -  \be_L \nu   \right)}_{\geq 0} +  \al\eta\dbvec{\rho_C} + 
   \al^2\left( \zeta + \chi   \right ) \\
   & \geq  \al\left[\eta\dbvec{\rho_C} - 
   \al (\abs{\zeta} + \abs{\chi})   \right ].
\end{aligned}
 \]




Seja $\al_L\in(0,1]$ o maior número tal que  $g_L(\al)\geq 0$, para  $\al\in[0,\al_L]$. Por conta da última inequação acima, temos que se 
\[
\al_L \geq \frac{\eta\dbvec{\rho_C} }{\abs{\zeta} + \abs{\chi}},
\]
 então certamente a condição \eqref{eq:symmetric-polynomials-a} será satisfeita. 






Por conta do que foi visto até aqui, devemos escolher o tamanho do passo $\al_k$ tal que 
\begin{equation}
\label{eq:bound-alpha} 
	\al_k \leq \bar{\al} = \arg\max \{\theta(\al):\al\in[0,\min\{\al_C,\al_L\}] \}
\end{equation}





\section{Procurar limitantes do que?}

\textcolor{red}{Esta seção será rescrita. Apenas está aqui para mostrar os  limitantes procurados.}


Observe que é preciso encontrar limitantes para $ \abs{\zeta}, \abs{\chi}, \abs{\zeta_i}, \abs{\chi_i} \text{ e } \abs{\xi_i}$.
Com efeito por conta  das definições dessas constantes acima  e do Teorema \ref{thm:next=residual} valem as seguintes observações:
\begin{enumerate}[(i)]
	\item $\abs{\zeta} = \abs{\dbvec{L_{0,0}}} = \abs{\dfrac{(\dex)^T\dez}{n}} \leq \dfrac{1}{n} \norm{\dex\dez}_1$.
	\item Como $\sig=0$, usando a equação \eqref{eq:Corrector-spllited}, temos que 
	\[
		\Decox = \mu\Dex^\mu \text{ e } \Decoz = \mu\Dez^\mu.
	\]
	Assim,  
	\[
	\begin{aligned}
	\abs{\chi} & = \abs{\eta\dbvec{\rho_C} \dbvec{L_{1,0}}} = \abs{\dfrac{(\dex)^T(\bar{\mu}\Dez^\mu) + (\bar{\mu}\Dex^\mu)^T\dez}{n}   }  \\ 
	&  = \abs{\dfrac{(\dex)^T(\Decoz) + (\Decox)^T\dez}{n} }  \\
	& \leq \dfrac{1}{n} \norm{\dex\Decoz + \Decox\dez}_1.
	\end{aligned}
	\]

\item $\abs{\zeta_i}  = \abs{(L_{0,0})_i - \ga \dbvec{L_{0,0}}} \leq \abs{(L_{0,0})_i} + \ga \abs{\dbvec{L_{0,0}}} \leq
					\norm{L_{0,0}}_1  + \dfrac{\ga}{n}\norm{L_{0,0}}_1 \leq 2\norm{L_{0,0}}_1 = 2\norm{\dex\dez}_1.$
\item Utilizando as ideias de (ii) segue que  
\[
\begin{aligned}
	\abs{\chi_i }  & = \abs{\eta\dbvec{\rho_C} \left( ({L_{1,0}})_i - \ga\dbvec{L_{1,0}} \right)} \\
	&  \leq 
						\abs{\eta\dbvec{\rho_C}  ({L_{1,0}})_i} + \ga\abs { \eta\dbvec{\rho_C}\dbvec{L_{1,0}} } \\
						& \leq \norm{\dex\Decoz + \Decox\dez}_1 + \dfrac{\ga}{n}\norm{\dex\Decoz + \Decox\dez}_1	\\
						& \leq 2 \norm{\dex\Decoz + \Decox\dez}_1
\end{aligned}
\]

\item Novamente, considerando $\sig=0$, usando a equação \eqref{eq:Corrector-spllited},  vale
\[
	(\eta\dbvec{\rho_C})^2(L_{2,0}) = (\eta\dbvec{\rho_C}\Dex^\mu)(\eta\dbvec{\rho_C}\Dez^\mu) = \Decox\Decoz.
\]
Assim
\[
	\abs{\xi_i}	 =  \abs{(\eta\dbvec{\rho_C})^2(L_{2,0})_i}  = \abs{(\Decox\Decoz)_i} \leq \norm{\Decox\Decoz}_1
\]

\end{enumerate}

Assim precisamos de limitantes para $\norm{\dex\dez}_1$, $\norm{\dex\Decoz + \Decox\dez}_1$ e $\norm{\Decox\Decoz}_1$.


% \chi_i  & = \eta\dbvec{\rho_C} \left( ({L_{1,0}})_i - \ga\dbvec{L_{1,0}} \right),   \\
% \xi_i	& =  (\eta\dbvec{\rho_C})^2(L_{2,0})_i , \\ 

  \begin{algorithm}[htb]
 \onehalfspacing
 \caption{Método de Escolha Adiada Simplificado.}
 \label{alg:optimized-choice-of-parameters-simplified} 
\begin{algorithmic}[1]
\Procedure{ResolveLP}{$A,b,c$}
\State $(\xzero ,\yzero,\zzero ) \gets$ \Call{PontoInicial}{$A,b,c$}.
\Comment{Assegure que  $(\xzero ,\zzero )>0$ e que $\eta\in(0,1)$}
	\For {$k=1,2,\ldots$}
		\State Encontre		$((\dex)^{k},(\dey)^{k},(\dez)^{k})$ resolvendo
				\begin{equation}
				\label{eq:predictor-linear-matrix-simplified}
				\bbm A & 0 & 0 \\
				0 & A^T & I\\
				Z^k & 0 & X^k \ebm
				\bbm (\dex)^k \\ (\dey)^k \\ (\dez)^k
				\ebm = 
				\bbm -r_P^k  \\ -r_D^k \\ -r_C^k
				\ebm.
			\end{equation}
		\State 	Faça $\bar{\mu}=\eta{(x^k)^Tz^k}/{n}$ e  $\bar{\sig}=0$, e resolva 
		% \[((\Decox)^{k},(\Decoy)^{k},(\Decoz)^{k}) = \bar{\mu}((\Dex^{\mu})^{k},(\Dey^{\mu})^{k},(\Dez^{\mu})^{k})\]
		% resolvendo
			\begin{equation}
				\label{eq:corrector-linear-matrix-simplified}
				\bbm A & 0 & 0 \\
				0 & A^T & I\\
				Z^k & 0 & X^k \ebm
				\bbm (\Decox)^{k} \\ (\Decoy)^{k} \\ (\Decoz)^{k}
				\ebm = 
				\bbm 0  \\ 0 \\  \bar{\mu}e %- \bar{\sig}\deX\dez
				\ebm.
			\end{equation}
		\State Encontre $\al^*$ resolvendo o subproblema de
		otimização global 	\eqref{eq:pop-subproblem}.		
		\State Escolha $\al_k = \min\{\al^*,\tilde{\al}_k\}$ com $\tilde{\al}_k$ dado por $\eqref{eq:ratio-test}$ e faça
		\[
		\begin{aligned}	
		& x^{k+1} = x^{k} + \al_k((\dex)^{k} + (\Decox)^{k} )
		\\
		& y^{k+1} = y^{k} + \al_k((\dey)^{k} + (\Decoy)^{k} )
		\\
		& z^{k+1} = z^{k} + \al_k((\dez)^{k} + (\Decoz)^{k} )
		 \end{aligned}. 
		\]		
	\EndFor
\EndProcedure
\end{algorithmic}
\end{algorithm}
