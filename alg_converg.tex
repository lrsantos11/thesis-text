%!TEX root = tese.tex
\section{Convergência do Algoritmo de Escolha Adiada de Parâmetros}



Vamos provar  a convergência do Algoritmo \ref{alg:optimized-choice-of-parameters}. Para tanto, estratégicamente fixaremos os parâmetros $(\mu,\sig)$. Neste sentido escolheremos para o parâmetro  $\mu$ o valor \[\bar{\mu}=\eta\dfrac{x^Tz}{n},
\] 
em que $\eta\in [\eta_{\min},\eta_{\max} ]$ enquanto o parâmetro $\sig$ será fixado como $\bar{\sig} = 0$. Essas escolhas  podem fazer com que nosso método, resolva  sistemas lineares similares aos dos métodos seguidores de caminho, vistos no Capítulo~\ref{chap:mpis}.  No entanto, para fins de clareza, e sem perda de generalidade, vamos escolher $\eta_{\min}=\eta_{\max}=\eta $ como em  
\cite{Zhang:1995fu}.


 Na prática, porém, esperamos que  o método  tenha desempenho melhor do que a convergência teórica, já que se buscará, em cada iteração $k$,
o minimizador global de $\nextphi$ através da solução do problema \eqref{eq:pop-subproblem}, dado por $(\al^*,\mu^*,\sig^*)$ e portanto	$\nextphi(\al^*,\mu^*,\sig^*)\leq \nextphi(\bar{\al},\bar{\mu},\bar{\sig})$, para  $\bar{\al}$ definido abaixo.


Com essas alterações, podemos rescrever o Algoritmo~\ref{alg:optimized-choice-of-parameters}, como o  Algoritmo Simplificado \ref{alg:optimized-choice-of-parameters-simplified}.


A fim de pode  resolver o o problema primal-dual (\ref{eq:primal}-\ref{eq:dual}), vamos considerar válido o Pressuposto \ref{ass:interior-nonempty}, isto é, que o interior da região factível é não vazio. Além disso, vamos considerar que o \ac{PL} em questão possui ao menos uma solução ótima. 





Pelo Teorema \ref{thm:varphi}, a função de mérito para o próximo ponto   é, nas variáveis  $(\al,\mu,\sig)$,
\begin{equation*}
% \label{eq:merit-function-al-mu-sig}
{\nextphi}(\al,\mu,\sig) =  (1-\al)(\dbvec{\rho_L} +
\dbvec{\rho_C}) + \al\mu + \al(\al-\sig)\dbvec{L_{0,0}} +
\al^2\dbvec{\Lambda(\mu,\sig)} ,
\end{equation*}
em que 
\[
\dbvec{\Lambda(\mu,\sig)} = \mu^2
 \dbvec{L_{2,0}} + \mu \dbvec{L_{1,0}} + 	\mu \sig \dbvec{L_{1,1}} +
 \sig^2 \dbvec{L_{0,2}} + \sig \dbvec{L_{0,1}}.
 \]


Como vimos na Observação \ref{obs:L_02-L20},  $\dbvec{L_{2,0}} = \dbvec{L_{0,2}} = 0$. Ao fixarmos $\bar{\mu} = \eta\frac{x^Tz}{n} = \eta\dbvec{\rho_C} $
 e $\bar{\sig}=0$ e usarmos a Proposição \ref{prop:nu_k},   podemos rescrever a função de mérito para o próximo ponto, dependendo apenas de uma escolha de  $\al$ como
\begin{equation}
	\label{eq:simplified-merit-function-al}
\nextphi(\al)  = (1-\al)(\nu\dbvec{\rho_L}_0 +
\dbvec{\rho_C}) + \al\eta\dbvec{\rho_C} + \al^2\left(\dbvec{L_{0,0}} + \eta\dbvec{\rho_C} \dbvec{L_{1,0}}
\right) .
\end{equation}

Seja 
\begin{equation}
	\label{eq:theta}
\theta(\al) =  \dfrac{\al\left[ \nu\dbvec{\rho_L}_0 + (1-\eta)\dbvec{\rho_C} + \al\left(\dbvec{L_{0,0}} + \eta\dbvec{\rho_C} \dbvec{L_{1,0}}
\right) \right]}{\nu\dbvec{\rho_L}_0 +
\dbvec{\rho_C}}.
\end{equation}
Usando \eqref{eq:theta}, podemos escrever a seguinte relação entre a função de mérito atual e a próxima:
\begin{equation}
	\label{eq:relation-phi-next-phi}
	 			{\nextphi} = (1- \theta(\al))\varphi
\end{equation}

Como precisa-se garantir que  $\nextphi  $  seja não negativo, deve-se escolher um tamanho de passo,  $\al_k$, em cada iteração $k$ ,tal que  $\theta_k = \theta(\al_k)<1$. Além disso, se existir $\theta>0$, tal que $\theta = \liminf \theta_k$, isto é, se a sequência $\{\theta_k\}$ for limitada inferiormente por um valor positivo, então a sequência $\{\varphi_k\}$, gerada pelo Algoritmo \ref{alg:optimized-choice-of-parameters-simplified}, converge para zero Q-linearmente \cite{Ortega:2000vd}.





% Com efeito, observe que como $\eta<1$, então $0<\theta'(0)<1$ e por continuidade, temos que é possível escolher $\al_k\in(0,1]$ tal que 
% \[
% 0 < \theta_k = \theta(\al_k) < 1.
% \]

 Em resumo, para garantirmos a convergência do método, é crucial que exista $\bar{\al}>0$  tal que,  para toda iteração $k$, tenha-se  $\al_k\in(0,\bar{\al}]$, de modo que a equação \eqref{eq:relation-phi-next-phi} seja válida e  e além disso que todos os  ponto  $({x}^{k} ,{y}^{k},{z}^{k})$ pertença a vizinhança $\Nset_{-\infty}(\gamma,\beta)$.

 Essa propriedade garante que  o Algoritmo \ref{alg:optimized-choice-of-parameters-simplified} gere uma sequência $\{\varphi_k\}$ que decresce de maneira significante a cada passo, com $\varphi_k \to 0$. Assim, garantimos que o Algoritmo gera pontos $({x}^{k} ,{y}^{k},{z}^{k})$ que tem as boas propriedades da vizinhança $\Nset_{-\infty}(\gamma,\beta)$ e ainda que $({x}^{k} ,{y}^{k},{z}^{k})$ convirga para  alguma solução ótima $({x}^{*} ,{y}^{*},{z}^{*})$ do problema primal-dual  (\ref{eq:primal}-\ref{eq:dual}) .



% Vamos escolher um $\gamma\in(0,1)$ adequado, para construir a vizinhança. Várias escolhas são possíveis, como, por exemplo a de \textcite{Colombo:2008ia} que fazem $\ga = 1/10$. No entanto, vamos utilizar a estratégia de \textcite{Zhang:2006ic}
% \[
% \gamma \leq \frac{\min(x^0z^0)}{(x^0)^Tz^0/n},
% \]
% que garante que o ponto inicial estará dentro da vizinhança. Mais detalhes sobre a escolha de $\gamma$ será feita na Capítulo~\ref{chap:numerical}, que versa a respeito da implementação.


\subsection{Limitantes para \texorpdfstring{$\norm{\dex\dez}$ e $\norm{\Decox\Decoz}$}{normas das direções} }


\subsection{Um limitante para \texorpdfstring{$\al_k$}{tamanho do passo} }



A fim de que o Algoritmo~\ref{alg:optimized-choice-of-parameters-simplified} esteja bem definido, é necessário que exista para cada iteração $k$ uma tripla $(\al_k,\mu_k,\sig_k)$, de modo que seja possível encontrar um próximo ponto $(\nextx,\nexty,\nextz)$. Com efeito, considerando que fixamos os valores de $\mu$ e $\sig$ como $\bar{\mu} $ e  $\bar{\sig}$, basta encontrar um o tamanho de passo ${\al_k}>0$ tal que o próximo ponto $(\nextx,\nexty,\nextz)$ satisfaça as restrições da vizinhança 
$\Nset_{-\infty}(\gamma,\beta)$ e além disso, garanta que  $0 < \theta(\al_k) <1$. 

Sejam as funções  $g_C^i$, para $i=1,\ldots,n$, e $g_L$ definidas nas Equações  \eqref{eq:g-Ci_explicit} e 
\eqref{eq:g-L_explicit}.  Considere-se também  as escolhas $\mu = \bar{\mu} = \eta\dbvec{\rho_C} $ e $\bar{\sig}=0$.

 Primeiramente, utilizando a Equação \eqref{eq:simplified-merit-function-al}, é possível reescrever a função $g_C^i $, para $i=1,\ldots,n$,  somente dependendo de uma escolha de $\al$. De fato, definimos $\bar{g}_C^i(\al) = g_C^i(\al,\bar{\mu},\bar{\sig})$, que pode ser escrita nos seguintes termos 
\[
\begin{aligned}
	\bar{g}_C^i (\al)				& = (1-\al)(\rho_C)_i+ \al\eta\dbvec{\rho_C}+ \al^2\left[(L_{0,0})_i + \eta\dbvec{\rho_C} ({L_{1,0}})_i 
				+ (\eta\dbvec{\rho_C})^2(L_{2,0})_i \right]  + \\
				& \quad -\ga\left[  (1-\al)\dbvec{\rho_C} + \al\eta\dbvec{\rho_C} + \al^2\left(\dbvec{L_{0,0}} + \eta\dbvec{\rho_C} \dbvec{L_{1,0}}.
\right)  \right]
\end{aligned}
\]

Definindo as constantes
\[
\begin{aligned}
\zeta_i & = (L_{0,0})_i - \ga \dbvec{L_{0,0}}, \\
\chi_i  & = \eta\dbvec{\rho_C} \left( ({L_{1,0}})_i - \ga\dbvec{L_{1,0}} \right),   \\
\xi_i	& =  (\eta\dbvec{\rho_C})^2(L_{2,0})_i , \\ 	
\end{aligned}
\]
e usando o fato de que o ponto atual pertence à vizinhança $\Nset_{-\infty}(\gamma,\beta)$ temos
\[
\begin{aligned}
	g_C^i (\al) & = \underbrace{(1-\al)((\rho_C)_i - \ga\dbvec{\rho_C})}_{\geq 0}  + (1-\ga)\eta\dbvec{\rho_C} \al+  (\zeta_i + 				\chi_i + \xi_i)\al^2  \\
				& \geq (1-\ga)\eta\dbvec{\rho_C} \al +  (\zeta_i + 				\chi_i + \xi_i)\al^2 \\ 
				& \geq (1-\ga)\eta\dbvec{\rho_C} \al -  (\abs{\zeta_i} + \abs{\chi_i} + \abs{\xi_i})\al^2 \\
				& = \al \left[	(1-\ga)\eta\dbvec{\rho_C}  -  (\abs{\zeta_i} + \abs{\chi_i} + \abs{\xi_i})\al	\right] \\
				& = h^i(\al).
\end{aligned}
\]
em que $h^i$ é uma quadrática côncava em função de $\al$ com uma raiz nula e uma positiva. Note que, para $\al>0$ e $i=1,\ldots,n$, se $h^i(\al)\geq0$ então $g_C^i(\al)\geq 0$. 

Com efeito, a única raiz positiva de $h^i$ é dada por
\[
\al_{C}^i = \dfrac{(1-\ga)\eta\dbvec{\rho_C}}{\abs{\zeta_i} + \abs{\chi_i} + \abs{\xi_i}}
\]
e $h^i(\al)\geq$ sempre que $\al\in[0,\al_{C}^i].$ 


Assim, seja
\[
\al_C = \min_i\{\al_C^i\}.
\] 
Para $i=1,\ldots,n$, teremos $g_C^i(\al)\geq 0$, quando  
$\al\in[0,\al_C]$. 







Por outro lado, usando as mesmas substituições de $\bar{\mu}$ e $\bar{\sig}$, a função $g_L$ torna-se uma função que depende apenas de $\al$, nos seguintes termos 
\[
g_L(\al) =     (1-\al)\left(\dbvec{\rho_C} -  \be_L \nu   \right) +  \al\eta\dbvec{\rho_C} + 
   \al^2\left( \dbvec{L_{0,0}} + \eta\dbvec{\rho_C}  \dbvec{L_{1,0}}   \right ) ,
\]
Usando novamente o fato de que o ponto atual pertence à  vizinhança $\Nset_{-\infty}(\gamma,\beta)$ e definindo as seguintes constantes 
\[
\begin{aligned}
\zeta & =  \dbvec{L_{0,0}}, \\
\chi  & = \eta\dbvec{\rho_C} \dbvec{L_{1,0}},
\end{aligned}
\]
segue que 
 \[
\begin{aligned}
\bar{g}_L(\al) & =     (1-\al)\underbrace{\left(\dbvec{\rho_C} -  \be_L \nu   \right)}_{\geq 0} +  \al\eta\dbvec{\rho_C} + 
   \al^2\left( \zeta + \chi   \right ) \\
   & \geq  \al\left[\eta\dbvec{\rho_C} - 
   \al (\abs{\zeta} + \abs{\chi})   \right ].
\end{aligned}
 \]




Seja $\al_L\in(0,1]$ o maior número tal que  $g_L(\al)\geq 0$, para  $\al\in[0\,al_L]$. Por conta da última inequação acima, temos que se 
\[
\al_L \geq \frac{\eta\dbvec{\rho_C} }{\abs{\zeta} + \abs{\chi}},
\]
 então certamente a condição \eqref{eq:symmetric-polynomials-a} será satisfeita. 






Por conta do que foi visto até aqui, devemos escolher o tamanho do passo $\al_k$ tal que 
\begin{equation}
\label{eq:bound-alpha} 
	\al_k \leq \bar{\al} = \arg\max \{\theta(\al):\al\in[0,\min\{\al_C,\al_L\}] \}
\end{equation}





\section{Procurar limitantes do que?}

\textcolor{red}{Esta seção será rescrita. Apenas está aqui para mostrar os  limitantes procurados.}


Observe que é preciso encontrar limitantes para $ \abs{\zeta}, \abs{\chi}, \abs{\zeta_i}, \abs{\chi_i} \text{ e } \abs{\xi_i}$.
Com efeito por conta  das definições dessas constantes acima  e do Teorema \ref{thm:next=residual} valem as seguintes observações:
\begin{enumerate}[(i)]
	\item $\abs{\zeta} = \abs{\dbvec{L_{0,0}}} = \abs{\dfrac{(\dex)^T\dez}{n}} \leq \dfrac{1}{n} \norm{\dex\dez}_1$.
	\item Como $\sig=0$, usando a equação \eqref{eq:Corrector-spllited}, temos que 
	\[
		\Decox = \mu\Dex^\mu \text{ e } \Decoz = \mu\Dez^\mu.
	\]
	Assim,  
	\[
	\begin{aligned}
	\abs{\chi} & = \abs{\eta\dbvec{\rho_C} \dbvec{L_{1,0}}} = \abs{\dfrac{(\dex)^T(\bar{\mu}\Dez^\mu) + (\bar{\mu}\Dex^\mu)^T\dez}{n}   }  \\ 
	&  = \abs{\dfrac{(\dex)^T(\Decoz) + (\Decox)^T\dez}{n} }  \\
	& \leq \dfrac{1}{n} \norm{\dex\Decoz + \Decox\dez}_1.
	\end{aligned}
	\]

\item $\abs{\zeta_i}  = \abs{(L_{0,0})_i - \ga \dbvec{L_{0,0}}} \leq \abs{(L_{0,0})_i} + \ga \abs{\dbvec{L_{0,0}}} \leq
					\norm{L_{0,0}}_1  + \dfrac{\ga}{n}\norm{L_{0,0}}_1 \leq 2\norm{L_{0,0}}_1 = 2\norm{\dex\dez}_1.$
\item Utilizando as ideias de (ii) segue que  
\[
\begin{aligned}
	\abs{\chi_i }  & = \abs{\eta\dbvec{\rho_C} \left( ({L_{1,0}})_i - \ga\dbvec{L_{1,0}} \right)} \\
	&  \leq 
						\abs{\eta\dbvec{\rho_C}  ({L_{1,0}})_i} + \ga\abs { \eta\dbvec{\rho_C}\dbvec{L_{1,0}} } = \norm{\dex\Decoz + \Decox\dez}_1 + \dfrac{\ga}{n}\norm{\dex\Decoz + \Decox\dez}_1	\\
						& \leq 2 \norm{\dex\Decoz + \Decox\dez}_1
\end{aligned}
\]

\item Novamente, considerando $\sig=0$, usando a equação \eqref{eq:Corrector-spllited},  vale
\[
	(\eta\dbvec{\rho_C})^2(L_{2,0}) = (\eta\dbvec{\rho_C}\Dex^\mu)(\eta\dbvec{\rho_C}\Dez^\mu) = \Decox\Decoz.
\]
Assim
\[
	\abs{\xi_i}	 =  \abs{(\eta\dbvec{\rho_C})^2(L_{2,0})_i}  = \abs{(\Decox\Decoz)_i} \leq \norm{\Decox\Decoz}_1
\]

\end{enumerate}

Assim precisamos de limitantes para $\norm{\dex\dez}_1$, $\norm{\dex\Decoz + \Decox\dez}_1$ e $\norm{\Decox\Decoz}_1$.


% \chi_i  & = \eta\dbvec{\rho_C} \left( ({L_{1,0}})_i - \ga\dbvec{L_{1,0}} \right),   \\
% \xi_i	& =  (\eta\dbvec{\rho_C})^2(L_{2,0})_i , \\ 

  \begin{algorithm}[htb]
 \onehalfspacing
 \caption{Método de Escolha Adiada Simplificado.}
 \label{alg:optimized-choice-of-parameters-simplified} 
\begin{algorithmic}[1]
\Procedure{ResolveLP}{$A,b,c$}
\State $(x^0,y^0,z^0) \gets$ \Call{PontoInicial}{$A,b,c$}.
\Comment{Assegure que  $(x^0,z^0)>0$ e que $\eta\in(0,1)$}
	\For {$k=1,2,\ldots$}
		\State Encontre		$((\dex)^{k},(\dey)^{k},(\dez)^{k})$ resolvendo
				\begin{equation}
				\label{eq:predictor-linear-matrix-simplified}
				\bbm A & 0 & 0 \\
				0 & A^T & I\\
				Z^k & 0 & X^k \ebm
				\bbm (\dex)^k \\ (\dey)^k \\ (\dez)^k
				\ebm = 
				\bbm -r_P^k  \\ -r_D^k \\ -r_C^k
				\ebm.
			\end{equation}
		\State 	Faça $\bar{\mu}=\eta{(x^k)^Tz^k}/{n}$ e  $\bar{\sig}=0$, e resolva 
		% \[((\Decox)^{k},(\Decoy)^{k},(\Decoz)^{k}) = \bar{\mu}((\Dex^{\mu})^{k},(\Dey^{\mu})^{k},(\Dez^{\mu})^{k})\]
		% resolvendo
			\begin{equation}
				\label{eq:corrector-linear-matrix-simplified}
				\bbm A & 0 & 0 \\
				0 & A^T & I\\
				Z^k & 0 & X^k \ebm
				\bbm (\Decox)^{k} \\ (\Decoy)^{k} \\ (\Decoz)^{k}
				\ebm = 
				\bbm 0  \\ 0 \\  \bar{\mu}e %- \bar{\sig}\deX\dez
				\ebm.
			\end{equation}
		\State Encontre $\al^*$ resolvendo o subproblema de
		otimização global 	\eqref{eq:pop-subproblem}.		
		\State Escolha $\al_k = \min\{\al^*,\tilde{\al}_k\}$ com $\tilde{\al}_k$ dado por $\eqref{eq:ratio-test}$ e faça
		\[
		\begin{aligned}	
		& x^{k+1} = x^{k} + \al_k((\dex)^{k} + (\Decox)^{k} )
		\\
		& y^{k+1} = y^{k} + \al_k((\dey)^{k} + (\Decoy)^{k} )
		\\
		& z^{k+1} = z^{k} + \al_k((\dez)^{k} + (\Decoz)^{k} )
		 \end{aligned}. 
		\]		
	\EndFor
\EndProcedure
\end{algorithmic}
\end{algorithm}
