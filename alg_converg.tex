%!TEX root = tese.tex


\section{Convergência  Polinomial do Algoritmo de Escolha Adiada de Parâmetros}



Vamos provar  a convergência do Algoritmo \ref{alg:optimized-choice-of-parameters}. Para tanto, estratégicamente fixaremos os parâmetros $(\mu,\sig)$. Neste sentido escolheremos para o parâmetro  $\mu$ o valor \[\nextmu=\eta\dfrac{x^Tz}{n},
\] 
em que $\eta\in [\eta_{\min},\eta_{\max} ]$, enquanto o parâmetro $\sig$ será fixado como $\nextsig = 0$. Essas escolhas  podem fazer com que nosso método, resolva  sistemas lineares similares aos dos métodos seguidores de caminho, vistos no Capítulo~\ref{chap:mpis}.  No entanto, para fins de clareza, e sem perda de generalidade, vamos escolher $\eta_{\min}=\eta_{\max}=\eta $ como em  
\cite{Zhang:1995fu}.


 Na prática, porém, esperamos que  o método  tenha desempenho melhor do que a convergência teórica, já que se buscará, em cada iteração $k$,
o minimizador global de $\nextphi$ através da solução do problema \eqref{eq:pop-subproblem}, dado por $(\al^*,\mu^*,\sig^*)$ e portanto	$\nextphi(\al^*,\mu^*,\sig^*)\leq \nextphi(\hat{\al},\nextmu,\nextsig)$, para  $\hat{\al}$ definido abaixo.


Com essas alterações, podemos rescrever o Algoritmo~\ref{alg:optimized-choice-of-parameters}, como o  Algoritmo Simplificado \ref{alg:optimized-choice-of-parameters-simplified}.


A fim de pode  resolver o o problema primal-dual (\ref{eq:primal}-\ref{eq:dual}), vamos considerar válido o Pressuposto \ref{ass:interior-nonempty}, isto é, que o interior da região factível é não vazio. Além disso, vamos considerar que o \ac{PL} em questão possui ao menos uma solução ótima. 





Pelo Teorema \ref{thm:varphi}, a função de mérito para o próximo ponto   é, nas variáveis  $(\al,\mu,\sig)$,
\begin{equation*}
% \label{eq:merit-function-al-mu-sig}
{\nextphi}(\al,\mu,\sig) =  (1-\al)(\dbvec{\rho_L} +
\dbvec{\rho_C}) + \al\mu + \al(\al-\sig)\dbvec{L_{0,0}} +
\al^2\dbvec{\Lambda(\mu,\sig)} ,
\end{equation*}
em que 
\[
\dbvec{\Lambda(\mu,\sig)} = \mu^2
 \dbvec{L_{2,0}} + \mu \dbvec{L_{1,0}} + 	\mu \sig \dbvec{L_{1,1}} +
 \sig^2 \dbvec{L_{0,2}} + \sig \dbvec{L_{0,1}}.
 \]


Como vimos na Observação \ref{obs:L_02-L20},  $\dbvec{L_{2,0}} = \dbvec{L_{0,2}} = 0$. Fazendo $\mu = \nextmu = \eta\frac{x^Tz}{n} = \eta\dbvec{\rho_C} $
 e $\sig = \nextsig=0$ e usando a Proposição \ref{prop:nu_k},   podemos rescrever a função de mérito para o próximo ponto, dependendo apenas de uma escolha de  $\al$ como
\begin{equation}
	\label{eq:simplified-merit-function-al}
\nextphi(\al)  = (1-\al)(\nu\dbvec{\rho_L}_0 +
\dbvec{\rho_C}) + \al\eta\dbvec{\rho_C} + \al^2\left(\dbvec{L_{0,0}} + \eta\dbvec{\rho_C} \dbvec{L_{1,0}}
\right) .
\end{equation}

Seja a função auxiliar  dada por
\begin{equation}
	\label{eq:theta}
\theta(\al) =  \dfrac{\al\left[ \nu\dbvec{\rho_L}_0 + (1-\eta)\dbvec{\rho_C} - \al\left(\dbvec{L_{0,0}} + \eta\dbvec{\rho_C} \dbvec{L_{1,0}}
\right) \right]}{\nu\dbvec{\rho_L}_0 +
\dbvec{\rho_C}}.
\end{equation}
Usando tal definição, podemos escrever a seguinte relação entre a função de mérito atual -- $\varphi$ --  e a próxima -- $\nextphi$ --: 
\begin{equation}
	\label{eq:relation-phi-next-phi}
	 			{\nextphi} = (1- \theta(\al))\varphi.
\end{equation}


Como é necessário garantir que  $\nextphi  $  seja não negativa, deve-se escolher um tamanho de passo  $\al_k$, em cada iteração $k$, tal que  $\theta_k = \theta(\al_k)<1$. Além disso, se existir um escalar $\theta>0$, tal que $\theta = \liminf (\theta_k)$, isto é, se a sequência $\{\theta_k\}$ for limitada inferiormente por um valor positivo,	 então a sequência $\{\varphi_k\}$, gerada pelo Algoritmo \ref{alg:optimized-choice-of-parameters-simplified}, converge para zero Q-linearmente~\cite{Ortega:2000vd}.




 Em vista disso, para garantirmos a convergência dos pontos gerados pelo Algoritmo para uma solução ótima $(\xstar,\ystar,\zstar)$ do problema de \ac{PL}, além de ser  crucial que exista $\hat{\al}>0$  tal que,  para toda iteração $k$, tenha-se  $\al_k\in(0,\hat{\al}]$, de modo que a equação~\eqref{eq:relation-phi-next-phi} seja válida, é preciso que todo   ponto  $({x}^{k} ,{y}^{k},{z}^{k})$ pertença à vizinhança $\Nset_{-\infty}(\gamma,\beta)$.

Para tanto, sejam  
\begin{equation}
	\label{eq:al-C+al-L}
\begin{cases}
	\hat{\al}_{C}^{i} = \displaystyle \max_{\al\in (0,1]} \{ \al : g_{C}^{i}(\upsilon,\nextmu,\nextsig)\geq0 \text{ para todo } 0 \leq \upsilon\leq \al  \}, & i = 1,\ldots,n,\\
	\hat{\al}_{C} = \displaystyle \min_{1\leq i \leq n} \{\hat{\al}_{C}^{i}\}, \\
	\hat{\al}_{L} = \displaystyle \max_{\al\in (0,1]} \{ \al : g_{L}(\upsilon,\nextmu,\nextsig)\geq0 \text{ para todo } 0 \leq  \upsilon\leq \al  \},
\end{cases}
\end{equation}
em que $g_{C}^{i}$ e $g_{L}$ são dadas pelas equações  \eqref{eq:g-Ci_explicit} e \eqref{eq:g-L_explicit}, já tendo sido fixados os parâmetros $\mu=\nextmu$ e $\sig=\nextsig$.


Com isso, escolheremos  $\hat\al$ que garanta decréscimo suficiente de $\nextphi$ ao mesmo em que haja a pertinência de $({x}^{k} ,{y}^{k},{z}^{k})$ à $\Nset_{-\infty}(\gamma,\beta)$ por sujeitar $\nextphi$ tanto à  \eqref{eq:g-Ci_explicit} quanto à \eqref{eq:g-L_explicit}. Isso será feito simplesmente por escolher

\begin{equation}
	\label{eq:nex-al}
\nextal = \min\{\nextal_{C},\nextal_{L}\},
\end{equation}
em que $\nextal_{C}$ e $\nextal_{L}$ são definidos nos Lemas \ref{lemma:alC_delta-1} e  \ref{lemma:alL_delta-2}.

% -- neste caso,  em que $\mu$ e $\sig$ estão fixados --  por escolher o tamanho de passo $\nextal$ como
% \begin{equation}
% 	\label{eq:nex-al}
% \nextal = \argmax \left\{ \theta(\al): \al\in\left[0,\min\{\nextal_{C},\nextal_{L}\}\right]\right\}.
% \end{equation}
  


% De fato, a função $\theta(\al)$ é uma quadrática, e por conta disso, o problema \eqref{eq:nex-al} é resolvido por comparar os valores funcionais de $\theta(\al)$ em seus pontos críticos no interior do intervalo e o valor funcional no ponto mais a direita do intervalo.


 Essa escolha garante que  o Algoritmo \ref{alg:optimized-choice-of-parameters-simplified} gere uma sequência $\{\varphi_k\}$ que decresce de maneira suficiente a cada passo, tal que $\varphi_k \to 0$, quando $k\to \infty$. Assim, garantimos que o Algoritmo gera pontos $({x}^{k} ,{y}^{k},{z}^{k})$ que tem as boas propriedades da vizinhança $\Nset_{-\infty}(\gamma,\beta)$ e ainda que $({x}^{k} ,{y}^{k},{z}^{k})$ convirga para  alguma solução ótima $(\xstar ,\ystar,\zstar)$ do problema primal-dual  (\ref{eq:primal}-\ref{eq:dual}).




% \section{Ponto Inicial}


\section{Resultados técnicos}



A fim de demonstrarmos a convergência e a complexidade do Algoritmo proposto, na presente seção provaremos alguns lemas técnicos que serão utilizados para estabelecer nossos principais resultados de convergência.



Primeirmente vamos  estabelecer um ponto inicial adequado para utilizarmos em nosso Algoritmo. Para tanto, tomaremos como hipótese que existe uma solução ótima $(\xstar,\ystar,\zstar)$  do problema (\ref{eq:primal}-\ref{eq:dual}) para a qual é válido  
\begin{equation}
	\label{eq:norm-xz-star}
	\norm{(\xstar,\zstar)}_{\infty} \leq \bdxzstar,
\end{equation}
em que 
 \[
 	\bdxzstar = \max\{\abs{A_{ij}},\abs{b_{i}},\abs{c_{j}}, \text{ para } 1\leq i\leq m \text{ e } 1\leq j \leq n  \}
 \]
é o maior valor absoluto de todos os dados de entrada do problema. 

%FIXME comentar a respeito dessa hipótese, depois dos resultados numéricos
Definimos nosso ponto inicial como

\begin{equation}
	\label{eq:initial-point}
	(\xzero ,\xzero,\zzero ) = (\bdxzstar e, 0, \bdxzstar e ).
\end{equation}
Naturalmente nesses termos, $\xstar_{i}\leq   \bdxzstar$, bem como $\zstar_{i}\leq \bdxzstar $, para $i=1,\ldots,n$ e por isso
\begin{equation}
	\label{eq:x0}
0\leq x^{0} - \xstar \leq \bdxzstar e \quad \text{e} \quad 0\leq \zzero  - \zstar \leq \bdxzstar e. 
\end{equation}


Além disso, faz-se necessário  escolher  $\gamma\in(0,1)$  e $\be\geq 1$ adequados, para construir a vizinhança $\Nset_{-\infty}(\ga,\be)$. Para o primeiro parâmetro, várias escolhas são possíveis como, por exemplo, a de 
\textcite{Colombo:2008ia} que utilizam $\ga = 1/10$. No entanto, vamos escolher $\ga$ tal que  
\[
\gamma \leq  \min\left\{ \frac{\displaystyle \min_{i}(\xzero_{i} \zzero_{i} )}{(\xzero )^T\zzero /n}, \frac{1}{10}\right\},
\]
o que garante que o ponto inicial satisfaz a desigualdade~\eqref{eq:symmetric-polynomials-b}.

Com relação à $\be$,  usando um raciocínio similar, para qualquer  $\be\geq 1$, o ponto inicial sempre satisfaz \eqref{eq:symmetric-polynomials-a}. Quanto maior for $\be$, mais acelerada é a redução  das médias dos resíduos lineares, dadas por $\dbvec{\rho_L}$, em relação à redução da média dos resíduos complementares $\dbvec{\rho_C}$. Para facilitar os raciocínios, do ponto de vista teórico, considere seja escolhido como $\be=1$.


\begin{obs}
	Em vários momentos da análise que segue vamos considerar que se $(\Dex,\Dey,\Dez)$ é uma terna de direções qualquer,  tal que 
\begin{equation}
	\label{eq:Ax0-ATyz0}
	A\Dex=0 \text{ e } A^T\Dey + \Dez = 0,
\end{equation}
então
\begin{equation}
	\label{eq:xTziszero}
	\Dex^T\Dez =  - \Dex^T(A^T\Dey)  = -(A\Dex)^T\Dey = 0.
\end{equation}
\end{obs}



Esta análise é  baseada naquelas feitas por \textcite[cap. 6]{Wright:Primal-dual-interior-point:1997h},  \textcite{Zhang:2006ic,Zhang:1995fu}. 

O primeiro lema importante mostra a existência de um limitante para $\nu_k\norm{(\xk,\zk)}_{1}$. Sua demonstração pode ser encontrada em~\cite[Lema 6.1]{Wright:Primal-dual-interior-point:1997h}, fonte da qual a transcrevemos com as adaptações pertinentes à nossa notação. Ademais, na demonstração deste e dos outros resultados, omitiremos o índice da iteração $k$ dos vetores, para fins de clareza. 


% \subsection{Limitante para \texorpdfstring{$\nu_k\norm{(\xk,\zk)}_{1}$}{a norma dos iterandos}}


\begin{lema}\label{lemma:boundxz1}
	Suponha que o ponto inicial seja dado por \eqref{eq:initial-point}. Então para qualquer iterado $(\xk,\yk,\zk)$,  temos que 
	\begin{equation}
		\label{eq:bound-xkzk}
		\bdxzstar\nu_k\norm{(\xk,\zk)}_{1} \leq  4\be n \dbvec{\rho_C^{k}}
	\end{equation}
\end{lema}

\begin{proof}
	Seja o ponto auxiliar
	\[
		(\tilde{x},\tilde{y},\tilde{z})  = \nu(\xzero,\yzero,\zzero) + (1-\nu)(\xstar,\ystar,\zstar) - (x,y,z). 
	\]
em que $(\xzero,\yzero,\zzero)$ é ponto inicial dado por \eqref{eq:initial-point} e $(\xstar,\ystar,\zstar)$ é  solução ótima que satisfaz \eqref{eq:norm-xz-star}.
É fácil ver que tal ponto satisfaz \eqref{eq:Ax0-ATyz0} e portanto vale \eqref{eq:xTziszero}, isto é, $(\tilde{x})^{T}\tilde{z} = 0$. 

Logo temos que 
\begin{align}
	0  = \tilde{x}^{T}\tilde{z} & = (\nu\xzero + (1-\nu)\xstar - x)^{T}(\nu\zzero + (1-\nu_k)\zstar - x)\notag\\
	  & = \nu^{2}(\xzero)^T\zzero + (1-\nu)^{2}(\xstar)^{T}\zstar + \nu(1-\nu)((\xzero)^{T}\zstar + (\xstar)^{T}\zzero) \notag\\
	  & \textcolor{white}{=} + x^{T}z - \nu (x^{T}\zzero + (\xzero)^{T}z) - (1-\nu) (x^{T}\zstar + (\xstar)^{T}z).\label{eq:boundxz1}
\end{align}

Como todas as componentes de $(x,z)$ e $(\xstar,\zstar)$ são não-negativas, vale $(x^{T}\zstar + (\xstar)^{T}z) \geq 0$. Além disso, $(\xstar,\ystar,\zstar)$ é uma solução ótima, e por isso $(\xstar)^{T}\zstar = 0$. Usando essas observações e levando em conta que $\nu\in(0,1)$,  pode-se reorganizar  a equação \eqref{eq:boundxz1} como
\begin{equation}
	\label{eq:xTzero+xzeroTz}
	  \nu (x^{T}\zzero + (\xzero)^{T}z) \leq 
  \nu^{2}(\xzero)^T\zzero  +  x^{T}z+  \nu(1-\nu)((\xzero)^{T}\zstar + (\xstar)^{T}\zzero).
\end{equation}

Note agora que 
\begin{equation}
\label{eq:normxz1}
	x^{T}\zzero + (\xzero)^{T}z = \sum_{i=1}^{n} \bdxzstar\abs{x_i} +\sum_{i=1}^{n} \bdxzstar\abs{z_i} = \bdxzstar\norm{x}_{1} + 
	\bdxzstar\norm{z}_{1} = \bdxzstar\norm{(x,z)}_{1},
\end{equation}
e que
\begin{equation}
\label{eq:norm-star-zero-infty-1}
	(\xzero)^{T}\zstar + (\xstar)^{T}\zzero  = \norm{\xzero}_{\infty}\norm{\zstar}_{1} + \norm{\zzero}_{\infty}\norm{\xstar}_{1}  = \norm{(\xzero,\zzero)}_{\infty}\norm{(\xstar,\zstar)}_{1}.
\end{equation}


Além disso, pela escolha do ponto inicial, temos que 
\begin{gather*}
	\norm{(\xzero,\zzero)}_\infty = \bdxzstar \label{eq:initial-conseq-a}\\
	\norm{(\xstar,\zstar)}_1\leq 2n\norm{(\xstar,\zstar)}_{\infty} \leq 2n \bdxzstar \label{eq:initial-conseq-b}\\
	(\xzero)^{T}\zzero = n\dbvec{\rho_C^0} = n\bdxzstar^{2}\label{eq:initial-conseq-c}
\end{gather*}

Substituindo esses valores em \eqref{eq:xTzero+xzeroTz}, utilizando \eqref{eq:normxz1}, \eqref{eq:norm-star-zero-infty-1} e o fato de que $(x,y,z)$ pertence à vizinhança $\Nset_{\infty}(\ga,\be)$ obtemos
\begin{align*}
	\bdxzstar\nu\norm{(x,z)}_{1} &\leq \nu^{2}n\dbvec{\rho_C^0} + n\dbvec{\rho_C} + \nu(1-\nu)(\bdxzstar2n\bdxzstar) \\ 
								&\leq \nu n\dbvec{\rho_C^0}+ n\dbvec{\rho_C} + \nu(2n\bdxzstar^{2}) \\ 
								&\leq \be n\dbvec{\rho_C}+ n\dbvec{\rho_C} + \be\left({\dbvec{\rho_C}}/{\dbvec{\rho_C^{0}}}\right)(2n\bdxzstar^{2}) \\	
								&\leq 4\be n \dbvec{\rho_C},									
\end{align*}
conforme queríamos demonstrar.
\end{proof}





\subsection{Limitantes paras normas de direções} 

A próxima proposição será utilizada nas demonstrações que seguem e é um fato conhecido e bastante utilizado em \ac{MPI}.

\begin{prop}\label{prop:norm-uv}
	Seja $D$ uma matriz diagonal não-singular de ordem $n$ e sejam $u,v\in\Real^n$, então 
	\begin{equation}
		\label{eq:prop-uvDuDv}
		\norm{uv}\leq \norm{uv}_1 \leq \norm{Du}\norm{D^{-1}v} \leq \frac{1}{2}\left(\norm{Du}^{2} + \norm{D^{-1}v}^{2}\right)
	\end{equation}
\end{prop}
\begin{proof} A primeira desiguldade é valida por conta da equivalência de normas~\cite{Golub:1996wp}. Quanto às demais, vejamos que 
	\begin{align*}
		\norm{uv}_1^{2} & = \left( \sum_{i=1}^{n}u_i v_i  \right)^{2} \leq  \sum_{i=1}^{n}\abs{u_iv_i}^{2}\\
						& = \norm{uv}^{2} = (uv)^{T}(uv) \\
						& = v^{T}D^{-T}D^{T}u^{T}DD^{-1} uv\\
						& = v^{T}D^{-T}D^{T}u^{T}DuD^{-1}v\\
						& = \norm{Du}^{2}\norm{D^{-1}v}^{2}.
	\end{align*}
e logo a segunda desigualdade de \eqref{eq:prop-uvDuDv} está provada. 

Com relação a terceira desigualdade, note que  
\[
0 \leq \left(\norm{Du}^{1} - \norm{D^{-1}v}^{2}\right)^{2} = \norm{Du}^{2} - 2\norm{Du}\norm{D^{-1}v} + \norm{D^{-1}v}^{2}
\]
e portanto temos a validade de
\[
\norm{Du}\norm{D^{-1}v} \leq \frac{1}{2}\left(\norm{Du}^{1} + \norm{D^{-1}v}^{2}\right). \qedhere
\]
\end{proof}

De agora em diante, definimos a matriz $D$ como sendo
\[
D = X^{-1/2}Z^{1/2}.
\]


\begin{lema}\label{lemma:boundDxDzaff}
	Suponha um ponto inicial escolhido como em \eqref{eq:initial-point}. Então existe uma constante $\omega_1>1$ e  independente de $n$ tal que 
	\begin{equation}
		\norm{D^{k} (\dex)^{k} }\leq \omega_1 n\dbvec{\rho_C^{k}}^{1/2}\text{ e } \quad  \norm{(D^{k})^{-1}(\dez)^{k}}\leq \omega_1 n\dbvec{\rho_C^{k}}^{1/2}.
	\end{equation}
\end{lema}
\begin{proof}
	Seja o ponto auxiliar
	\[
		(\tilde{x},\tilde{y},\tilde{z})  = (\dex,\dey,\dez) + \nu(\xzero - \xstar,\yzero - \ystar,\zzero - \zstar), 
	\]
em que $(\xzero,\yzero,\zzero)$ é ponto inicial dado por \eqref{eq:initial-point} e $(\xstar,\ystar,\zstar)$ é  solução ótima que satisfaz \eqref{eq:norm-xz-star}.
	Também é fácil  ver que tal ponto satisfaz \eqref{eq:Ax0-ATyz0} e logo 
	\begin{equation}
		\label{eq:DxDzaff1}
			 0 = \tilde{x}^{T}\tilde{z} = \left(\dex +  \nu(\xzero - \xstar)\right)^{T}\left(\dez +  \nu(\zzero - \zstar)\right).
	\end{equation}
	

	 Usando este ponto auxiliar na equação \eqref{eq:affine-scaling-system-compl}, obtemos 
	 \[
	 Z(\left(\dex +  \nu(\xzero - \xstar)\right)+ X \left(\dez +  \nu(\zzero - \zstar)\right) = -xz + \nu Z(\xzero - \xstar) + \nu X(\zzero - \zstar).
	 \]
	 Multiplicando toda essa expressão por $(XZ)^{-1/2}$, e notando que $D=(XZ)^{-1/2}Z$ e que $D^{-1}=(XZ)^{-1/2}X$, resulta em 
	\begin{multline}
		\label{eq:DxDzaff2}
		 D\left(\dex +  \nu(\xzero - \xstar)\right)+ D^{-1} \left(\dez +  \nu(\zzero - \zstar)\right) = \\ -(xz)^{1/2} + \nu D(\xzero - \xstar) + \nu D^{-1}(\zzero - \zstar).
		\end{multline}
	 
	 Note que, como vale  \eqref{eq:DxDzaff1}, pode-se utilizar o Teorema de Pitágoras para norma-2 de vetores e logo, tomando a norma-2 ao quadrado do lado esquerdo de \eqref{eq:DxDzaff2} obtemos
	\begin{multline*}
			 	 \norm{D\left(\dex +  \nu(\xzero - \xstar)\right)+ D^{-1} \left(\dez +  \nu(\zzero - \zstar)\right)}^{2} =\\
	 	 	 	 	 \norm{D\left(\dex +  \nu(\xzero - \xstar)\right)}^{2}+ \norm{D^{-1} \left(\dez +  \nu(\zzero - \zstar)\right)}^{2}.
	\end{multline*}
	 Se  usarmos este resultado e a desigualdade triangular, após tomarmos a norma-2 ao quadrado de ambos os lados de \eqref{eq:DxDzaff2}, obtemos 
	 \begin{multline}\label{eq:DxDzaff3}
	 \norm{D\left(\dex +  \nu(\xzero - \xstar)\right)}^{2}+ \norm{D^{-1} \left(\dez +  \nu(\zzero - \zstar)\right)}^{2} \leq \\ 
	 \left\{ \norm{(xz)^{1/2}} + \nu \norm{D(\xzero - \xstar)} + \nu \norm{D^{-1}(\zzero - \zstar)} \right\}^{2}.
	 \end{multline}

Isolando o primeiro termo dessa equação, obtemos
\[
\norm{D\left(\dex +  \nu(\xzero - \xstar)\right)} \leq  
	  \norm{(xz)^{1/2}} + \nu \norm{D(\xzero - \xstar)} + \nu \norm{D^{-1}(\zzero - \zstar)}.
\]
Uma aplicação direta da desigualdade triangular e a adição de um termo  $\nu \norm{D^{-1}(\zzero - \zstar)}$ extra resulta em
\begin{align}
	\norm{D\dex} & =  \norm{D\left(\dex +  \nu(\xzero - \xstar)  -  \nu(\xzero - \xstar)\right)} \notag \\ 
				 & \leq	 \norm{D\left(\dex +  \nu(\xzero - \xstar)\right)} +  \nu\norm{D(\xzero - \xstar)} \notag \\
				 & \leq \norm{(xz)^{1/2}} + 2 \nu \norm{D(\xzero - \xstar)} + 2\nu \norm{D^{-1}(\zzero - \zstar)}.\label{eq:DxDzaff4}
\end{align}

Vamos mostrar a existência de um limitante para cada termo do lado direito de \eqref{eq:DxDzaff4} de magnitude $\Oset(\dbvec{\rho_C}^{1/2})$. Neste caso, o mesmo se aplicará à $\norm{D^{-1}\dez}$ o que finalizará esta demonstração.

Para o primeiro termo, note que 
\begin{equation}
	\label{eq:DxDzaff5}
	\norm{(xz)^{1/2}} = \left(\sum_{i=1}^{n}x_iz_i\right)^{1/2} = (x^{T}z)^{1/2} = n^{1/2}\dbvec{\rho_C}^{1/2} \leq \frac{n}{\ga^{1/2}}\dbvec{\rho_C}^{1/2},
\end{equation}
já que $\ga\in(0,1)$ e $\sqrt{n}\leq n$, para todo $n$ natural.

Para os últimos dois termos, considere primeiramente que pela escolha do ponto inicial, vale \eqref{eq:x0} e por conta disso, 
\[
\norm{\xzero - \xstar} \leq \bdxzstar \text{ bem como } \norm{\zzero - \zstar} \leq \bdxzstar.
\]
Por outro lado, a norma-2 da matriz  $D$ será
\[
 \norm{D} = \max_{i=1,\ldots,n} \abs{D_{ii}} = \norm{De}_{\infty} = \norm{(XZ)^{-1/2}z}_{\infty} \leq \norm{(XZ)^{-1/2}}\norm{z}_{1},
\]
e similarmente
\[
\norm{D^{-1}} \leq  \norm{(XZ)^{-1/2}} \norm{x}_{1}.
\]

Mais que isso, como $(x,y,z)\in\Nset_{-\infty}(\ga,\be)$, temos que 
\begin{equation}
\label{eq:DxDzaff6}
	\norm{(XZ)^{-1/2}} = \max_{i=1,\ldots,n}\frac{1}{(x_iz_i)^{1/2}} \leq \frac{1}{\ga^{1/2}\dbvec{\rho_{C}}^{1/2}}.
\end{equation}

Com essas desigualdades para norma de $D$ e de $D^{-1}$, utilizando consistência de norma de matrizes, segue 
\begin{align*}
 \nu \norm{D(\xzero - \xstar)} + \nu \norm{D^{-1}(\zzero - \zstar)}  & \leq \nu \norm{D}\norm{\xzero - \xstar} + \nu \norm{D^{-1}}\norm{\zzero - \zstar} \notag \\
 					& \leq \nu\bdxzstar\left[  \norm{D} + \norm{D^{-1}}   \right] \notag \\ 
 					& \leq \nu\bdxzstar\norm{(x,z)}_{1}\norm{(XZ)^{-1/2}} .\notag
\end{align*}

Da Equação~\eqref{eq:DxDzaff6}, do Lema~\ref{lemma:boundxz1} e da desigualdade acima, segue que 
\begin{equation} \label{eq:DxDzaff7}
 \nu \norm{D(\xzero - \xstar)} + \nu \norm{D^{-1}(\zzero - \zstar)}  \leq 4\be n \dbvec{\rho_C} \frac{1}{\ga^{1/2}\dbvec{\rho_{C}}^{1/2}} = \frac{4\be}{\ga^{1/2}}n\dbvec{\rho_C}^{1/2}.
\end{equation}

O resultado é obtido por observar \eqref{eq:DxDzaff5} e \eqref{eq:DxDzaff7}, comparar com \eqref{eq:DxDzaff4} e escolher 
\begin{equation}
	\label{eq:omega1}
	\omega_1 = \frac{9\be}{\ga^{1/2}},
\end{equation}
Como $\be\geq1$ e $\ga\in(0,1)$, temos que $\omega_1>1$ como requisitado.
\end{proof}

Um corolário  das desigualdades dadas pela Proposição~\ref{prop:norm-uv} e do Lema~\ref{lemma:boundDxDzaff} acima é que 
\begin{multline}
\label{eq:normDxDzaff}
	\norm{(\dex)^{k}(\dez)^{k}} \leq \norm{(\dex)^{k}(\dez)^{k}}_{1}  \\ \leq \frac{1}{2}\left(\norm{D^{k}(\dex)^{k}}^{2} + \norm{(D^{-1})^{k}(\dez)^{k}}^{2}\right)
  					 \leq \left(\omega_1 n\dbvec{\rho_C^{k}}^{1/2}\right)^{2}
  					 = \omega_1^{2} n^{2}\dbvec{\rho_C^{k}}.
\end{multline}

\begin{lema}\label{lemma:boundDxDzc}
	Suponha um ponto inicial escolhido como em \eqref{eq:initial-point} e que \begin{equation}
	\label{eq:sig-eta-relation}
	\nextsig \geq \frac{\sqrt{\ga} - \eta}{\omega_1^{2}}.
\end{equation} Então existe uma constante $\omega_2\geq1$  independente de $n$ tal que 
	\begin{equation}\label{eq:lemma-boundDxDzcx}
		\norm{D^{k}(\Decox)^{k}}^{2} + \norm{(D^{k})^{-1}(\Decoz)^{k}}^{2} \leq \omega_2 n^{4}\dbvec{\rho_C^{k}}.
	\end{equation}
\end{lema}



\begin{proof}
	Note que por conta de \eqref{eq:linear-sytem-DeXDeZcorrec}, temos que $(\Decox,\Decoy,\Decoz)$ satisfaz \eqref{eq:Ax0-ATyz0} e logo $(\Decox)^{T}\Decoz = 0$. 

	


	Além disso, também por causa de \eqref{eq:linear-sytem-DeXDeZcorrec}, temos que
	\[
	z\Decox + x\Decoz = \mu e  - \sig\dex\dez.
	\]

	Multiplicando toda equação anterior por $(xz)^{-1/2}$ e  fixando $\mu = \nextmu$ e $\sig = \nextsig $, segue que 
	\[
		D\Decox + D^{-1}\Decoz = (xz)^{-1/2}(\nextmu e + \nextsig\dex\dez).
	\] 

	Por um lado temos que 
	\[
		\norm{D\Decox + D^{-1}\Decoz}^{2} = \norm{D\Decox}^{2} + \norm{D^{-1}\Decoz}^{2} + 2(\Decox)^{T}\Decoz = \norm{D\Decox}^{2} + \norm{D^{-1}\Decoz}^{2}.
	\]

Assim, usando consistência de normas e o fato de que  $(x,y,z)\in\Nset_{-\infty}(\ga,\be))$ obtemos
\begin{align}
	\norm{D\Decox}^{2} + \norm{D^{-1}\Decoz}^{2} & = \norm{(xz)^{-1/2}(\nextmu e+ \nextsig\dex\dez)}^{2} \notag\\ 
												& \leq \norm{(xz)^{-1/2}}^{2}\norm{(\nextmu e+ \nextsig\dex\dez)}^{2} \notag\\ 
												& \leq \min\left({x_iz_i}\right)^{-1}\left(\nextmu\norm{ e} +{\nextsig}\norm{\dex\dez}\right)^{2}\notag\\
												& \leq (\ga\dbvec{\rho_C})^{-1}\left(\nextmu\sqrt{n} +{\nextsig}\norm{\dex\dez}\right)^{2} \label{eq:DxDzc1}		
\end{align}



Utilizando as desigualdades de \eqref{eq:normDxDzaff}, fixando $\mu = \nextmu =\eta \dbvec{\rho_C}$ como anteriormente e notando que para todo $n$ natural vake $\sqrt{n} \leq n^{2}$, o lado esquerdo de~\eqref{eq:DxDzc1} é tal que
\begin{align*}
	% (\ga\dbvec{\rho_C})^{-1}\left(\nextmu\sqrt{n} +\abs{\nextsig}\norm{\dex\dez}\right)^{2}  
	\norm{D\Decox}^{2} + \norm{D^{-1}\Decoz}^{2}
							& \leq (\ga\dbvec{\rho_C})^{-1}\left(\eta\dbvec{\rho_C}\sqrt{n} +{\nextsig} \omega_1^{2} n^{2}\dbvec{\rho_C}\right)^{2}\\
							& \leq (\ga)^{-1}\left(\eta\sqrt{n} +{\nextsig} \omega_1^{2} n^{2}\right)^{2}\dbvec{\rho_C}\\
							& \leq (\ga)^{-1}\left(\eta +{\nextsig} \omega_1^{2} \right)^{2}n^{4}\dbvec{\rho_C}.
\end{align*}

Seja
\begin{equation}
\label{eq:omega2}
	\omega_2 = \frac{\left(\eta +{\nextsig} \omega_1^{2} \right)^{2}}{\ga}.
	\end{equation}
Observando que  $\omega_1 \geq 1$ e  $\ga\in(0,1)$,  se $\eta\in[0,1]$ e se \eqref{eq:sig-eta-relation} for satisfeita, então garante-se que $\omega_2\geq 1$, como queríamos. \qedhere
% \begin{equation}
% 	\label{eq:sig-eta-relation}
% 	\nextsig \geq \frac{\sqrt{\ga} - \eta}{\omega_1^{2}},
% \end{equation}




% Wolphran Alpha code
% solve{ (1/gamma)*(eta +sigma*omega^2)^2>=1 , gamma<1, omega>=1, sigma}
%with abs(sigma)
%solve{ (1/gamma)*(eta +abs(sigma)*omega^2)^2>=1 , gamma<1, omega>=1, sigma}
%sigma>0
% solve{ (1/gamma)*(eta +(sigma)*omega^2)^2>=1 , gamma<1, omega>=1, sigma>0, sigma}

\end{proof}

\begin{lema}\label{lemma:boundDxDzaff-c}
	Suponha um ponto inicial escolhido como em \eqref{eq:initial-point}. Então 
	\begin{equation}\label{eq:lemma-boundDxDzaff-c}
		\norm{(\dex)^{k}(\Decoz)^{k} + (\Decox)^{k}(\dez)^{k}}_{1} \leq  \tfrac{1}{2}\omega_{3}n^{3}\dbvec{\rho_C^{k}},
	\end{equation}
em que $\omega_{3} = 4(\omega_1\omega_2^{1/2})\geq 4 $.
\end{lema}

\begin{proof}
	Da equação \eqref{eq:prop-uvDuDv} e dos Lemas~\ref{lemma:boundDxDzaff} e \ref{lemma:boundDxDzc} segue que  
	\begin{align*}	
		\norm{\dex\Decoz}_{1} & \leq \norm{D\dex}\norm{D^{-1}\Decoz}\\
							  & \leq 	\omega_1 n\dbvec{\rho_C}^{1/2}(\omega_2 n^{4}\dbvec{\rho_C})^{1/2}  \\
							  & = (\omega_1\omega_2^{1/2})n^{3}\dbvec{\rho_C}.
	\end{align*}			  
Similarmente, 	$\norm{\Decox\dez}_{1} \leq (\omega_1\omega_2^{1/2})n^{3}\dbvec{\rho_C}$. Com isso temos o resultado desejado, isto é,
\[
		\norm{\dex\Decoz + \Decox\dez}_{1} \leq 2(\omega_1\omega_2^{1/2})n^{3}\dbvec{\rho_C}.\qedhere
\]
\end{proof}

\section{Teorema de Convergência}

A fim de que o Algoritmo~\ref{alg:optimized-choice-of-parameters-simplified} esteja bem definido, é necessário que exista para cada iteração $k$ uma tripla $(\al_k,\mu_k,\sig_k)$, de modo que seja possível encontrar um próximo ponto $(\nextx,\nexty,\nextz)$. Com efeito, considerando que fixamos os valores de $\mu$ e $\sig$ como $\nextmu $ e  $\nextsig$, basta encontrar um o tamanho de passo ${\al_k}>0$ tal que o próximo ponto $(\nextx,\nexty,\nextz)$ satisfaça as restrições da vizinhança 
$\Nset_{-\infty}(\gamma,\beta)$ e além disso, garanta que  $0 < \theta(\al_k) <1$. É o que os próximos resultados garantem. Observe que, por conta de \eqref{eq:sig-eta-relation}, garantiremos que $\eta>\sqrt{\ga}$.



\begin{lema}\label{lemma:alC_delta-1}
Seja $\nextal_{C}$ dado em \eqref{eq:al-C+al-L}. Então 
\[
\nextal_{C} \geq \delta_{1}/n^{4}
\]
em que 
\[
\delta_{1} = \dfrac{(1-\ga)\eta}{2\om_{1}^{2} + \om_{2}/2 + \om_{3}  } < 1.
\]
\end{lema}

\begin{proof}
 Primeiramente, considere que  utilizando a Equação \eqref{eq:simplified-merit-function-al}, é possível rescrever a função $g_C^i $, para $i=1,\ldots,n$, que foi dada em \eqref{eq:g-Ci_explicit}, somente dependendo de uma escolha de $\al$. De fato, tal função pode ser escrita escrita nos seguintes termos
\[
\begin{aligned}
{g}_C^i (\al)				& = (1-\al)(\rho_C)_i+ \al\eta\dbvec{\rho_C}+ \al^2\left[(L_{0,0})_i + \eta\dbvec{\rho_C} ({L_{1,0}})_i 
				+ (\eta\dbvec{\rho_C})^2(L_{2,0})_i \right]  + \\
				& \quad -\ga\left[  (1-\al)\dbvec{\rho_C} + \al\eta\dbvec{\rho_C} + \al^2\left(\dbvec{L_{0,0}} + \eta\dbvec{\rho_C} \dbvec{L_{1,0}}.
\right)  \right]
\end{aligned}
\]

Sejam as constantes
\begin{equation}
\label{eq:defin-zeta-i+chi-i+xi-i}
	\begin{aligned}
\zeta_i & = (L_{0,0})_i - \ga \dbvec{L_{0,0}}, \\
\chi_i  & = \eta\dbvec{\rho_C} \left( ({L_{1,0}})_i - \ga\dbvec{L_{1,0}} \right),   \\
\xi_i	& =  (\eta\dbvec{\rho_C})^2(L_{2,0})_i . \\ 	
\end{aligned}
\end{equation}



Usando o fato de que o ponto atual pertence à vizinhança $\Nset_{-\infty}(\gamma,\beta)$, e as definições da Equação~\eqref{eq:defin-zeta-i+chi-i+xi-i} reescreve-se  $g_C^i(\al) $ como
\[
\begin{aligned}
	g_C^i (\al) & = \underbrace{(1-\al)((\rho_C)_i - \ga\dbvec{\rho_C})}_{\geq 0}  + (1-\ga)\eta\dbvec{\rho_C} \al+  (\zeta_i + 				\chi_i + \xi_i)\al^2  \\
				& \geq (1-\ga)\eta\dbvec{\rho_C} \al +  (\zeta_i + 				\chi_i + \xi_i)\al^2 \\ 
				& \geq (1-\ga)\eta\dbvec{\rho_C} \al -  (\abs{\zeta_i} + \abs{\chi_i} + \abs{\xi_i})\al^2 \\
				& = \al \left[	(1-\ga)\eta\dbvec{\rho_C}  -  (\abs{\zeta_i} + \abs{\chi_i} + \abs{\xi_i})\al	\right] = h^i(\al),
\end{aligned}
\]
em que $h^i$ é uma quadrática côncava em função de $\al$ com uma raiz nula e uma positiva. 

Considere, para as substituições que seguem, as definições dos vetores $L_{i,j}$ dadas em~\eqref{eq:defining-Lij}. Neste caso, da Equação~\eqref{eq:normDxDzaff}  vem que  
\begin{align}
\abs{\zeta_i} & = \abs{(L_{0,0})_i - \ga \dbvec{L_{0,0}}} \leq \abs{(L_{0,0})_i} + \ga \abs{\dbvec{L_{0,0}}} \notag \\
			  & \leq	\norm{L_{0,0}}_1  + \dfrac{\ga}{n}\norm{L_{0,0}}_1   \leq 2\norm{L_{0,0}}_1 \notag \\
			  & = 2\norm{\dex\dez}_1   \leq 2\omega_1^{2} n^{2}\dbvec{\rho_C}. \label{eq:bound-zeta-i}
\end{align}
Como $\sig = \nextsig=0$, usando a equação \eqref{eq:Corrector-spllited} e sabendo que $\mu=\nextmu = \eta\dbvec{\rho_{C}}$ fixado, temos que 
	\[
		\eta\dbvec{\rho_{C}}\Dex^\mu =  \nextmu\Dex^\mu = \Decox   \quad \text{ e } \quad \eta\dbvec{\rho_{C}}\Dez^\mu=  \nextmu\Dez^\mu=  \Decoz .
	\]
	Assim, do Lema~\ref{lemma:boundDxDzaff-c} segue que
\begin{align}
	\abs{\chi_i }  	& = \abs{\eta\dbvec{\rho_C} \left( ({L_{1,0}})_i - \ga\dbvec{L_{1,0}} \right)} \leq \abs{\eta\dbvec{\rho_C}  ({L_{1,0}})_i} + \ga\abs { \eta\dbvec{\rho_C}\dbvec{L_{1,0}}} 
					\notag \\
					&\leq \norm{\eta\dbvec{\rho_C}  L_{1,0}}_1 + \dfrac{\ga}{n}\norm{\eta\dbvec{\rho_C}  L_{1,0}}_1 \notag\\
					& \leq \norm{\dex\Decoz + \Decox\dez}_1 + \dfrac{\ga}{n}\norm{\dex\Decoz + \Decox\dez}_1 \notag\\
					& \leq 2 \norm{\dex\Decoz + \Decox\dez}_1  \leq \omega_{3}n^{3}\dbvec{\rho_C}. \label{eq:bound-chi-i}
\end{align}
Se utilizarmos  a equação \eqref{eq:Corrector-spllited}, novamente com as escolhas $\sig = \nextsig=0$ e $\mu=\nextmu = \eta\dbvec{\rho_{C}}$ temos
\[
	(\eta\dbvec{\rho_C})^2(L_{2,0}) = (\eta\dbvec{\rho_C}\Dex^\mu)(\eta\dbvec{\rho_C}\Dez^\mu) = \Decox\Decoz.
\]
Portanto, do Lema~\ref{lemma:boundDxDzc} e da Proposição~\ref{eq:prop-uvDuDv} vem que
\begin{equation}\label{eq:bound-xi-i}
	\abs{\xi_i}	 =  \abs{(\eta\dbvec{\rho_C})^2(L_{2,0})_i}  = \abs{(\Decox\Decoz)_i} \leq \norm{\Decox\Decoz}_1 \leq \tfrac{1}{2} \omega_2 n^{4}\dbvec{\rho_C}.
\end{equation}
	


Agora, note que, para $\al>0$ e $i=1,\ldots,n$, se $h^i(\al)\geq0$ então $g_C^i(\al)\geq 0$. 

Com efeito, a única raiz positiva de $h^i$ é dada por
\[
\al_{C}^i = \dfrac{(1-\ga)\eta\dbvec{\rho_C}}{\abs{\zeta_i} + \abs{\chi_i} + \abs{\xi_i}}
\]
e $h^i(\al)\geq 0$ sempre que $\al\in[0,\al_{C}^i].$ 

Como $\nextal^{i}_{C}$, dada em \eqref{eq:al-C+al-L}, é o maior número em $(0,1]$ tal que $g_{C}^{i}(\al)\geq 0$, para $\al\leq\nextal^{i}_{C}$, e além disso para todo $\al$ vale $g_{C}^{i}(\al) \geq h^{i}(\al)$, claramente $\nextal^{i}_{C} \geq  \al_{C}^i$ para todo $i=1,\ldots,n$.

Consequentemente, utilizando \eqref{eq:bound-zeta-i}, \eqref{eq:bound-xi-i} e \eqref{eq:bound-chi-i} temos que 
\[
\begin{aligned}
\nextal_{C}^i & \geq \left(\dfrac{(1-\ga)\eta}{2\om_{1}^{2}/n^{2} + \om_{2}/2 + \om_{3}/n  }\right)\dfrac{1}{n^{4}} \\			  & \geq \left(\dfrac{(1-\ga)\eta}{2\om_{1}^{2} + \om_{2}/2 + \om_{3}  }\right)\dfrac{1}{n^{4}},
\end{aligned}
\]
 para todo $i=1,\ldots,n$. 
 % O lema fica provado, considerando que para algum $j\in\{1,\ldots,n\}$, $ \al_{C}^{j}=\nextal_{C} $.
 \end{proof}






\begin{lema}\label{lemma:alL_delta-2}
Seja $\nextal_{L}$ dado em \eqref{eq:al-C+al-L}. Então 
\[
\nextal_{L} \geq \delta_{2}/n^{2}
\]
em que 
\[
\delta_{2} = \frac{\eta }{\om_{1}^{2} + \om_{3}/2} < 1.
\]
\end{lema}


\begin{proof}

Usando as mesmas substituições de $\nextmu$ e $\nextsig$, a função $g_L$ dada em \eqref{eq:g-L_explicit} torna-se uma função que depende apenas de $\al$, nos seguintes termos 
\[
g_L(\al) =     (1-\al)\left(\dbvec{\rho_C} -  \be_L \nu   \right) +  \al\eta\dbvec{\rho_C} + 
   \al^2\left( \dbvec{L_{0,0}} + \eta\dbvec{\rho_C}  \dbvec{L_{1,0}}   \right ) ,
	\]
Usando  o fato de que o ponto atual pertence à  vizinhança segue que 
 \[
\begin{aligned}
{g}_L(\al) & =     (1-\al)\underbrace{\left(\dbvec{\rho_C} -  \be_L \nu   \right)}_{\geq 0} +  \al\eta\dbvec{\rho_C} + 
   \al^2\left( \dbvec{L_{0,0}} + \eta\dbvec{\rho_C}  \dbvec{L_{1,0}}   \right ) \\
   & \geq  \al\left[\eta\dbvec{\rho_C} - 
   \al \left(\abs{\dbvec{L_{0,0}}} + \abs{\eta\dbvec{\rho_C}  \dbvec{L_{1,0}}}\right)   \right ].
\end{aligned}
 \]
Note que  $\nextal_L\in(0,1]$ -- definido em \eqref{eq:al-C+al-L} -- é o maior número tal que  $g_L(\al)\geq 0$, para  $\al\in[0,\nextal_L]$. Por conta da última inequação acima, temos que 
\begin{equation}
	\label{eq:next-al-ineq}
\nextal_L \geq \frac{\eta\dbvec{\rho_C} }{\abs{\dbvec{L_{0,0}}} + \abs{\eta\dbvec{\rho_C}  \dbvec{L_{1,0}}}}.
\end{equation}
Da Equação~\eqref{eq:normDxDzaff}  vem que  
\begin{align}
\abs{\dbvec{L_{0,0}}} &  = \abs{\dfrac{(\dex)^T\dez}{n}} \notag \\
			& \leq \dfrac{1}{n} \norm{\dex\dez}_1  \leq  \omega_1^{2} n\dbvec{\rho_C}\label{eq:bound-zeta}
\end{align}
Novamente, fazemos $\sig = \nextsig=0$ e  $\mu=\nextmu = \eta\dbvec{\rho_{C}}$. Usando a equação \eqref{eq:Corrector-spllited} temos que 
	\[
		\eta\dbvec{\rho_{C}}\Dex^\mu =  \nextmu\Dex^\mu = \Decox   \quad \text{ e } \quad \eta\dbvec{\rho_{C}}\Dez^\mu=  \nextmu\Dez^\mu=  \Decoz .
	\]
	Assim, do Lema~\ref{lemma:boundDxDzaff-c} segue que
	\begin{align}
	\abs{\eta\dbvec{\rho_C} \dbvec{L_{1,0}}} & = \abs{\dfrac{(\dex)^T(\nextmu\Dez^\mu) + (\nextmu\Dex^\mu)^T\dez}{n}   }  \notag \\ 
	&  = \abs{\dfrac{(\dex)^T(\Decoz) + (\Decox)^T\dez}{n} } \notag \\
	& \leq \dfrac{1}{n} \norm{\dex\Decoz + \Decox\dez}_1 \notag\\
	&\leq \tfrac{1}{2}\omega_{3}n^{2}\dbvec{\rho_C}.\label{eq:bound-chi}
	\end{align}

A prova do lema termina por juntar a Equação \eqref{eq:next-al-ineq} com \eqref{eq:bound-zeta} e \eqref{eq:bound-chi}, derivando 
\[
\nextal_L \geq \left( \frac{\eta }{\om_{1}^{2}/n + \om_{3}/2}\right)\dfrac{1}{n^{2}}. \qedhere
\]
\end{proof}


\begin{lema}\label{lemma:next-phi-delta-n4}
A sequência $\{(x^{k},y^{k},z^{k})\}$ gerada pelo Algoritmo~\ref{alg:optimized-choice-of-parameters-simplified} é tal que
\begin{equation}
		\label{eq:varphi-delta-n4}
				\varphi_{k+1}\leq \left(1 - \frac{\hat{\delta}}{n^{4}}\right)\varphi_{k},
	\end{equation}
	para todo $k$, em que 
	\begin{equation}
		\label{eq:nextdel}
		 \nextdel =(1 -  \eta)\delta_{1} - \delta_{1}^{2}\left(\omega_1^{2} + \tfrac{1}{2}\omega_{3} \right).
	\end{equation}
\end{lema}


\begin{proof} Por conta da Equação \eqref{eq:nex-al}, e dos Lemas \ref{lemma:alC_delta-1} e \ref{lemma:alL_delta-2}, podemos escolher, sem perda de generalidade, $\nextal = \delta_{1}/n^{4}$. Com isso, basta mostrar que $\theta(\nextal)= \Oset(\nextdel/n^{4})$. Com efeito, considerando que $\dfrac{\dbvec{\rho_C}}{\nu\dbvec{\rho_L}_0 + \dbvec{\rho_C}}\leq 1$ temos que


\[
	\begin{aligned}
		\theta(\nextal)% & =  \dfrac{1}{\nu\dbvec{\rho_L}_0 + \dbvec{\rho_C}}\left[ \nu\dbvec{\rho_L}_0\nextal + (1- \eta)\dbvec{\rho_C}\nextal - \nextal^{2}\left(\dbvec{L_{0,0}} + \eta\dbvec{\rho_C} \dbvec{L_{1,0}} \right) \right] \\
						& = \dfrac{1}{\nu\dbvec{\rho_L}_0 + \dbvec{\rho_C}}\left[ (\nu\dbvec{\rho_L}_0 + \dbvec{\rho_C})\nextal -  \eta\dbvec{\rho_C}\nextal - \nextal^{2}\left(\dbvec{L_{0,0}} + \eta\dbvec{\rho_C} \dbvec{L_{1,0}} \right) \right] \\
						& \geq \dfrac{1}{\nu\dbvec{\rho_L}_0 + \dbvec{\rho_C}}\left[ (\nu\dbvec{\rho_L}_0 + \dbvec{\rho_C})\nextal -  \eta\dbvec{\rho_C}\nextal - \nextal^{2}\left(\omega_1^{2} n\dbvec{\rho_C} + \tfrac{1}{2}\omega_{3}n^{2}\dbvec{\rho_C} \right) \right]\\
						& \geq \nextal - \dfrac{\dbvec{\rho_C}}{\nu\dbvec{\rho_L}_0 + \dbvec{\rho_C}} \left[   \eta\nextal + \nextal^{2}\left(\omega_1^{2} n + \tfrac{1}{2}\omega_{3}n^{2} \right) \right]\\
						& \geq \nextal \left[ 1 -  \eta - \nextal\left(\omega_1^{2} n + \tfrac{1}{2}\omega_{3}n^{2} \right) \right] \\
						& \geq \frac{\delta_{1}}{n^{4}} \left[ 1 -  \eta - \frac{\delta_{1}}{n^{4}}\left(\omega_1^{2} n + \tfrac{1}{2}\omega_{3}n^{2} \right) \right] \\
						& \geq 	 \left[ (1 -  \eta)\delta_{1} - \delta_{1}^{2}\left(\omega_1^{2} + \tfrac{1}{2}\omega_{3} \right) \right] \frac{1}{n^{4}}	\\
						& = \frac{\nextdel}{n^{4}},		
		\end{aligned}
	\]
	com 	$\nextdel$ dado na Equação \eqref{eq:nextdel}. A demonstração será finalizada ao mostrarmos que $0<\nextdel<1$.

	Com efeito, ao utilizarmos as equações \eqref{eq:omega1}, \eqref{eq:omega2} e o Lema \ref{lemma:boundDxDzaff-c}, encontramos o valor de $\nextdel$ a depender das constantes $\ga$ e $\eta$, ou seja, 
	\begin{equation}
		\label{eq:nextdel-full}
		\nextdel = \frac{(2\ga\eta)(1-\ga)(324 -	(414 - 162\ga)\eta - (107 - 36\ga)\eta^{2} -  \eta^{3})}{(324 + 72\eta + \eta^{2})^{2}}.
	\end{equation}
	O terceiro termo do numerador da equação acima é o polinômio 
	\[
		p(\eta,\ga) =   324 -	(414 - 162\ga)\eta - (107 - 36\ga)\eta^{2} -  \eta^{3}. 
	\]
	Considerando que  $\sqrt{\ga}<\eta<1$, para $0 < \ga \leq 1/10$, segue que
	\[p(\sqrt{1/10},\ga) >0 \text{ e  } p(1,\ga) <0.\] 

	Sob essas condições, pelo Teorema do Valor Intermediário de Bolzano \cite[Teorema 5.3.7]{Bartle:2011tr}, escolhido $\ga$, existe uma raiz $\eta_1$ do polinômio $p$, tal que  $\eta_{1}\in (1/10, 1)$. Neste caso, por continuidade, para qualquer valor de $\eta  < \eta_{1}$, $p(\eta,\ga) >0$.  Finalmente,  para $0 < \ga \leq 1/10$ e  $\sqrt{\ga}<\eta<1$, temos que  $0<\nextdel<1$  sempre que $\eta < \eta_{1} < 1$, o que completa a demonstração.
% \textcolor{red}{Como terminar? $\nextdel$ precisa ser menor que 1? Maior que 0? O que mais? Já sei que $\eta > \sqrt{\ga}$, para essa escolha de $\mu$ e $\sigma$.}
\end{proof}




% \begin{proof} 
% Por conta dos Lemas \ref{lemma:alC_delta-1} e \ref{lemma:alL_delta-2}, é suficiente considerar o caso em que $\nextal \neq \min\{\nextal_{C},\nextal_{L} \}$.  Considerando as definições dadas na Equação~\eqref{eq:defin-zeta-chi}, podemos rescrever \eqref{eq:theta} como
% \[
% \theta(\al) =  \dfrac{\al\left[ \nu\dbvec{\rho_L}_0 + (1-\eta)\dbvec{\rho_C} - \al\left(\zeta + \chi
% \right) \right]}{\nu\dbvec{\rho_L}_0 +
% \dbvec{\rho_C}}.
% \]
	
% 	%FIXME Verificar
% Note que \[\theta'(0) = 1 - \frac{\eta\dbvec{\rho_{C}}}{\nu\dbvec{\rho_{L}}_{0}+\dbvec{\rho_{C}}}.\]
% Se 
% \[
% 0< \eta < \dfrac{\nu\dbvec{\rho_{L}}_{0}}{\dbvec{\rho_{C}}} + 1 % \leq \frac{\be}{\dbvec{\rho_{C}}_{0}}  + 1,
% \]
% então $\theta'(0)>0$ e por continuidade, temos que é possível escolher $\al_k\in(0,1]$ tal que 
% \[
% 0 < \theta_k = \theta(\al_k) < 1.
% \]

% Portanto, vamos estabelecer um limitante inferior para $\nextal$. Teremos dois casos a considerar. 
% \begin{enumerate}[{Caso} (i):]
% 	\item $(\zeta + \chi) > 0$. Neste caso, $\theta(\al)$ é uma quadrática com concavidade voltada para baixo. Assim, o seu máximo será atingido no vértice, isto é, em
% 	\[
% 		\nextal = \frac{\nu\dbvec{\rho_L}_0 + (1-\eta)\dbvec{\rho_C}}{2(\zeta + \chi)}.
% 	\]

% 	Note que, neste caso, $\zeta + \chi = \abs{\zeta + \chi} \leq \abs{\zeta} + \abs{\chi}$.  Utilizando as Equações \eqref{eq:bound-zeta} e \eqref{eq:bound-chi}, vemos que  

% 	\[
% 		\nextal \geq \left( \frac{\nu\dbvec{\rho_L}_0/\dbvec{\rho_C} + (1-\eta)}{2( \omega_1^{2}/n  + \tfrac{1}{2}\omega_{3})}\right)\dfrac{1}{n^{2}}.
% 	\]
% 	\color{red}
% 	Sei que \[ \frac{\norm{\rho_{L}}}{x^{T}z} = \frac{\dbvec{\rho_{L}}}{\dbvec{\rho_{C}}}=  \frac{\nu\dbvec{\rho_L}_0}{\dbvec{\rho_C}} \leq \frac{\beta}{\dbvec{\rho_C}_0} \] para toda iteração, por conta da vizinhança $\Nset_{\infty}(\ga,\be)$. Como garantir que o limitante de $\nextal$ seja grande o suficiente? Posso considerar que $\rho_{L}\to 0$ antes que $\rho_{C}\to 0$. Nesse caso, posso considerar que se eu escolher $\be\geq 1$ tal que $\frac{\beta}{\dbvec{\rho_C}_0} \leq 1$, então temos um limitante suficiente? 


% 	\color{black}
% 	\item $(\zeta + \chi) < 0$. Agora, $\theta(\al)$ é uma quadrática com concavidade voltada para cima. Com isso, o seu máximo será atingido em algum ponto à direita da única raiz não nula, isto é, em $\nextal$, tal que 
% 	\[
% 		\nextal \geq  \frac{\nu\dbvec{\rho_L}_0 + (1-\eta)\dbvec{\rho_C}}{(\zeta + \chi)}.
% 	\]
% \end{enumerate}
% \end{proof}





Finalmente podemos enunciar e provar o teorema de convergência do Algoritmo. 


\begin{teo}[Convergência do Algoritmo \ref{alg:optimized-choice-of-parameters-simplified}]
	\label{teo:alg-convergence-varphi} Seja $0 < \varepsilon <1$ dado. Suponha que a sequência $\{(x^{k},y^{k},z^{k})\}$ gerada pelo Algoritmo~\ref{alg:optimized-choice-of-parameters-simplified} é  tal que 
	para o ponto inicial $(\xzero,\yzero,\zzero)$ seja válido	
	\begin{equation}
	\label{eq:varphi0-eps-kappa}
		\varphi_{0}\leq \dfrac{1}{\varepsilon^{\kappa}}
	\end{equation} 
	com  $\kappa$ uma constante positiva. Então existe um índice \[K =  \Oset\left(n^{4}\abs{\ln\frac{1}{\varepsilon}}\right),\] tal que 
	 $\varphi_{k}\leq \varepsilon \text{ para todo } k \geq K$.
	 
\end{teo}

\begin{proof} 
	Aplicando o logaritmo em ambos os lados da inequação \eqref{eq:varphi-delta-n4} dada no Lema \ref{lemma:next-phi-delta-n4}, obtemos
	\[
	\ln\varphi_{k+1}\leq \ln \left(1 - \frac{\hat{\delta}}{n^{4}}\right) + \ln\varphi_{k},
	\]
Repetindo tal procedimento na fórmula acima e utilizando \eqref{eq:varphi0-eps-kappa} segue que
\[
	\begin{aligned}
		\ln\varphi_{k} & \leq k \ln\left(1 - \frac{\hat{\delta}}{n^{4}}\right) + \ln \varphi_{0} \\
					& \leq k \ln\left(1 - \frac{\hat{\delta}}{n^{4}}\right) + \kappa\ln \frac{1}{\varepsilon}. 
	\end{aligned}
\] 

Conforme \textcite[Lema 4.1, pg 68]{Wright:Primal-dual-interior-point:1997h}, temos que $\ln(1+r) \leq r$, sempre que $r>-1$. Assim
\[
	\ln\varphi_{k}\leq k \left(- \frac{\hat{\delta}}{n^{4}}\right) + \kappa\ln \frac{1}{\varepsilon}.
\] 

Para que o critério de convergência $\varphi_{k}\leq\varepsilon$ seja satisfeito, devemos garantir que 
\[
	k \left(- \frac{\hat{\delta}}{n^{4}}\right) + \kappa\ln \frac{1}{\varepsilon} \leq \ln\varepsilon.
\] 
De fato, tal inequação é válida para 
\[
	k \geq \dfrac{n^{4}}{\hat{\delta}}(1+\kappa)\ln\frac{1}{\varepsilon},
\]
o que termina a demonstração.
\end{proof}




O Teorema \ref{teo:alg-convergence-varphi} implica que para um $k$ suficientemente grande, o valor de $\varphi$ será tão pequeno quanto se deseja. No entanto, tal resultado ainda não implica que algum critério de parada usual de Métodos de Pontos Interiores é satisfeito. Porém, é possível garantir que o Critério de Parada do \texttt{PCx}  dado na Equação~\eqref{eq:termination-criteria-pcx}, seja satisfeito, para uma escolha particular de $\varepsilon$. Tal garantia é dada pelo Corolário a seguir.


\begin{corol}
Suponha que escolha-se  $\tol = 10^{-8}$  e $\varepsilon$   tal que 
\begin{equation}
	\label{eq:choosing-eps}
		\varepsilon < \min\left\{  \frac{\tol(1+\norm{b})}{m+n}, \frac{\tol(1+\norm{c})}{m+n},\tol(1+ |c^{T}z|)\right\}.
\end{equation}
Então se o Algoritmo \ref{alg:optimized-choice-of-parameters-simplified} converge então o critério de parada dado na Equação~\eqref{eq:termination-criteria-pcx} é satisfeito.
\end{corol}		

\begin{proof}
	 Pelas Definições \ref{def:residual-vector} e \ref{def:merit-function}, vale
	\begin{equation}
		\label{eq:varphi-rhoP-rhoD}
				\varphi_{k} =  \frac{\norm{\rho^{k}_L}_1}{m+n} + 
\frac{(x^{k})^Tz^{k}}{n}  = \frac{\norm{H_{P}(Ax^{k} - b)}_1 +\norm{H_{D}(A^{T}y^{k} + z^{k}- c)}_1 }{m+n} + 
\frac{(x^{k})^Tz^{k}}{n}.
	\end{equation}
Além disso, temos que $\norm{Ax^{k} - b} \leq \norm{Ax^{k} - b}_{1} = \norm{H_{P}(Ax^{k} - b)}_{1}$ e que $\norm{H_{D}(A^{T}y^{k} + z^{k}- c)} \leq \norm{A^{T}y^{k} + z^{k}- c}_1 = \norm{H_{D}(A^{T}y^{k} + z^{k}- c)}_1$.

Ademais, se  o Algoritmo \ref{alg:optimized-choice-of-parameters-simplified} converge, existe $k$ suficientemente grande tal que $\varphi_{k}<\eps$. Assim, cada um dos termos da última parte de \eqref{eq:varphi-rhoP-rhoD} é menor que $\varepsilon$.


Daí, pela Equação \ref{eq:choosing-eps}, segue que 
\[
\dfrac{\norm{Ax^{k} - b}}{1 + \norm{b}} \leq  \frac{\norm{H_{P}(Ax^{k} - b)}_1  }{1 + \norm{b}} < \frac{\eps (m+n)}{1 + \norm{b}} <\frac{\tol(1+\norm{b})}{m+n} \frac{m+n}{1 + \norm{b}} = \tol,
\]

\[
\dfrac{\norm{A^{T}y^{k} + z^{k}- c}}{1 + \norm{c}} \leq  \frac{\norm{H_{D}(A^{T}y^{k} + z^{k}- c)}_{1}  }{1 + \norm{c}} < \frac{\eps (m+n)}{1 + \norm{c}} <\frac{\tol(1+\norm{c})}{m+n} \frac{m+n}{1 + \norm{c}} = \tol,
\]


\[
\dfrac{(x^{k})^{T}z^{k}/n}{1+ |c^{T}z^{k}|} <   \frac{\eps}{1+ |c^{T}z^{k}|} <\frac{\tol(1+ |c^{T}z^{k}|)}{1+ |c^{T}z^{k}|} = \tol.
\]


 \end{proof}

% \subsection{Um limitante para \texorpdfstring{$\al_k$}{o tamanho do passo} alternativo }

% \color{red} 
% A fim de que o Algoritmo~\ref{alg:optimized-choice-of-parameters-simplified} esteja bem definido, é necessário que exista para cada iteração $k$ uma tripla $(\al_k,\mu_k,\sig_k)$, de modo que seja possível encontrar um próximo ponto $(\nextx,\nexty,\nextz)$. Com efeito, considerando que fixamos os valores de $\mu$ e $\sig$ como $\bar{\mu} $ e  $\bar{\sig}$, basta encontrar um o tamanho de passo ${\al_k}>0$ tal que o próximo ponto $(\nextx,\nexty,\nextz)$ satisfaça as restrições da vizinhança 
% $\Nset_{-\infty}(\gamma,\beta)$ e além disso, garanta que  $0 < \theta(\al_k) <1$. 

% Para tanto, considere-se novamente que vamos fixar $\mu$ e $\sig$ com  as seguintes escolhas $ \mu = \bar{\mu} = \eta\dbvec{\rho_C} $ e $\sig = \bar{\sig} \in[\sig_{\min},\sig_{\min}]$ a definir.

%  Primeiramente, utilizando a Equação \eqref{eq:simplified-merit-function-al}, é possível rescrever a função $g_C^i $, para $i=1,\ldots,n$, que foi dada em \eqref{eq:g-Ci_explicit}, somente dependendo de uma escolha de $\al$. De fato, tal função pode ser escrita escrita nos seguintes termos:
% \begin{multline*}
% {g}_C^i (\al)	 = (1-\al)(\rho_C)_i+ \al\left(\eta\dbvec{\rho_C}\right) +  \al(\al - \barsig) ({L_{0,0}})_i + 
% 				\\ + \al^2\left[  (\eta\dbvec{\rho_C})^2(L_{2,0})_i + \eta\dbvec{\rho_C} ({L_{1,0}})_i +	\barsig\eta\dbvec{\rho_C} ({L_{1,1}})_i		+	    \barsig ({L_{0,1}})_i + \barsig^2(L_{0,2})_i \right]  + \\
% 				 -\ga\left[  (1-\al)\dbvec{\rho_C} + \al\left(\eta\dbvec{\rho_C} + (\al- \barsig) \dbvec{L_{0,0}} \right) + \al^2\left( \eta\dbvec{\rho_C} \dbvec{L_{1,0}} + \barsig\eta\dbvec{\rho_C} 
% 				 \dbvec{L_{1,1}}	+	    \barsig \dbvec{L_{0,1}} \right)  \right].
% \end{multline*}

% Definindo as constantes
% \[
% \begin{aligned}
% \zeta_i & = (L_{0,0})_i - \ga \dbvec{L_{0,0}}, \\
% \chi_i  & = \eta\dbvec{\rho_C} \left( ({L_{1,0}})_i - \ga\dbvec{L_{1,0}} \right) + \barsig\left( ({L_{0,1}})_i - \ga\dbvec{L_{0,1}} \right),   \\
% \xi_i	& =  (\eta\dbvec{\rho_C})^2(L_{2,0})_i + \barsig\eta\dbvec{\rho_C} 
% 				 \left(({L_{1,1}})_i - \ga	 \dbvec{L_{1,1}}\right) + \barsig^2(L_{0,2})_i, \\ 	
% \end{aligned}
% \]
%  usando o fato de que o ponto atual pertence à vizinhança $\Nset_{-\infty}(\gamma,\beta)$ e as propriedades de módulo de um escalar, temos
% \[
% \begin{aligned}
% 	g_C^i (\al) & = \underbrace{(1-\al)((\rho_C)_i - \ga\dbvec{\rho_C})}_{\geq 0}  + (1-\ga)\eta\dbvec{\rho_C} \al + \al(\al - \sig) \zeta_i +  (\chi_i + \xi_i)\al^2  \\
% 				& \geq \left( (1-\ga)\eta\dbvec{\rho_C}  - \barsig\zeta_i\right)\al + (\zeta_i + 				\chi_i + \xi_i)\al^2 \\ 
% 				& \geq \left( (1-\ga)\eta\dbvec{\rho_C}  - {\barsig}{\zeta_i}\right)\al -  (\abs{\zeta_i} + \abs{\chi_i} + \abs{\xi_i})\al^2 \\
% 				& = \al \left[	(1-\ga)\eta\dbvec{\rho_C}  - {\barsig}{\zeta_i}	  -  (\abs{\zeta_i} + \abs{\chi_i} + \abs{\xi_i})\al	\right] \\
% 				& = h^i(\al).
% \end{aligned}
% \] 

% em que $h^i$ é uma quadrática côncava em função de $\al$ com apenas uma raiz nula. Como queremos que $\al>0$ precisamos que a outra raiz seja positiva. Para tanto, exigimos que 
% \begin{equation}
% 	\label{eq:barsig-bound1}
% 	\barsig\zeta_{i} < (1-\ga)\eta\dbvec{\rho_C}.
% \end{equation}


% \begin{itemize}
% 	\item Caso $\eta=0$, como $\barsig>0$, precisamos que $\zeta_{i}<0$, isto é, que  $(L_{0,0})_i - \ga \dbvec{L_{0,0}} < 0$ ou ainda
% \[
% (L_{0,0})_i < \ga \dbvec{L_{0,0}}.
% \]
% \end{itemize}

% Dessa forma,  para  $i=1,\ldots,n$, se $h^i(\al)\geq0$ então $g_C^i(\al)\geq 0$. 

% Com efeito, se \eqref{eq:barsig-bound1} for satisfeito, a única raiz positiva de $h^i$ é dada por
% \[
% \al_{C}^i = \dfrac{(1-\ga)\eta\dbvec{\rho_C} - {\barsig}{\zeta_i}}{\abs{\zeta_i} + \abs{\chi_i} + \abs{\xi_i}}
% \]
% e $h^i(\al)\geq 0 $ sempre que $\al\in[0,\al_{C}^i].$ 


% Assim, seja
% \[
% \al_C = \min_i\{\al_C^i\}.
% \] 
% Para $i=1,\ldots,n$, teremos $g_C^i(\al)\geq 0$, quando  
% $\al\in[0,\al_C]$. 



% \[
% \sig<0 \text{ and } \xi>0 \text{ and } \frac{1}{\sig} <\zeta\leq 0 \text{ and } 0<\ga<1-\zeta \sig \text{ and } -\frac{(\zeta \sig)}{(\ga-1)}<\eta\leq 1 \text{ and } \al =\frac{(\zeta \sig+\eta (\ga-1))}{\xi}
% \]




% Por outro lado, us\text{ and }o as mesmas substituições de $\bar{\mu}$ e $\bar{\sig}$, a função $g_L$, dada em \eqref{eq:g-L_explicit} torna-se uma função que depende apenas de $\al$, nos seguintes termos 
% \[
% g_L(\al) =     (1-\al)\left(\dbvec{\rho_C} -  \be_L \nu   \right) +  \al\eta\dbvec{\rho_C} + 
%    \al^2\left( \dbvec{L_{0,0}} + \eta\dbvec{\rho_C}  \dbvec{L_{1,0}}   \right ) ,
% 	\]
% Usando novamente o fato de que o ponto atual pertence à  vizinhança $\Nset_{-\infty}(\gamma,\beta)$ e definindo as seguintes constantes 
% \[
% \begin{aligned}
% \zeta & =  \dbvec{L_{0,0}}, \\
% \chi  & = \eta\dbvec{\rho_C} \dbvec{L_{1,0}},
% \end{aligned}
% \]
% segue que 
%  \[
% \begin{aligned}
% {g}_L(\al) & =     (1-\al)\underbrace{\left(\dbvec{\rho_C} -  \be_L \nu   \right)}_{\geq 0} +  \al\eta\dbvec{\rho_C} + 
%    \al^2\left( \zeta + \chi   \right ) \\
%    & \geq  \al\left[\eta\dbvec{\rho_C} - 
%    \al (\abs{\zeta} + \abs{\chi})   \right ].
% \end{aligned}
%  \]




% Seja $\al_L\in(0,1]$ o maior número tal que  $g_L(\al)\geq 0$, para  $\al\in[0,\al_L]$. Por conta da última inequação acima, temos que se 
% \[
% \al_L \geq \frac{\eta\dbvec{\rho_C} }{\abs{\zeta} + \abs{\chi}},
% \]
%  então certamente a condição \eqref{eq:symmetric-polynomials-a} será satisfeita. 






% Por conta do que foi visto até aqui, devemos escolher o tamanho do passo $\al_k$ tal que 
% \begin{equation}
% \label{eq:bound-alpha} 
% 	\al_k \leq \bar{\al} = \arg\max \{\theta(\al):\al\in[0,\min\{\al_C,\al_L\}] \}
% \end{equation}



% \subsection{Um limitante para \texorpdfstring{$\al_k$}{o tamanho do passo} alternativo }

% \color{blue} 
% A fim de que o Algoritmo~\ref{alg:optimized-choice-of-parameters-simplified} esteja bem definido, é necessário que exista para cada iteração $k$ uma tripla $(\al_k,\mu_k,\sig_k)$, de modo que seja possível encontrar um próximo ponto $(\nextx,\nexty,\nextz)$. Com efeito, considerando que fixamos os valores de $\mu$ e $\sig$ como $\bar{\mu} $ e  $\bar{\sig}$, basta encontrar um o tamanho de passo ${\al_k}>0$ tal que o próximo ponto $(\nextx,\nexty,\nextz)$ satisfaça as restrições da vizinhança 
% $\Nset_{-\infty}(\gamma,\beta)$ e além disso, garanta que  $0 < \theta(\al_k) <1$. 

% Para tanto, considere-se novamente que vamos fixar $\mu$ e $\sig$ com  as seguintes escolhas $ \mu = 0$ e $\sig = 1$ a definir.

%  Primeiramente, utilizando a Equação \eqref{eq:simplified-merit-function-al}, é possível rescrever a função $g_C^i $, para $i=1,\ldots,n$, que foi dada em \eqref{eq:g-Ci_explicit}, somente dependendo de uma escolha de $\al$. De fato, tal função pode ser escrita escrita nos seguintes termos:
% \begin{multline*}
% {g}_C^i (\al)	 = (1-\al)(\rho_C)_i+   \al(\al - \barsig) ({L_{0,0}})_i  
% 				 + \al^2\left[  ( \barsig ({L_{0,1}})_i + \barsig^2(L_{0,2})_i \right]  + \\
% 				 -\ga\left[  (1-\al)\dbvec{\rho_C} +  \al(\al- \barsig) \dbvec{L_{0,0}}  + \al^2\left(   \barsig \dbvec{L_{0,1}} \right)  \right].
% \end{multline*}

% Definindo $\nextsig=1$, e as constantes
% \[
% \begin{aligned}
% \zeta_i & = (L_{0,0})_i - \ga \dbvec{L_{0,0}}, \\
% \chi_i  & =   ({L_{0,1}})_i - \ga\dbvec{L_{0,1}} ,   \\
% \xi_i	& =    (L_{0,2})_i, \\ 	
% \end{aligned}
% \]
%  usando o fato de que o ponto atual pertence à vizinhança $\Nset_{-\infty}(\gamma,\beta)$ e as propriedades de módulo de um escalar, temos
% \[
% \begin{aligned}
% 	g_C^i (\al) & = \underbrace{(1-\al)((\rho_C)_i - \ga\dbvec{\rho_C})}_{\geq 0}  + \al(\al - 1) \zeta_i +  (\chi_i + \xi_i)\al^2  \\
% 				& \geq  - \zeta_i\al + (\zeta_i + 				\chi_i + \xi_i)\al^2 \\ 
% 				& = h^i(\al).
% \end{aligned}
% \] 
% em que $h^i$ é uma quadrática em função de $\al$ com apenas uma raiz nula dada por  
% \[
% \bar{\al} = \frac{-\zeta_{i}}{\zeta_i + \chi_i + \xi_i}.
% \]


% Se $\zeta_i + \chi_i + \xi_i>0$, então $h^{i}$ tem concavidade voltada para cima e  exigimos que $\bar{\al}<0$. Para isso, precisamos que $\zeta_{i}>0$, isto é $(L_{0,0})_i - \ga \dbvec{L_{0,0}} > 0$ ou ainda
% \[
% (L_{0,0})_i > \ga \dbvec{L_{0,0}}.
% \]


% Caso $\zeta_i + \chi_i + \xi_i<0$, então $h^{i}$ tem concavidade voltada para baixo e  exigimos que $\bar{\al}>0$. Para isso, fazemos  $\zeta_{i}<0$, isto é $(L_{0,0})_i - \ga \dbvec{L_{0,0}} < 0$ ou ainda
% \[
% (L_{0,0})_i < \ga \dbvec{L_{0,0}}.
% \]



% COMO GARANTIR ISSO?

% \color{black}

% \section{Procurar limitantes do que?}

% \textcolor{red}{Esta seção será rescrita. Apenas está aqui para mostrar os  limitantes procurados.}


% Observe que é preciso encontrar limitantes para $ \abs{\zeta}, \abs{\chi}, \abs{\zeta_i}, \abs{\chi_i} \text{ e } \abs{\xi_i}$.
% Com efeito por conta  das definições dessas constantes acima  e do Teorema \ref{thm:next=residual} valem as seguintes observações:
% \begin{enumerate}[(i)]
% 	\item $\abs{\zeta} = \abs{\dbvec{L_{0,0}}} = \abs{\dfrac{(\dex)^T\dez}{n}} \leq \dfrac{1}{n} \norm{\dex\dez}_1$.
% 	\item Como $\sig=0$, usando a equação \eqref{eq:Corrector-spllited}, temos que 
% 	\[
% 		\Decox = \mu\Dex^\mu \text{ e } \Decoz = \mu\Dez^\mu.
% 	\]
% 	Assim,  
% 	\[
% 	\begin{aligned}
% 	\abs{\chi} & = \abs{\eta\dbvec{\rho_C} \dbvec{L_{1,0}}} = \abs{\dfrac{(\dex)^T(\bar{\mu}\Dez^\mu) + (\bar{\mu}\Dex^\mu)^T\dez}{n}   }  \\ 
% 	&  = \abs{\dfrac{(\dex)^T(\Decoz) + (\Decox)^T\dez}{n} }  \\
% 	& \leq \dfrac{1}{n} \norm{\dex\Decoz + \Decox\dez}_1.
% 	\end{aligned}
% 	\]

% \item $\abs{\zeta_i}  = \abs{(L_{0,0})_i - \ga \dbvec{L_{0,0}}} \leq \abs{(L_{0,0})_i} + \ga \abs{\dbvec{L_{0,0}}} \leq
% 					\norm{L_{0,0}}_1  + \dfrac{\ga}{n}\norm{L_{0,0}}_1 \leq 2\norm{L_{0,0}}_1 = 2\norm{\dex\dez}_1.$
% \item Utilizando as ideias de (ii) segue que  
% \[
% \begin{aligned}
% 	\abs{\chi_i }  & = \abs{\eta\dbvec{\rho_C} \left( ({L_{1,0}})_i - \ga\dbvec{L_{1,0}} \right)} \\
% 	&  \leq 
% 						\abs{\eta\dbvec{\rho_C}  ({L_{1,0}})_i} + \ga\abs { \eta\dbvec{\rho_C}\dbvec{L_{1,0}} } \\
% 						& \leq \norm{\dex\Decoz + \Decox\dez}_1 + \dfrac{\ga}{n}\norm{\dex\Decoz + \Decox\dez}_1	\\
% 						& \leq 2 \norm{\dex\Decoz + \Decox\dez}_1
% \end{aligned}
% \]

% \item Novamente, considerando $\sig=0$, usando a equação \eqref{eq:Corrector-spllited},  vale
% \[
% 	(\eta\dbvec{\rho_C})^2(L_{2,0}) = (\eta\dbvec{\rho_C}\Dex^\mu)(\eta\dbvec{\rho_C}\Dez^\mu) = \Decox\Decoz.
% \]
% Assim
% \[
% 	\abs{\xi_i}	 =  \abs{(\eta\dbvec{\rho_C})^2(L_{2,0})_i}  = \abs{(\Decox\Decoz)_i} \leq \norm{\Decox\Decoz}_1
% \]

% \end{enumerate}

% Assim precisamos de limitantes para $\norm{\dex\dez}_1$, $\norm{\dex\Decoz + \Decox\dez}_1$ e $\norm{\Decox\Decoz}_1$.


% \chi_i  & = \eta\dbvec{\rho_C} \left( ({L_{1,0}})_i - \ga\dbvec{L_{1,0}} \right),   \\
% \xi_i	& =  (\eta\dbvec{\rho_C})^2(L_{2,0})_i , \\ 

  \begin{algorithm}[htb]
 \onehalfspacing
 \caption{Método de Escolha Adiada Simplificado.}
 \label{alg:optimized-choice-of-parameters-simplified} 
\begin{algorithmic}[1]
\Procedure{ResolveLP}{$A,b,c$}
\State $(\xzero ,\yzero,\zzero ) \gets$ \Call{PontoInicial}{$A,b,c$}.
\Comment{Assegure que  $(\xzero ,\zzero )>0$ e que $\eta\in(0,1)$}
	\For {$k=1,2,\ldots$}
		\State Encontre		$((\dex)^{k},(\dey)^{k},(\dez)^{k})$ resolvendo
				\begin{equation}
				\label{eq:predictor-linear-matrix-simplified}
				\bbm A & 0 & 0 \\
				0 & A^T & I\\
				Z^k & 0 & X^k \ebm
				\bbm (\dex)^k \\ (\dey)^k \\ (\dez)^k
				\ebm = 
				\bbm -r_P^k  \\ -r_D^k \\ -r_C^k
				\ebm.
			\end{equation}
		\State 	Faça $\nextmu=\eta{(x^k)^Tz^k}/{n}$ e  $\nextsig=0$, e resolva 
		% \[((\Decox)^{k},(\Decoy)^{k},(\Decoz)^{k}) = \bar{\mu}((\Dex^{\mu})^{k},(\Dey^{\mu})^{k},(\Dez^{\mu})^{k})\]
		% resolvendo
			\begin{equation}
				\label{eq:corrector-linear-matrix-simplified}
				\bbm A & 0 & 0 \\
				0 & A^T & I\\
				Z^k & 0 & X^k \ebm
				\bbm (\Decox)^{k} \\ (\Decoy)^{k} \\ (\Decoz)^{k}
				\ebm = 
				\bbm 0  \\ 0 \\  \nextmu e %- \bar{\sig}\deX\dez
				\ebm.
			\end{equation}
		\State Encontre $\al^*$ resolvendo o subproblema de
		otimização global 	\eqref{eq:pop-subproblem}.		
		\State Escolha $\al_k = \min\{\al^*,\tilde{\al}_k\}$ com $\tilde{\al}_k$ dado por $\eqref{eq:ratio-test}$ e faça
		\[
		\begin{aligned}	
		& x^{k+1} = x^{k} + \al_k((\dex)^{k} + (\Decox)^{k} )
		\\
		& y^{k+1} = y^{k} + \al_k((\dey)^{k} + (\Decoy)^{k} )
		\\
		& z^{k+1} = z^{k} + \al_k((\dez)^{k} + (\Decoz)^{k} )
		 \end{aligned}. 
		\]		
	\EndFor
\EndProcedure
\end{algorithmic}
\end{algorithm}
