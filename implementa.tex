%!TEX root = tese.tex

\setchapterpreamble[u]{% 
\dictum[Donald Knuth]{``The best programs are written so that computing machines can perform them quickly and so that human beings can understand them clearly.''}
}

\chapter{Implementação}
\label{chap:implementa}
Neste capítulo, fazemos alguns comentários sobre a implementação do Método de Escolha Otimizada de Parâmetros. 


O Algoritmo \ref{alg:optimized-choice-of-parameters} foi implementado em  \texttt{C/C++}, utilizando como base a implementação PCx~\cite{Czyzyk:1999hk} do algoritmo preditor-corretor de \textcite{Mehrotra:1992wr}. O PCx pode ser configurado para executar correções de ordem superior de \textcite{Gondzio:1996uw} e permitimos que fossem feitas no máximo 2 correções através do arquivo de configuração, além do  refinamento da solução por gradiente conjugado presente na implementação original. Denominamos essa implementação de referência como PCx-r.

  Chamaremos nossa implementação de PCx-EOP. Com efeito, a chamada de PCx-EOP coincide com a da implementação de referência e há a introdução do código de nosso algoritmo dentro do \emph{loop} principal de PCx-r,  retornando em seguida para a rotina principal e dando prosseguimento à sua execução usual. Isso foi feito para fazermos uma comparação justa entre nossos resultados e os que o PCx-r produz. 

  Assim, herdamos todas as rotinas de preprocessamento -- dentre as quais leitura de dados, precondicionamento e escalonamento, incluindo escalonamento de~\textcite{Curtis:1972cp}, reordenação de linhas e colunas, tratamento de colunas densas e fatoração simbólica de Cholesky --, de ponto inicial, de critério de parada e saída de dados,   bem como todas as rotinas de álgebra linear -- incluindo a estratégia  de \textcite{Ng:1993uz} para resolução de sistemas lineares esparsos via fatoração de Cholesky  e refinamento por gradientes conjugados. 


  Além disso, para problemas canalizados, a estrutura de dados do PCx é construída de forma a fazer o papel da matriz $E$ e das transformações usadas em \eqref{eq:introPL-primal-bounded} e \eqref{eq:introPL-dual-bounded}, tal que os problemas canalizados tenham sua resolução executada com  as mesmas funcionalidades dos problemas não canalizados. Isso significa que todos os passos de  PCx-r e PCx-MEOP são executados de modo sem prejuízo por conta das canalizações.



Além disso, para todos os códigos foram usadas as mesmas opções de compilação e o mesmo computador. Com essa estratégia, diferenças nos tempos de CPU ou no número de iterações podem ser atribuídas apenas à implementação de cada algoritmo. 

O código utilizado foi o de \textcite{VillasBoas:2012ur,VillasBoas2013:wn} adaptado ao nosso algoritmo. 




\section{Ponto Inicial}
\label{sec:initial:point}
O ponto inicial dado por \textcite{Mehrotra:1992wr} foi genericamente descrito na seção \ref{subsec:initial-point}. Ali, mostramos que tal estratégia encontra o ponto $(\xtil,\ytil,\ztil)$, que é solução de norma mínima que satisfaz as restrições primais e
duais. Tal terna de pontos é encontrada através de
\begin{equation}
	\label{eq:intial-point-LSquare}
	\xtil = A^T(AA^T)^{-1}b, \quad \ytil = (AA^T)^{-1}Ac\quad \text{ e }
\quad \ztil = c - A^T\ytil.
\end{equation}

Além disso, o ponto é transladado para o ortante positivo 
através do uso de constantes  $\vartheta_x>0$ e $\vartheta_ z>0$ tais que  
\begin{equation}
	\label{eq:intial-point-generated}
(x^0,y^0,z^0) = (\xtil+ \vartheta_x e,\ytil,\ztil+\vartheta_z e),
\end{equation}
garantindo que $(x^0,z^0)>0$.


 A rotina \verb|InitialPoint| do PCx é responsável por gerar o ponto inicial, e portanto calcular  $\vartheta_x$ e $\vartheta_ z$. A implementação feita no PCx possui  pequenas alterações em relação ao proposto por \citeauthor{Mehrotra:1992wr}, mas que não  estão documentadas no Manual de Utilização do PCx~\cite{Czyzyk:1998vw}. Por isso, para fins de clareza, documentaremos como o ponto inicial é de fato encontrado, deixando o mérito de tal estratégia para os autores citados.

  Além disso, tal ponto inicial é utilizado em ambos PCx-r e PCx-EOP e a análise de convergência e complexidade relatada no Capítulo~\ref{chap:convergence}, em particular a Condição~\ref{cond:xzzero-xzstar}, continuam válidas ao utilizá-lo para iniciar nosso algoritmo.

O que a rotina  \verb|InitialPoint| faz é primeiramente encontrar  $(\xtil,\ytil,\ztil)$ conforme a Equação~\eqref{eq:intial-point-LSquare}. Em seguida,  são calculadas
\begin{equation}
	\label{eq:initial-point-tilde-var}
\tilde{\vartheta_x} = \max\{\displaystyle\num{-1.5}\cdot\min_{i}\{\xtil_{i}\},\num{e-2}\} \quad \text{ e }\quad  \tilde{\vartheta_z} = \max\{\displaystyle\num{-1.5}\cdot\min_{i}\{\ztil_{i}\},\num{e-2}\}.
\end{equation}

Só então é que se obtém as translações dadas por 
\begin{equation}
	\label{eq:initial-point-var-x}
\vartheta_{x} = \tilde{\vartheta_x} + \num{0.5}\cdot\dfrac{(\xtil+\tilde{\vartheta_x} e)^{T}(\ztil+\tilde{\vartheta_z} e)}{\norm{\ztil+\tilde{\vartheta_z} e}_{1}} 
\end{equation}
e
\begin{equation}
	\label{eq:initial-point-var-z}
\vartheta_{z} = \tilde{\vartheta_z} + \num{0.5}\cdot\dfrac{(\xtil+\tilde{\vartheta_x} e)^{T}(\ztil+\tilde{\vartheta_z} e)}{\norm{\xtil+\tilde{\vartheta_x} e}_{1}} 
\end{equation}
a fim de finalmente produzir o ponto inicial como em \eqref{eq:intial-point-generated}.

No trabalho original de \textcite{Mehrotra:1992wr}, o menor valor possível para  $\tilde{\vartheta_x}$ e $\tilde{\vartheta_z}$ dados pela Equação~\eqref{eq:initial-point-tilde-var} é $\num{0}$, ao invés de \num{e-2}, 	que é o valor utilizado pelo PCx original. Em nossos testes, tanto com PCx-r quanto com PCx-EOP, utilizamos \num{e-1}.  


\section{Critério de Parada}


O PCx tem  dois critérios de parada que podem ser escolhidos pelo usuário. Sua   documentação~\cite{Czyzyk:1998vw} declara que é utilizado o critério dado em \eqref{eq:termination-criteria}.   Além disso, é possível trocar o critério de parada relacionado à complementaridade, dentro do código. Conforme explanamos na Seção~\ref{subsection:termination-criteria}, uma escolha possível seria a dada na equação \eqref{eq:termination-criteria-pcx}. Porém ao invés do teste dado pela Equação~\eqref{eq:termination-criteria-pcx-gap} -- ou por \eqref{eq:termination-criteria-gap} o que seria razoável --, a implementação do PCx usa

\[
\dfrac{x^{T}z/n}{1 + \abs{c^Tx}}\leq
	\tol,
\]
em que $\tol$ é a tolerância aceita. Essa escolha tem um viés dimensional, já que divide por $n$ o \emph{gap} de complementaridade, fazendo com que o programa acuse otimalidade  antes do que deveria. Isso porque usar esse teste equivale a utilizar  o critério dado em \eqref{eq:termination-criteria-pcx-gap} na forma $x^{T}z/(1 + \abs{c^Tx})\leq
	n\cdot \tol$. 


Fizemos duas  alterações no código, no momento em que testamos o critério de parada, que nos parecem  contribuir para um melhor desempenho, pelo menos  no que diz respeito à verificação da otimalidade do ponto em questão. 

Primeiramente,  reutilizamos a rotina \verb|RecomputeDualVariables| do PCx. Tal rotina  é utilizada no PCx original somente quando já há a indicação de otimalidade e o programa saiu do \emph{loop} principal. Passamos a descrever  a rotina em questão. 

Seja $(\xtil,\ytil,\ztil)$ o ponto que o critério de parada do PCx indicou como sendo ótimo. O que a rotina faz é primeiro calcular o vetor  $t= c - A^T\ytil$. É evidente que, por conta do modo como funcionam os \ac{MPI}, deveríamos ter $t\geq0$ e, mais que isso, pela definição de problema dual, deveríamos ter $\ztil = t$. 

 O código então primeiro testa, para cada  $i = 1,\ldots,n$, se $t_{i}<0$.  Nos casos afirmativos, o programa assinala $t_{i}\leftarrow 0$. Finalmente estabelece-se que  $\ztil \leftarrow t $, o que finaliza a rotina. Basicamente isso significa que, a rotina garante que a variável dual $\ztil$, tenha o valor calculado pela definição do problema dual, e não pelo algoritmo. O teste para verificar se $t_{i}$ é negativo serve para detectar possíveis valores inconvenientes, já que $\zstar\geq 0$.

 Pois bem, a alteração que fizemos utiliza os princípios da rotina   \verb|RecomputeDualVariables| \emph{em toda iteração} $k$. Neste caso, o que fazemos é encontrar $t= c - A^T\yk$. Testamos então se para algum  $i$ ocorre $t_{i}<0$ e em caso afirmativo assinalamos $t_{i}\leftarrow 0$. Com isso, consideramos a terna $(\xk,\yk,t)$ com a qual realizamos os testes de otimalidade dados por
 \eqref{eq:termination-criteria-pcx}, porém com as seguintes modificações: o critério \eqref{eq:termination-criteria-pcx-dual} é transformado em
\[\dfrac{\norm{t}}{1 + \norm{c}} \leq \tol,\]	
e o critério  \eqref{eq:termination-criteria-pcx-gap} é transformado em 
\[
	\dfrac{(\xk)^{T}t}{1 + \abs{c^Tt}}\leq\tol.
\]
Se esses testes são satisfeitos, então assinalamos otimalidade no algoritmo e saímos do \emph{loop} principal. 
Isso é feito porque,  como $t$ não é mais interior, $(\xk)^{T}t$ tem possivelmente valor menor que $(\xk)^{T}\zk$. Essa estratégia diz que $(\xk,\yk,t)$ pode ser um ponto melhor que $(\xk,\yk,\zk)$.  

Caso algum teste  não seja satisfeito, continuamos a execução do programa com o ponto atual sendo $(\xk,\yk,\zk)$. Em ambos os casos $t$ é descartado e o ponto utilizado é sempre $(\xk,\yk,\zk)$. Considere que se tivermos otimalidade, \verb|RecomputeDualVariables| será chamada -- pois estamos fora do \emph{loop} e o programa assinala $(\xk,\yk,\zk)\leftarrow (\xk,\yk,t)$. 

Uma outra alteração  que fizemos no teste de parada tem relação com o escalonamento dos dados. No manual do PCx~\cite[p. 20]{Czyzyk:1998vw},  os problemas \texttt{brandy}, \texttt{ganges}, \texttt{grow15}, \texttt{grow22}, \texttt{perold}, \texttt{pilot.ja}, \texttt{pilot}, \texttt{pilot4}, \texttt{pilot87}, \texttt{pilotnov} e \texttt{scfxm1} estão indicados como problemas para os quais houve convergência. No entanto, o resíduo relativo de infactibilidade primal-dual é maior que \num{e-8}. Isso significa que o PCx parou antes do que deveria. De fato, descobrimos que  o teste de parada, que é feito com os dados escalonados, era válido, porém, ao desescalonar, o critério já não era mais satisfeito. Além disso, obviamente o PCx faz o relatório de saída com os dados desescalonados. 

A fim de evitar isso, usamos a rotina \verb|ComputeUnscaledInfeasibilities| que testa a infactibilidade relativa dos dados desescalonados. Essa rotina tem como entrada cópia dos ponteiros da solução atual do PCx e logo não estamos modificando tal solução. 

O teste de infactibilidade primal-dual é feito então com os dados desescalonados, através da mesma rotina do PCx que  faz o desescalonamento no final do algoritmo. Se para os dados desescalonados o critério for satisfeito, então indicamos otimalidade. Caso contrário, continuamos a execução do algoritmo, com os dados escalonados. 


O critério de parada usado  na implementação em PCx-r e PCx-EOP é o dado na Equação~\eqref{eq:termination-criteria-pcx}, porém com  as alterações relatadas acima. Note que PCx não convergia para \texttt{greenbeb} e \texttt{wood1p}, porém, com essas alterações, PCx-r converge para ambos os problemas. 




% \begin{itemize}
% \item Descrever a heurística utilizada para resolver o subproblema de otimização
% global de polinômios. Artigo escrito em conjunto com os orientadores está sendo
% finalizado para submissão em que tal subproblema também aparece, porém num
% método similar. Tal trabalho contempla também uma biblioteca para resolver o
% subproblema de otimização de polinômios.
% \begin{itemize}
%   \item Este método similar, já em fase final de implementação,
%   demonstra-se competitivo com o \texttt{PCx} nos testes preliminares
%  \end{itemize}
% \end{itemize}





\section{Solução do subproblema de otimização}

Em cada iteração de nosso método, precisamos resolver o problema dado em \eqref{eq:pop-subproblem}, isto é, 
\begin{equation}
	\label{eq:pop-subproblem-1}
	\begin{array}{lc}
\displaystyle \min_{(\al,\mu,\sig)} & \hat\varphi(\al,\mu,\sig) \\
\text{s. a.} &\begin{cases} g_C^i(\al,\mu,\sig) \geq 0 \quad \forall i = 1,\ldots,n \\
				g_L(\al,\mu,\sig)   \geq 0 	\\
				 0\leq (\al,\mu,\sig) \leq u,
				 	
				 \end{cases}.
\end{array}
\end{equation}
em que $\varphi$ bem como $g_{C}^{i}$ e $g_{L}$ são polinômios de grau total 6 nas variáveis $(\al,\mu,\sig)$. Este é um  Problema de Otimização de Polinômios (POP) e é considerado um problema não linear  difícil de ser resolvido~\cite{Laurent:2010kp}.

Testamos algumas  implementações especializadas em POP, dentre elas GloptiPoly~\cite{Henrion:2009eb} e SaparsePOP~\cite{Waki:2008ie}, as quais se baseiam na transformação do POP em um problema de Programação Semidefinida~\cite{Lasserre:2001fw}. Nenhuma delas, entretanto, foi capaz de resolver os subproblemas, em particular pela quantidade de restrições, que é  $n+1$.  


O subproblema foi resolvido utilizando a implementação de \textcite{VillasBoas:2012ur,VillasBoas2013:wn}, especialmente desenvolvida para resolver um problema com as mesmas características que o nosso.  Tal estratégia é brevemente resumida a seguir, adaptada à nossa notação.

Seja
\[\Omega = \left\{(\al,\mu,\sig): g_C^i(\al,\mu,\sig) \geq0, \forall i = 1,\ldots,n,\: g_L(\al,\mu,\sig)   \geq 0,\:		 0\leq (\al,\mu,\sig) \leq u\right\}\]
o conjunto factível de~\eqref{eq:pop-subproblem-1}. Note que pelas demonstrações que fizemos no Capítulo \ref{chap:convergence}, $\Omega$ é não-vazio. Considere então o problema alternativo
\begin{equation}
		\label{eq:pop-subproblem-2}
\displaystyle  \min_{\mu}\left\{  \min_{\sig} \left\{ \min_{\al} \left\{ \nextphi(\al,\mu,\sig):    (\al,\mu,\sig) \in \Omega  \right\} \right\}\right\}, \\
\end{equation}
ou mais convenientemente 
\[
\displaystyle \min_{\mu}\Psi(\mu),
\]
em que  $\displaystyle \Psi = \min_{\sig}\Phi(\mu,\sig)$ e 
\[
\Phi(\mu,\sig) = \left\{
	\begin{array}{l}
\displaystyle \min_{\al}  \nextphi(\al,\mu,\sig) \\
\text{s. a. }\:  (\al,\mu,\sig) \in\Omega
\end{array}\right..
\]
  
Primeiramente, vamos verificar a relação existente entre os mínimos de \eqref{eq:pop-subproblem-1} e de \eqref{eq:pop-subproblem-2}. Sejam $(\al^*,\mu^*,\sig^*)$ uma solução global do  primeiro problema e $(\bar{\al}, \bar{\mu},\bar{\sig})$ uma solução global de segundo. Então por definição
$ \nextphi(\al^*,\mu^*,\sig^*) \leq \nextphi (\bar{\al},\bar{\mu},\bar{\sig}).$
Por outro lado 
\[
\nextphi(\baral,\barmu,\barsig) = \min_{\al}\{ \min_{\sig}\Phi(\barmu,\barsig)\} \leq \Phi(\mu^{*},\sig^{*}) \leq \nextphi(\al^*,\mu^*,\sig^*).
\]
Assim, os dois problemas têm o mesmo mínimo. A segunda formulação, entretanto, permite encontrar aproximações numéricas do minimizador de $\nextphi$ de maneira mais fácil. 

Para esse problema alternativo, um gráfico típico de $\Phi(\mu,\sig)$ é dado na Figura \ref{fig:bnl1-Phi}, tomada na iteração \num{4} do problema \texttt{bnl1}. O aspecto desse gráfico --  com um \emph{zoom} conveniente perto do mínimo --  acaba se repetindo  em todos os problema da \Netlib-108 que estudamos, o que sugere que $\Phi$ tenha derivadas contínuas por partes próximo ao mínimo.

\begin{figure}[htbp]
\centering
\includegraphics[width=.5\textwidth]{figuras/bnl1-phi}
  	\caption{\label{fig:bnl1-Phi} Gráfico de $\Phi(\mu,\sig)$ para $\al$ fixo de \texttt{bnl1}.}
  \end{figure}

 Uma característica interessante desse problema alternativo é que, em quase todos os problemas e iterações, a restrição de $\Omega$ que estava ativa ao final da resolução do subproblema é exatamente aquela relacionada ao índice de bloqueio indicado pelo teste da razão.

Primeiramente considere  um malha construída para o intervalo $[0,u_{\mu}]$ no qual  $\mu$ está definido. Usando o fato de que o índice de bloqueio do teste da razão tem relação com a restrição ativa, uma partição $\mathcal{P} = 0<\sig_{1}<\sig_{2}<\cdots<u_{\sig}$ do intervalo $[0,u_{\sig}]$, para  a  qual $\sig$ está definido, é construída tal que se $\barsig\in I_{j} = [\sig_{j},\sig_{j+1}]$, então existe um índice $t$ tal que para este $\barsig$,  os testes da razão para as componentes não negativas das variáveis $(x,z)$ podem ser substituídos por um único teste da razão, usando o índice $t$. 

Dessa forma, para cada subintervalo $I_{j} = [\sig_{j},\sig_{j+1}]$ define-se o teste da razão principal. Esse teste da razão independente de $\sig$ é, segundo \textcite{VillasBoas:2012ur,VillasBoas2013:wn}, a ferramenta principal para conseguir resolver o subproblema na medida em que define a partição $\mathcal{P}$.

Esse índice $t$ permite que escolhamos a restrição $g_{C}^{t}(\al,\sig)\geq0$, para algum índice $t$ que é mais restritiva que o teste da razão,   sendo  então declarada como \emph{restrição principal}  de $I_{j}$. De fato, a partição $\mathcal{P}$ agora é reconstruída tal que cada ponto na fronteira do subintervalo está associado com a interseção de restrições principais, gerando então uma região de relevância.

Essa estratégia leva a descartar restrições que são irrelevantes, reduzindo o número das que precisam ser utilizadas  na solução do subproblema.  Para cada intervalo $I_{j}$, determina-se um tamanho de passo máximo $\al^{\max}_{j}$ e o mínimo de $\nextphi$ em cada intervalo. Isso pode ser feito de maneira analítica usando condições de primeira ordem, mas na implementação atual tal solução é encontrada usando \emph{splines} cúbicos. Finalmente, o menor dos $\nextphi$ encontrados para cada intervalo é declarado como ótimo.

Na prática, esse método funciona porque o tamanho da partição é muito menor que o tamanho dos problemas -- para todas as iterações do conjunto \Netlib-108, a partição tem em média tamanho \num{3.7} com desvio padrão \num{2.9} e máximo \num{39}. De fato, o tempo  de CPU gasto na solução do subproblema  para toda \Netlib-108 foi \num{1,37}\% do tempo total. Um detalhe é que, em grande parte dos casos, o melhor resultado de $\nextphi$ foi encontrado com $\mu=0$.



