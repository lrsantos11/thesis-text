\begin{center}
  \large{\textbf{Resumo}}
\end{center}

Nesta tese, propomos um método de pontos interiores do tipo preditor-corretor para programação linear em um contexto primal-dual, em que o próximo iterado será escolhido através de um subproblema de minimização de uma função de mérito polinomial a três variáveis: a primeira variável é o tamanho de passo, a segunda define a trajetória central e a última modela o peso que uma direção corretora deve ter.  A minimização da função de mérito é feita  sujeitando-a à restrições  definidas por uma vizinhança da trajetória central que permite passos largos. Dessa maneira, combinamos diferentes direções, tais como preditora, corretora e de centralização com o objetivo de produzir uma direção melhor. O método proposto generaliza grande parte dos métodos de pontos interiores preditores-corretores, a depender da escolha do valor das variáveis  acima descritas. É feita, então uma análise de convergência do método proposto, considerando um ponto inicial que tem bom desempenho na prática, e que resulta em convergência linear dos iterados em complexidade polinomial. São feitos experimentos numéricos, utilizando o conjunto de testes Netlib, que mostram que essa abordagem é competitiva, quando comparada a implementações de pontos interiores bem estabelecidas como o PCx.


% FIXME Remover deste ponto até a linha 14.
% Esse é o resumo. Ele não deve conter mais de 500 palavras. Uma maneira
% fácil de obter uma boa aproximação do número de palavras do seu resumo é:
% \begin{lstlisting}
% $ detex -l resumo.tex | wc -l
% \end{lstlisting}

% As palavras chaves permitidas podem ser consultadas na
% \href{http://acervus.unicamp.br/}{Base Acervus}, buscando pelo campo
% \emph{Assunto}.
% TODO Inserir o resumo em português aqui.

\vspace{.2cm}
\textbf{Palavras-chave}:
% FIXME Remover a linha abaixo.
Programação Linear, Métodos de Pontos Interiores.
% TODO Inserir as palavras-chave aqui.
